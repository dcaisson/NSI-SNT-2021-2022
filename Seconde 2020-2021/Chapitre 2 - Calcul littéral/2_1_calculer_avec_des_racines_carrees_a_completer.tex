\documentclass[handout]{beamer}

% Lignes réponses
\usepackage{pgffor} % pour la commande \foreach permettant de réaliser une boucle
\newcommand{\pointilles}{{\\\rule{0pt}{1pt}\dotfill\rule{0pt}{1pt}}}
\newcommand{\rep}[1]{\foreach \n in {1,...,#1} {\pointilles}}

% Commandes pour cacher/révéler du texte facilement à l'aide d'un booléen
\usepackage{xstring}
\usepackage{ifthen}

\newboolean{reveal}
\setboolean{reveal}{false}

\newlength{\stextwidth} % une nouvelle longueur

\newcommand\x{6}

\newcommand{\guess}[1]{\ifthenelse{\boolean{reveal}}{{\color{red}#1}}{\settowidth{\stextwidth}{#1}\makebox[\stextwidth]{\dotfill}}}

\newcommand{\guessmath}[1]{\ifthenelse{\boolean{reveal}}{\textcolor{red}{#1}}{\settowidth{\stextwidth}{$#1$}\makebox[1.9\stextwidth]{\dotfill}}}

\newcommand{\guessmathbin}[1]{\ifthenelse{\boolean{reveal}}{\mathbin{\color{red}#1}}{\settowidth{\stextwidth}{$#1$}\makebox[2\stextwidth]{\dotfill}}}

% ========================================================================%

\usetheme{focus}

\usepackage{pgfpages}
\pgfpagesuselayout{4 on 1}[a4paper,landscape]

\usepackage[french]{babel}

\usepackage{xcolor}

\usepackage{pstricks,pst-plot,pst-text,pst-tree,pst-eps,pst-fill,pst-node,pst-math}
\usepackage{pstricks-add,pst-xkey}

\input ../tabvar

\usepackage{multicol}
\usepackage[np]{numprint}

\begin{document}

\title{}

\date{}

\begin{frame}
  \frametitle{Calculer avec des racines carrées}
  \textbf{Définition. --} Soit $a$ un nombre réel positif ou nul. La racine carrée de $a$, notée $\guessmath{\sqrt{a}}$, est l'unique nombre réel positif ou nul dont le \guess{carré} vaut $a$.

  \bigskip

  \textit{Exemples. -- On a ainsi :
    \begin{center}
      $\sqrt{16}=\guessmath{4\;}$, $\sqrt{0}=\guessmath{0\;}$ et $\sqrt{1}=\guessmath{1\;}$.
    \end{center}
  }
\end{frame}

\begin{frame}
  \textbf{Proposition. --} Quels que soient les réels $a$ et $b$ positifs :
  \[
    \begin{aligned}
      \left(\sqrt{a}\right)^2 &= \guessmath{a\;}\\
			      &\\
      \sqrt{a\times b} &= \guessmath{\sqrt{a}}\times\guessmath{\sqrt{b}}\\
			      &\\
      \sqrt{\dfrac{a}{b}} &= \dfrac{\guessmath{\sqrt{a}}}{\guessmath{\sqrt{b}}}
    \end{aligned}
  \]

  \textit{Exemples. -- Compléter les lignes suivantes :
    \begin{itemize}
      \item[--] $\sqrt{18} = \sqrt{9\times\guessmath{2}} = \guessmath{\sqrt{9}} \times \guessmath{\sqrt{2}} = \guessmath{3}\sqrt{\guessmath{2}}$
      \item[--] $\sqrt{7}\times\sqrt{5}=\sqrt{\guessmath{7}\times\guessmath{5}}=\sqrt{\guessmath{35}}$
      \item[--] $\sqrt{\dfrac{16}{9}}=\dfrac{\sqrt{\guessmath{16}}}{\sqrt{\guessmath{9}}}=\dfrac{\guessmath{4}}{\guessmath{3}}$
    \end{itemize}
  }
\end{frame}

\begin{frame}
  \textbf{Proposition. --} Quel que soit le réel $a$ positif ou nul :
  \[\sqrt{a^2}=
      \begin{cases}
	a &\text{si $\guessmath{a\geq 0}$}\\
	-a &\text{si $\guessmath{a<0}$}
      \end{cases}\]

  On a donc, pour tout réel $a$ positif ou nul :

  \[\sqrt{a^2}=\guessmath{|a|}.\]
\end{frame}

\begin{frame}
  \textbf{Proposition. --} Quels que soient les réels $a$ et $b$ strictement positifs :
  \[\sqrt{a+b}<\guessmath{\sqrt{a}}+\guessmath{\sqrt{b}}.\]
\end{frame}

\end{document}

%%% Local Variables:
%%% mode: latex
%%% TeX-master: t
%%% End:
