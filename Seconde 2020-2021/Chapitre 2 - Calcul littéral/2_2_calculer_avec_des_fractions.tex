\documentclass[handout]{beamer}

% Lignes réponses
\usepackage{pgffor} % pour la commande \foreach permettant de réaliser une boucle
\newcommand{\pointilles}{{\\\rule{0pt}{1pt}\dotfill\rule{0pt}{1pt}}}
\newcommand{\rep}[1]{\foreach \n in {1,...,#1} {\pointilles}}

% Commandes pour cacher/révéler du texte facilement à l'aide d'un booléen
\usepackage{xstring}
\usepackage{ifthen}

\newboolean{reveal}
\setboolean{reveal}{true}

\newlength{\stextwidth} % une nouvelle longueur

\newcommand\x{6}

\newcommand{\guess}[1]{\ifthenelse{\boolean{reveal}}{{\color{red}#1}}{\settowidth{\stextwidth}{#1}\makebox[\stextwidth]{\dotfill}}}

\newcommand{\guessmath}[1]{\ifthenelse{\boolean{reveal}}{\textcolor{red}{#1}}{\settowidth{\stextwidth}{$#1$}\makebox[1.9\stextwidth]{\dotfill}}}

\newcommand{\guessmathbin}[1]{\ifthenelse{\boolean{reveal}}{\mathbin{\color{red}#1}}{\settowidth{\stextwidth}{$#1$}\makebox[2\stextwidth]{\dotfill}}}

% ========================================================================%

\usetheme{focus}

\usepackage{pgfpages}
\pgfpagesuselayout{4 on 1}[a4paper,landscape]

\usepackage[french]{babel}

\usepackage{xcolor}

\usepackage{pstricks,pst-plot,pst-text,pst-tree,pst-eps,pst-fill,pst-node,pst-math}
\usepackage{pstricks-add,pst-xkey}

\input ../tabvar

\usepackage{multicol}
\usepackage[np]{numprint}

\begin{document}

\title{}

\date{}

\begin{frame}
  \frametitle{Calculer avec des fractions}
  \textit{Remarque. -- 
    \[a = \dfrac{a}{\guessmathbin{1}}\]
    \[\dfrac{-a}{b}=\guessmath{\dfrac{a}{-b}}=\guessmath{-\dfrac{a}{b}}\]
  }

  \bigskip

  On peut ajouter ou soustraire facilement des fractions qui ont le même dénominateur :

  \bigskip

  \textit{Exemples. -- 
    \[\dfrac{5}{3}+\dfrac{8}{3}=\guessmath{\dfrac{13}{3}}\]
    \[\dfrac{2}{7}-\dfrac{12}{7}=\guessmath{-\dfrac{10}{7}}\]
  }
\end{frame}

\begin{frame}
  Pour ajouter ou soustraire deux fractions qui n'ont pas le même dénominateur, il suffit de les réduire au même dénominateur :

  \bigskip

  \textit{Exemples. --
    \[\dfrac{3}{4}+\dfrac{9}{8}=\guessmath{\dfrac{6}{8}+\dfrac{9}{8}}=\guessmath{\dfrac{15}{8}}\]
    \[\dfrac{5}{6}-\dfrac{7}{9}=\guessmath{\dfrac{15}{18}-\dfrac{14}{18}}=\guessmath{\dfrac{1}{18}}\]
  }
\end{frame}

\begin{frame}
  Pour multiplier deux fractions, on peut multiplier les numérateurs entre eux et les dénominateurs entre eux, tout en faisant attention aux simplifications possibles :

  \bigskip

  \textit{Exemples. --
    \[\dfrac{5}{6}\times\dfrac{9}{7}=\guessmath{\dfrac{5\times9}{6\times7}}=\guessmath{\dfrac{5\times3\times3}{2\times3\times7}}=\guessmath{\dfrac{15}{14}}\]
    \[\dfrac{\np{7777531}}{37}\times\dfrac{5}{\np{7777531}}=\guessmath{\dfrac{\np{7777531}\times5}{37\times\np{7777531}}}=\guessmath{\dfrac{5}{37}}\]
    }
\end{frame}

\begin{frame}
  Diviser par une fraction revient à multiplier par son inverse :

  \bigskip

  \textit{Exemples. --
    \[\dfrac{\dfrac{5}{6}}{\dfrac{9}{7}}=\guessmath{\dfrac{5}{6}\times\dfrac{7}{9}}=\guessmath{\dfrac{5\times7}{6\times9}}=\guessmath{\dfrac{35}{54}}\]
  }
\end{frame}

\end{document}

%%% Local Variables:
%%% mode: latex
%%% TeX-master: t
%%% End:
