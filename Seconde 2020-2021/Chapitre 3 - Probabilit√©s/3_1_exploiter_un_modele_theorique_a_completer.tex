\documentclass[handout]{beamer}

% Lignes réponses
\usepackage{pgffor} % pour la commande \foreach permettant de réaliser une boucle
\newcommand{\pointilles}{{\\\rule{0pt}{1pt}\dotfill\rule{0pt}{1pt}}}
\newcommand{\rep}[1]{\foreach \n in {1,...,#1} {\pointilles}}

% Commandes pour cacher/révéler du texte facilement à l'aide d'un booléen
\usepackage{xstring}
\usepackage{ifthen}

\newboolean{reveal}
\setboolean{reveal}{false}

\newlength{\stextwidth} % une nouvelle longueur

\newcommand\x{6}

\newcommand{\guess}[1]{\ifthenelse{\boolean{reveal}}{{\color{red}#1}}{\settowidth{\stextwidth}{#1}\makebox[\stextwidth]{\dotfill}}}

\newcommand{\guessmath}[1]{\ifthenelse{\boolean{reveal}}{\textcolor{red}{#1}}{\settowidth{\stextwidth}{$#1$}\makebox[1.9\stextwidth]{\dotfill}}}

\newcommand{\guessmathbin}[1]{\ifthenelse{\boolean{reveal}}{\mathbin{\color{red}#1}}{\settowidth{\stextwidth}{$#1$}\makebox[2\stextwidth]{\dotfill}}}

% ========================================================================%

\usetheme{focus}

\usepackage{pgfpages}
\pgfpagesuselayout{4 on 1}[a4paper,landscape]

\usepackage[french]{babel}

\usepackage{xcolor}

\usepackage{pstricks,pst-plot,pst-text,pst-tree,pst-eps,pst-fill,pst-node,pst-math}
\usepackage{pstricks-add,pst-xkey}

\input ../tabvar

\usepackage{multicol}
\usepackage[np]{numprint}

\usepackage{booktabs}

\begin{document}

\title{}

\date{}

\begin{frame}
  \frametitle{Exploiter un modèle théorique}
  \textbf{Définition. --} L'univers d'une expérience aléatoire est \guess{l'ensemble de toutes ses issues} c'est-à-dire l'ensemble des \guess{résultats de cette expérience aléatoire}. On le note souvent $\guessmath{\Omega}$.

  \bigskip

  \textit{Exemple. -- On lance un dé à six faces numérotées de $1$ à $6$ et on note le numéro obtenu. L'univers est : \[\Omega=\guessmath{\{1;2;3;4;5;6\}}.\]}
\end{frame}

\begin{frame}
  \textbf{Définition. --} Une loi de probabilité est définie en associant à chaque \guess{issue} une \guess{probabilité}. \guess{Modéliser} une expérience aléatoire, c'est faire le choix d'une telle loi.

  \bigskip

  \textit{Remarque. -- On présente très souvent une loi de probabilité sous forme d'un tableau. Par exemple, le tableau suivant modélise le gain d'un jeu de grattage :
    \begin{center}
      \begin{tabular}{@{}cccccc@{}}
	\toprule
	Gain (en XPF) & $0$ & $100$ & $200$ & $500$ & $\np{1000}$\\
	\midrule
	Probabilité & $0,5$ & $0,3$ & $0,1$ & $0,09$ & $0,01$\\
	\bottomrule
      \end{tabular}
    \end{center}
  }
\end{frame}

\begin{frame}
On rappelle les propositions suivantes très importantes :

\bigskip

\textbf{Proposition. --} Une probabilité est un nombre réel compris entre $\guessmath{0}$ et $\guessmath{1}$.

\bigskip

\textbf{Proposition. --} La somme des probabilités de toutes les issues d'une expérience aléatoire est égale à $\guessmath{1}$.
\end{frame}

\begin{frame}
  \textbf{Définition et proposition. --} Si toutes les issues d'une expérience aléatoire ont la même probabilité de se réaliser, alors on dit qu'il s'agit d'une \guess{situation d'équiprobabilité}.

  \smallskip

  Dans ce cas, si l'univers est composé de $n$ issues, la probabilité de chaque issue vaut $\guessmath{\dfrac{1}{n}}$.

  \bigskip

  \textit{Exemple. -- On lance un dé \textbf{équilibré} à six faces numérotées de $1$ à $6$ et on note le numéro obtenu.\\
    \smallskip
  Puisque le dé est supposé \guess{équilibré}, il s'agit d'une \guess{situation d'équiprobabilité}. La probabilité d'obtenir chaque numéro est égale à $\guessmath{\dfrac{1}{6}}$.}
\end{frame}

\begin{frame}
  \textbf{Proposition. --} La probabilité d'un événement est la somme des probabilités des \guess{issues qui le réalisent}.

  \bigskip

  \textit{Exemples. -- Déterminer la probabilité des événements \og{}obtenir un gain inférieur ou égal à $200$ XPF\fg{} (exemple du jeu de grattage) et \og{}le numéro obtenu est supérieur ou égal à $4$\fg{} (exemple du dé équilibré).\rep{10}}
\end{frame}

\begin{frame}
  \textbf{Proposition. --} \begin{itemize}
    \item[--] La probabilité d'un événement impossible (un événement qui n'est réalisé par aucune issue) vaut $\guessmath{0}$.
    \item[--] La probabilité d'un événement certain (un événement réalisé par toutes les issues) vaut $\guessmath{1}$.
  \end{itemize}
\end{frame}

\end{document}

%%% Local Variables:
%%% mode: latex
%%% TeX-master: t
%%% End:
