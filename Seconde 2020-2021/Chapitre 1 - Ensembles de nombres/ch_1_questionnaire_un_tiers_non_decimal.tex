\documentclass[a4paper,dvipsnames]{article}

\input ../header
\newcommand{\checkedbox}{\makebox[0pt][l]{$\square$}\raisebox{.15ex}{\hspace{0.1em}$\checkmark$}}
\newcommand{\checkbox}{\makebox[0pt][l]{$\square$}\raisebox{.15ex}{\hspace{0.1em}}\hspace{3mm}}

\usepackage{gensymb}

\begin{document}

\title{Questionnaire -- $1/3$ n'est pas décimal}

\date{}
\author{}

\maketitle{}

\thispagestyle{empty}
\pagestyle{empty}

\textbf{Ce questionnaire est à compléter après avoir regardé la vidéo disponible sur Moodle. Il est également possible de faire des recherches pour répondre à certaines questions.}

\bigskip

\begin{enumerate}
  \item Rappeler comment l'ensemble des nombres décimaux est noté: $\hdots\hdots\hdots$
  \item Quelle définition d'un nombre décimal est utilisée dans la vidéo ? La recopier ci-dessous.\\
    \textit{Remarque. -- Il s'agit aussi de la définition du cours.}\rep{4}
  \item En utilisant cette définition, on écrit par exemple :
    \begin{center}
      $23,46=\dfrac{\np{2346}}{100}$ donc $23,46$ est un nombre décimal.
    \end{center}
    Compléter les phrases suivantes :
    \begin{itemize}
      \item[] $\np{1,74901}=$\dotfill
      \item[] $\np{-134,561}=$\dotfill
      \item[] $\np{-23,98}=$\dotfill
    \end{itemize}
  \item Comment peut-on savoir rapidement si un nombre est divisible par $3$ ? Donner ci-dessous un critère de divisibilité par $3$ et quelques exemples d'utilisation du critère.\rep{6}
  \item Une puissance de $10$ est-elle divisible par $3$ ? Justifier la réponse.\rep{3}
  \item Comment s'appelle le type de démonstration utilisée dans la vidéo pour prouver que $\dfrac{1}{3}$ n'est pas un nombre décimal ?\rep{2}
  \item Expliquer le principe de la démonstration en complétant le texte ci-dessous.
    \begin{center}
      Pour faire une démonstration par $\hdots\hdots\hdots\hdots\hdots$ on suppose le $\hdots\hdots\hdots\hdots\hdots$ de ce qu'on veut démontrer, et on aboutit à une $\hdots\hdots\hdots\hdots\hdots$ Cela signifie que ce qu'on a supposé au départ est $\hdots\hdots\hdots$ et donc, que son $\hdots\hdots\hdots\hdots\hdots$ est vrai.
    \end{center}
\end{enumerate}

\end{document}
