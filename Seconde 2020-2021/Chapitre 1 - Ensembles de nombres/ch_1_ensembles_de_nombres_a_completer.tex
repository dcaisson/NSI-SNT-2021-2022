\documentclass[handout]{beamer}

% Lignes réponses
\usepackage{pgffor} % pour la commande \foreach permettant de réaliser une boucle
\newcommand{\pointilles}{{\\\rule{0pt}{1pt}\dotfill\rule{0pt}{1pt}}}
\newcommand{\rep}[1]{\foreach \n in {1,...,#1} {\pointilles}}

% Commandes pour cacher/révéler du texte facilement à l'aide d'un booléen
\usepackage{xstring}
\usepackage{ifthen}

\newboolean{reveal}
\setboolean{reveal}{false}

\newlength{\stextwidth} % une nouvelle longueur

\newcommand\x{6}

\newcommand{\guess}[1]{\ifthenelse{\boolean{reveal}}{{\color{red}#1}}{\settowidth{\stextwidth}{#1}\makebox[\stextwidth]{\dotfill}}}

\newcommand{\guessmath}[1]{\ifthenelse{\boolean{reveal}}{\textcolor{red}{#1}}{\settowidth{\stextwidth}{$#1$}\makebox[1.9\stextwidth]{\dotfill}}}

\newcommand{\guessmathbin}[1]{\ifthenelse{\boolean{reveal}}{\mathbin{\color{red}#1}}{\settowidth{\stextwidth}{$#1$}\makebox[2\stextwidth]{\dotfill}}}

% ========================================================================%

\usetheme{focus}

\usepackage{pgfpages}
\pgfpagesuselayout{4 on 1}[a4paper,landscape]

\usepackage[french]{babel}

\usepackage{xcolor}

\usepackage{pstricks,pst-plot,pst-text,pst-tree,pst-eps,pst-fill,pst-node,pst-math}
\usepackage{pstricks-add,pst-xkey}

\input ../tabvar

\usepackage{multicol}
\usepackage[np]{numprint}

\begin{document}

\title{Chapitre 1 -- Les ensembles de nombres}

\date{}

\maketitle{}

\begin{frame}
  \frametitle{1. Les ensembles de nombres}
  \textbf{Définition. --} Les entiers naturels ou nuls sont\pause{} les nombres \guess{entiers positifs ou nuls}.\pause{} L'ensemble des entiers naturels est noté $\guessmath{\mathbb{N}}$.\pause{}
  \[\mathbb{N}=\left\{0;1;2;\hdots\right\}\]\pause{}

  \textit{Notation. --} On écrit par exemple $2\guessmathbin{\in}\mathbb{N}$\pause{} (se lit \og{}$2$ \guess{appartient} à $\mathbb{N}$\fg{}).
\end{frame}

\begin{frame}
\textbf{Définition. --} Les entiers \guess{relatifs} sont\pause{} les \guess{entiers positifs, nuls ou} \guess{négatifs}.\pause{} L'ensemble des entiers relatifs est noté $\guessmath{\mathbb{Z}}$.\pause{}
  \[\mathbb{Z}=\left\{\hdots;-3;-2;-1;0;1;2;3;\hdots\right\}\]\pause{}

  \bigskip

  \textbf{Proposition. --} Tout entier \guess{naturel} est aussi un entier \guess{relatif} :\pause{} on dit que l'ensemble des entiers \guess{naturels} $\guessmath{\mathbb{N}}$ est \guess{inclus} dans l'ensemble des entiers \guess{relatifs} $\guessmath{\mathbb{Z}}$.\pause{} Cette inclusion se note :

  \[\mathbb{N}\guessmathbin{\subset}\mathbb{Z}\]
\end{frame}

\begin{frame}
  \textbf{Définition. --} Les nombres décimaux sont\pause{} les nombres qui s'écrivent comme quotient d'un entier (relatif) par une puissance de $10$, c'est-à-dire par $1$, $10$, $100$, $\np{1000}$ etc (ou plus généralement $10^k$ où $k$ est un entier naturel).\pause{}
  L'ensemble des nombres décimaux est noté $\mathbb{D}$.\pause{}

  \bigskip

  \textit{Exemples. --} \begin{enumerate}
    \item Par exemple, $0,2$ est un nombre décimal car on peut écrire $0,2=\dfrac{2}{10}$. Donner deux autres exemples de nombres décimaux.
    \item L'entier naturel $4$ est-il un nombre décimal ? Et l'entier relatif $-7$ ?
  \end{enumerate}
\end{frame}

\begin{frame}
  \textbf{Proposition. --} L'ensemble des entiers relatifs est inclus dans l'ensemble des nombres décimaux : $\mathbb{Z}\subset\mathbb{D}$.\pause{} On a donc :

  \[\mathbb{N}\subset\mathbb{Z}\subset\mathbb{D}\]
\end{frame}

\begin{frame}
  \textbf{Définition. --} Les nombres rationnels sont\pause{} les nombres qui s'écrivent comme le quotient de deux entiers.\pause{} L'ensemble des nombres rationnels est noté $\mathbb{Q}$.\pause{}

 \bigskip

 \textit{Exemples. --} 
 \begin{enumerate}
   \item Le nombre $\dfrac{2}{3}$ est le quotient des entiers $2$ et $3$ donc $\dfrac{2}{3}$ est un nombre rationnel.
   \item Les nombres $\dfrac{4}{7}$, $3$, $-4$ et $0,23$ sont-ils des nombres rationnels?
 \end{enumerate}
\end{frame}

\begin{frame}
  \textbf{Proposition. --} L'ensemble des nombres décimaux est inclus dans l'ensemble des nombres rationnels :\pause{} $\mathbb{D}\subset\mathbb{Q}$.\pause{} On a donc :

  \[\mathbb{N}\subset\mathbb{Z}\subset\mathbb{D}\subset\mathbb{Q}\]
\end{frame}

\begin{frame}
  \textbf{Définition. --} À chaque point de la droite graduée ci-dessous, on a associé un nombre unique, qui est appelé\pause{} son abscisse.\pause{} Inversement, à chaque nombre correspond un unique point de la droite graduée.\pause{}

  \begin{center}
    \newrgbcolor{ududff}{0.30196078431372547 0.30196078431372547 1.}
    \psset{xunit=1.0cm,yunit=1.0cm,algebraic=true,dimen=middle,dotstyle=o,dotsize=5pt 0,linewidth=1.6pt,arrowsize=3pt 2,arrowinset=0.25}
    \begin{pspicture*}(-5.,-1.)(5.,1.)
      \psaxes[labelFontSize=\scriptstyle,xAxis=true,yAxis=false,Dx=10.,Dy=1.,ticksize=-2pt 0]{->}(0,0)(-5.,-1.)(5.,1.)
      \begin{scriptsize}
	\psline[linewidth=1pt,linecolor=blue](-3.33,-0.1)(-3.33,0.1)
	\uput[u](-3.33,0){\ududff{$A$}}
	\uput[d](-3.33,-0.07){\color{blue}$-\dfrac{11}{3}$}
	\psline[linewidth=1pt,linecolor=blue](-2,-0.1)(-2,0.1)
	\uput[u](-2,0){\ududff{$B$}}
	\uput[d](-2,-0.07){\color{blue}$-2$}
	\psline[linewidth=1pt,linecolor=blue](1.73,-0.1)(1.73,0.1)
	\uput[u](1.73,0){\ududff{$C$}}
	\uput[d](1.73,-0.07){\color{blue}$\sqrt{3}$}
	\psline[linewidth=1pt,linecolor=blue](3.14,-0.1)(3.14,0.1)
	\uput[u](3.14,0){\ududff{$D$}}
	\uput[d](3.14,-0.07){\color{blue}$\pi$}
	\psline[linewidth=1pt,linecolor=blue](4.6024,-0.1)(4.6024,0.1)
	\uput[u](4.6024,0){\ududff{$E$}}
	\uput[d](4.6024,-0.07){\color{blue}${4,6024}$}
	\uput[d](0,-0.07){\color{red}$0$}
	\uput[d](1,-0.07){\color{red}$1$}
	\psline[linewidth=1pt,linecolor=red](0,-0.1)(0,0.1)
	\psline[linewidth=1pt,linecolor=red](1,-0.1)(1,0.1)
      \end{scriptsize}
    \end{pspicture*}  
  \end{center}\pause{}

  Les nombres réels sont les abscisses de tous les points d'une droite graduée.\pause{} L'ensemble des nombres réels est noté $\mathbb{R}$.
\end{frame}

\begin{frame}
  \textbf{Proposition. --} Il existe des nombres réels\pause{} qui ne sont pas rationnels, comme\pause{} $\sqrt{2}$\pause{} (il faudra savoir le démontrer) ou\pause{} $\pi$.\pause{} Ces nombres sont appelés des nombres irrationnels.
\end{frame}

\begin{frame}
  \frametitle{2. Intervalles de $\mathbb{R}$}
  \textbf{Définition. --} Soient $a$ et $b$ deux nombres réels tels que $a<b$. L'intervalle $[a;b]$ est l'ensemble des réels tels que $a\leq x\leq b$. On définit de même les intervalles $[a;b[$, $]a;b]$ et $]a;b[$.

  \bigskip

  \begin{center}
    \renewcommand{\arraystretch}{1.2}
    \begin{tabular}{|>{\centering}m{2cm}|>{\centering}m{4cm}|>{\centering}m{4cm}|}
      \hline
      \textbf{Intervalles} & \textbf{Ensemble des réels $x$ tels que \dots} & \textbf{Représentation graphique}\tabularnewline
      \hline
      $[a;b]$ & $a\leq x<b$ & \tabularnewline
      \hline
      $[a;b[$ & & \tabularnewline
      \hline
      $]a;b]$ & & \tabularnewline
      \hline
      $]a;b[$ & & \tabularnewline
      \hline
    \end{tabular}
  \end{center}
\end{frame}

\begin{frame}
  \textbf{Définition. --} Soit $a$ un nombre réel. L'intervalle $[a;+\infty[$ est l'ensemble des réels tels que $x\geq a$. On définit de la même façon les intervalles $]a;+\infty[$, $]-\infty;a]$ et $]-\infty;a[$.

  \begin{center}
    \renewcommand{\arraystretch}{1.2}
    \begin{tabular}{|>{\centering}m{2cm}|>{\centering}m{4cm}|>{\centering}m{4cm}|}
      \hline
      \textbf{Intervalles} & \textbf{Ensemble des réels $x$ tels que \dots} & \textbf{Représentation graphique}\tabularnewline
      \hline
      $[a;+\infty[$ & $x\geq a$ & \tabularnewline
      \hline
      $]a;+\infty[$ & & \tabularnewline
      \hline
      $]-\infty;a]$ & & \tabularnewline
      \hline
      $]-\infty;a[$ & & \tabularnewline
      \hline
    \end{tabular}
  \end{center}
\end{frame}

\begin{frame}
  \frametitle{3. Valeur absolue d'un nombre, distance entre deux nombres réels}
  \textbf{Définition. --} Soit $x$ un nombre réel. On appelle valeur absolue de $x$ le nombre noté $|x|$ défini par :
  \[x=\left\{
      \begin{aligned}
	x & \text{ si $x\geq 0$}\\
	-x& \text{ si $x<0$}
    \end{aligned}\right.\]

    \bigskip

    \textit{Exemples. --} Donner la valeur absolue des nombres $5$, $-2,4$, $\pi-5$ et $\dfrac{1}{7}-0,1$.
\end{frame}

\begin{frame}
  \textbf{Proposition. --} On retiendra les propriétés suivantes :
  \begin{itemize}
    \item La valeur absolue d'un nombre est positive ou nulle.
    \item Un nombre et son opposé ont la même valeur absolue.
  \end{itemize}
\end{frame}

\begin{frame}
  \textbf{Définition. --} On appelle distance entre deux réels $a$ et $b$ le nombre $|b-a|$ (qui est aussi égal à $|a-b|$). Sur une droite graduée, si $A$ est le point d'abscisse $a$ et $B$ le point d'abscisse $b$, la distance entre $a$ et $b$ est égale à la distance $AB$.

  \vspace{2cm}

  \textit{Exemples. --} Déterminer la distance entre $3$ et $-1$, puis la distance entre $-15$ et $12$.
\end{frame}

\begin{frame}
  \textit{Exemples. --} Après avoir traduit chacune des égalités et inégalités suivantes à l'aide d'une distance, représenter l'ensemble des réels $x$ tels que :
  \begin{multicols}{2}
    \begin{enumerate}
    \item $|x-4|=2$
    \item $|x-2|=3$
    \item $|x+3|=1$
    \item $|x+1|=2$
    \item $|x-3|\leq 2$
    \item $|x+7|<1$
    \item $|x-5|\geq 3$
    \item $|x+6|>1$
    \end{enumerate}
  \end{multicols}
\end{frame}

\begin{frame}
  \textbf{Proposition. --} On remarquera que l'intervalle $[a-r;a+r]$ est l'ensemble des réels $x$ tels que \dotfill

  \vspace{2cm}

  \textit{Exemples. --} Compléter chacune des phrases suivantes :
  \begin{enumerate}
    \item L'intervalle $[2;8]$ est l'ensemble des réels $x$ tels que \rep{1}
    \item L'intervalle $[2,25;6,35]$ est l'ensemble des réels $x$ tels que \rep{1}
    \item Traduire à l'aide d'une valeur absolue la condition $y\in[7,4;7,6]$.\rep{1}
  \end{enumerate}
\end{frame}

\begin{frame}
  \frametitle{4. Intersection et réunion de deux intervalles}
  \textbf{Définition. --} Soient $I$ et $J$ deux intervalles.\pause{} L'intersection de $I$ et $J$\pause{} est l'ensemble des réels qui appartiennent à $I$ et à $J$.\pause{} Cet intervalle est noté $I\cap J$\pause{} (et se lit \og{}$I$ inter $J$\fg{}).\pause{}

  \bigskip

  \textit{Exemples. --} Dans chacun des cas suivants, préciser l'intersection des intervalles $I$ et $J$.
  \begin{multicols}{2}
    \begin{enumerate}
      \item $I=[-3;5]$ et $J=]-1;7]$
      \item $I=]-3;-1]$ et $J=]-1;10[$
      \item $I=[5;10]$ et $J=]6;7[$
      \item $I=]-\infty;10[$ et $J=]-3;12]$
      \item $I=[-2;+\infty[$ et $J=]-4;6[$
      \item $I=]-10;1]$ et $J=[1;+\infty[$
    \end{enumerate}\columnbreak
    \dotfill\rep{5}
  \end{multicols}
\end{frame}

\begin{frame}
  \textbf{Définition. --} Soient $I$ et $J$ deux intervalles.\pause{} La réunion de $I$ et $J$\pause{} est l'ensemble des réels qui appartiennent à $I$ ou à $J$ (ou aux deux !).\pause{} Cet ensemble n'est pas nécessairement un intervalle : il est noté $I\cup J$ (se lit \og{}$I$ union $J$\fg{}).\pause{}

  \bigskip

  \textit{Exemples. --} Dans chacun des cas précédents, décrire le plus simplement possible la réunion des intervalles $I$ et $J$.
  \begin{multicols}{2}
    \begin{enumerate}
      \item $I=[-3;5]$ et $J=]-1;7]$
      \item $I=]-3;-1]$ et $J=]-1;10[$
      \item $I=[5;10]$ et $J=]6;7[$
      \item $I=]-\infty;10[$ et $J=]-3;12]$
      \item $I=[-2;+\infty[$ et $J=]-4;6[$
      \item $I=]-10;1]$ et $J=[1;+\infty[$
    \end{enumerate}\columnbreak
    \dotfill\rep{5}
  \end{multicols}
\end{frame}

\begin{frame}
  \frametitle{5. Encadrement décimal d'un réel}

  \textbf{Définition. --} L'encadrement décimal d'un réel $x$ à $10^{-n}$ près\pause{} (où $n$ est un entier naturel non nul)\pause{} est l'encadrement $d\leq x<d+10^{-n}$\pause{} où $d$ est un nombre décimal.\pause{}

  \bigskip

  \textit{Exemples. --} L'encadrement décimal de $\pi$ à $10^{-4}$ près est donc\pause{} $\np{3,1415}\leq\pi<\np{3,1416}$.\pause{}
  Donner l'encadrement décimal de $\sqrt{2}$ à $10^{-5}$ près, puis celui de $\dfrac{215}{368}$ à $10^{-2}$ près.
\end{frame}
\end{document}

%%% Local Variables:
%%% mode: latex
%%% TeX-master: t
%%% End:
