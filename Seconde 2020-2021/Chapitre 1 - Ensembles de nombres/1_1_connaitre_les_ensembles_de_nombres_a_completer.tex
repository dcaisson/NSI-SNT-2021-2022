\documentclass[handout]{beamer}

% Lignes réponses
\usepackage{pgffor} % pour la commande \foreach permettant de réaliser une boucle
\newcommand{\pointilles}{{\\\rule{0pt}{1pt}\dotfill\rule{0pt}{1pt}}}
\newcommand{\rep}[1]{\foreach \n in {1,...,#1} {\pointilles}}

% Commandes pour cacher/révéler du texte facilement à l'aide d'un booléen
\usepackage{xstring}
\usepackage{ifthen}

\newboolean{reveal}
\setboolean{reveal}{false}

\newlength{\stextwidth} % une nouvelle longueur

\newcommand\x{6}

\newcommand{\guess}[1]{\ifthenelse{\boolean{reveal}}{{\color{red}#1}}{\settowidth{\stextwidth}{#1}\makebox[\stextwidth]{\dotfill}}}

\newcommand{\guessmath}[1]{\ifthenelse{\boolean{reveal}}{\textcolor{red}{#1}}{\settowidth{\stextwidth}{$#1$}\makebox[1.9\stextwidth]{\dotfill}}}

\newcommand{\guessmathbin}[1]{\ifthenelse{\boolean{reveal}}{\mathbin{\color{red}#1}}{\settowidth{\stextwidth}{$#1$}\makebox[2\stextwidth]{\dotfill}}}

% ========================================================================%

\usetheme{focus}

\usepackage{pgfpages}
\pgfpagesuselayout{4 on 1}[a4paper,landscape]

\usepackage[french]{babel}

\usepackage{xcolor}

\usepackage{pstricks,pst-plot,pst-text,pst-tree,pst-eps,pst-fill,pst-node,pst-math}
\usepackage{pstricks-add,pst-xkey}

\input ../tabvar

\usepackage{multicol}
\usepackage[np]{numprint}

\begin{document}

\title{}

\date{}

\begin{frame}
  \frametitle{Connaître les ensembles de nombres}
  \textbf{Définition. --} Les entiers naturels sont\ les nombres \guess{entiers positifs ou nuls}.\ L'ensemble des entiers naturels est noté $\guessmath{\mathbb{N}}$.\
  \[\mathbb{N}=\left\{0;1;2;\hdots\right\}\]\

  \textit{Notation. --} On écrit par exemple $2\guessmathbin{\in}\mathbb{N}$\ (se lit \og{}$2$ \guess{appartient} à $\mathbb{N}$\fg{}).
\end{frame}

\begin{frame}
\textbf{Définition. --} Les entiers \guess{relatifs} sont\ les \guess{entiers positifs, nuls ou} \guess{négatifs}.\ L'ensemble des entiers relatifs est noté $\guessmath{\mathbb{Z}}$.\
  \[\mathbb{Z}=\left\{\hdots;-3;-2;-1;0;1;2;3;\hdots\right\}\]\

  \bigskip

  \textbf{Proposition. --} Tout entier \guess{naturel} est aussi un entier \guess{relatif} :\ on dit que l'ensemble des entiers \guess{naturels} $\guessmath{\mathbb{N}}$ est \guess{inclus} dans l'ensemble des entiers \guess{relatifs} $\guessmathbin{\mathbb{Z}}$.\ Cette inclusion se note :

  \[\mathbb{N}\guessmathbin{\subset}\mathbb{Z}\]
\end{frame}

\begin{frame}
  \textbf{Définition. --} Les nombres décimaux sont\ les nombres qui s'écrivent comme quotient d'un entier (relatif) par une puissance de $10$, c'est-à-dire par $1$, $10$, $100$, $\np{1000}$ etc (ou plus généralement $10^k$ où $k$ est un entier naturel).\
  L'ensemble des nombres décimaux est noté $\mathbb{D}$.\

  \bigskip

  \textit{Exemples. --} \begin{enumerate}
    \item Par exemple, $0,2$ est un nombre décimal car on peut écrire $0,2=\dfrac{2}{10}$. Donner deux autres exemples de nombres décimaux.
    \item L'entier naturel $4$ est-il un nombre décimal ? Et l'entier relatif $-7$ ?
  \end{enumerate}
\end{frame}

\begin{frame}
  \textbf{Proposition. --} L'ensemble des \guess{entiers relatifs} est \guess{inclus} dans l'ensemble des \guess{nombres décimaux} : $\mathbb{Z}\guessmathbin{\subset}\mathbb{D}$.\ On a donc :

  \[\mathbb{N}\guessmathbin{\subset}\mathbb{Z}\guessmathbin{\subset}\mathbb{D}\]
\end{frame}

\begin{frame}
  \textbf{Définition. --} Les nombres \guess{rationnels} sont\ les nombres qui s'écrivent comme le quotient de \guess{deux entiers}.\ L'ensemble des nombres rationnels est noté $\guessmathbin{\mathbb{Q}}$.\

 \bigskip

 \textit{Exemples. --} 
 \begin{enumerate}
   \item Le nombre $\dfrac{2}{3}$ est le quotient des entiers $2$ et $3$ donc $\dfrac{2}{3}$ est un nombre rationnel.
   \item Les nombres $\dfrac{4}{7}$, $3$, $-4$ et $0,23$ sont-ils des nombres rationnels?
 \end{enumerate}
\end{frame}

\begin{frame}
  \textbf{Proposition. --} L'ensemble des nombres \guess{décimaux} est inclus dans l'ensemble des nombres \guess{rationnels} :\ $\guessmath{\mathbb{D}}\subset\guessmath{\mathbb{Q}}$.\ On a donc :

  \[\guessmath{\mathbb{N}}\subset\guessmathbin{\mathbb{Z}}\subset\guessmath{\mathbb{D}}\subset\guessmath{\mathbb{Q}}\]
\end{frame}

\begin{frame}
  \textbf{Définition. --} À chaque point de la droite graduée ci-dessous, on a associé un nombre unique, qui est appelé\ son \guess{abscisse}.\ Inversement, à chaque nombre correspond un unique point de la droite graduée.\

  \begin{center}
    \newrgbcolor{ududff}{0.30196078431372547 0.30196078431372547 1.}
    \psset{xunit=1.0cm,yunit=1.0cm,algebraic=true,dimen=middle,dotstyle=o,dotsize=5pt 0,linewidth=1.6pt,arrowsize=3pt 2,arrowinset=0.25}
    \begin{pspicture*}(-5.,-1.)(5.,1.)
      \psaxes[labelFontSize=\scriptstyle,xAxis=true,yAxis=false,Dx=10.,Dy=1.,ticksize=-2pt 0]{->}(0,0)(-5.,-1.)(5.,1.)
      \begin{scriptsize}
	\psline[linewidth=1pt,linecolor=blue](-3.33,-0.1)(-3.33,0.1)
	\uput[u](-3.33,0){\ududff{$A$}}
	\uput[d](-3.33,-0.07){\color{blue}$-\dfrac{11}{3}$}
	\psline[linewidth=1pt,linecolor=blue](-2,-0.1)(-2,0.1)
	\uput[u](-2,0){\ududff{$B$}}
	\uput[d](-2,-0.07){\color{blue}$-2$}
	\psline[linewidth=1pt,linecolor=blue](1.73,-0.1)(1.73,0.1)
	\uput[u](1.73,0){\ududff{$C$}}
	\uput[d](1.73,-0.07){\color{blue}$\sqrt{3}$}
	\psline[linewidth=1pt,linecolor=blue](3.14,-0.1)(3.14,0.1)
	\uput[u](3.14,0){\ududff{$D$}}
	\uput[d](3.14,-0.07){\color{blue}$\pi$}
	\psline[linewidth=1pt,linecolor=blue](4.6024,-0.1)(4.6024,0.1)
	\uput[u](4.6024,0){\ududff{$E$}}
	\uput[d](4.6024,-0.07){\color{blue}${4,6024}$}
	\uput[d](0,-0.07){\color{red}$0$}
	\uput[d](1,-0.07){\color{red}$1$}
	\psline[linewidth=1pt,linecolor=red](0,-0.1)(0,0.1)
	\psline[linewidth=1pt,linecolor=red](1,-0.1)(1,0.1)
      \end{scriptsize}
    \end{pspicture*}  
  \end{center}\

  Les nombres \guess{réels} sont les abscisses de tous les points d'une droite graduée.\ L'ensemble des nombres réels est noté $\guessmathbin{\mathbb{R}}$.
\end{frame}

\begin{frame}
  \textbf{Proposition. --} Il existe des nombres réels\ qui ne sont pas rationnels, comme\ $\guessmath{\sqrt{2}}$\ (il faudra savoir le démontrer) ou\ $\guessmathbin{\pi}$.\ Ces nombres sont appelés des nombres \guess{irrationnels}.
\end{frame}

\end{document}

%%% Local Variables:
%%% mode: latex
%%% TeX-master: t
%%% End:
