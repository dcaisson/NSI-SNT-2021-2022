\documentclass[handout]{beamer}

% Lignes réponses
\usepackage{pgffor} % pour la commande \foreach permettant de réaliser une boucle
\newcommand{\pointilles}{{\\\rule{0pt}{1pt}\dotfill\rule{0pt}{1pt}}}
\newcommand{\rep}[1]{\foreach \n in {1,...,#1} {\pointilles}}

% Commandes pour cacher/révéler du texte facilement à l'aide d'un booléen
\usepackage{xstring}
\usepackage{ifthen}

\newboolean{reveal}
\setboolean{reveal}{true}

\newlength{\stextwidth} % une nouvelle longueur

\newcommand\x{6}

\newcommand{\guess}[1]{\ifthenelse{\boolean{reveal}}{{\color{red}#1}}{\settowidth{\stextwidth}{#1}\makebox[\stextwidth]{\dotfill}}}

\newcommand{\guessmath}[1]{\ifthenelse{\boolean{reveal}}{\textcolor{red}{#1}}{\settowidth{\stextwidth}{$#1$}\makebox[1.9\stextwidth]{\dotfill}}}

\newcommand{\guessmathbin}[1]{\ifthenelse{\boolean{reveal}}{\mathbin{\color{red}#1}}{\settowidth{\stextwidth}{$#1$}\makebox[2\stextwidth]{\dotfill}}}

% ========================================================================%

\usetheme{focus}

\usepackage{pgfpages}
\pgfpagesuselayout{4 on 1}[a4paper,landscape]

\usepackage[french]{babel}

\usepackage{xcolor}

\usepackage{pstricks,pst-plot,pst-text,pst-tree,pst-eps,pst-fill,pst-node,pst-math}
\usepackage{pstricks-add,pst-xkey}

\input ../tabvar

\usepackage{multicol}
\usepackage[np]{numprint}

\begin{document}

\title{}

\date{}

\begin{frame}
  \frametitle{Caractériser un ensemble à l'aide d'une valeur absolue}
  \textbf{Définition. --} Soit $x$ un nombre réel. On appelle \guess{valeur absolue} de $x$ le nombre noté $|x|$ défini par :
  \[|x|=\left\{
      \begin{aligned}
	x & \text{ si \guess{$x\geq 0$}}\\
	-x& \text{ si \guess{$x<0$}}
    \end{aligned}\right.\]

    \bigskip

    \textit{Exemples. -- Donner la valeur absolue des nombres $5$, $-2$, $\pi-5$ et $\dfrac{1}{7}-0,1$.}\rep{3}
\end{frame}

\begin{frame}
  \textbf{Proposition. --} On retiendra les propriétés suivantes :
  \begin{itemize}
    \item La valeur absolue d'un nombre est \guess{positive ou nulle}.
    \item Un nombre et son opposé ont \guess{la même valeur absolue}.
  \end{itemize}
\end{frame}

\begin{frame}
  \textbf{Définition. --} On appelle \guess{distance} entre deux réels $a$ et $b$ le nombre \guess{$|b-a|$} (qui est aussi égal à \guess{$|a-b|$}). Sur une droite graduée, si $A$ est le point d'abscisse $a$ et $B$ le point d'abscisse $b$, la distance entre $a$ et $b$ est égale à la distance $AB$.

  \vspace{2cm}

  \textit{Exemples. -- Déterminer la distance entre $3$ et $-1$, puis la distance entre $-15$ et $12$.}
\end{frame}

\begin{frame}
  \textit{Remarque. -- La valeur absolue $|x|$ d'un réel $x$ est donc la distance entre \guess{$x$} et \guess{$0$}.}

  \medskip

  \textit{Exemples. -- Après avoir traduit chacune des égalités et inégalités suivantes à l'aide d'une distance, représenter l'ensemble des réels $x$ tels que :
  \begin{multicols}{2}
    \begin{enumerate}
    \item $|x-4|=2$
    \item $|x-2|=3$
    \item $|x+3|=1$
    \item $|x+1|=2$
    \item $|x-3|\leq 2$
    \item $|x+7|<1$
    \item $|x-5|\geq 3$
    \item $|x+6|>1$
    \end{enumerate}
  \end{multicols}
}
\end{frame}

\begin{frame}
  \textbf{Proposition. --} On remarquera que l'intervalle $[a-r;a+r]$ est l'ensemble des réels $x$ tels que \guess{$|x-a|\leq r$}.

  \vspace{2cm}

  \textit{Exemples. -- Compléter chacune des phrases suivantes :
  \begin{enumerate}
    \item L'intervalle $[2;8]$ est l'ensemble des réels $x$ tels que \rep{2}
    \item L'intervalle $[2,25;6,35]$ est l'ensemble des réels $x$ tels que \rep{2}
    \item Traduire à l'aide d'une valeur absolue la condition $y\in[7,4;7,6]$.\rep{2}
  \end{enumerate}
}
\end{frame}

\begin{frame}
  \textbf{Proposition. --} On remarquera que l'intervalle $]a-r;a+r[$ est l'ensemble des réels $x$ tels que \guess{$|x-a|<r$}.

  \vspace{2cm}

  \textit{Exemples. -- Compléter chacune des phrases suivantes :
  \begin{enumerate}
    \item L'intervalle $]20;30[$ est l'ensemble des réels $x$ tels que \rep{2}
    \item L'intervalle $]2,15;10,55[$ est l'ensemble des réels $x$ tels que \rep{2}
    \item Traduire à l'aide d'une valeur absolue la condition $y\in]18;25[$.\rep{2}
  \end{enumerate}
}
\end{frame}
\end{document}

%%% Local Variables:
%%% mode: latex
%%% TeX-master: t
%%% End:
