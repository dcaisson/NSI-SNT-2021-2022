\documentclass[a4paper,dvipsnames]{article}

\input ../header
\newcommand{\checkedbox}{\makebox[0pt][l]{$\square$}\raisebox{.15ex}{\hspace{0.1em}$\checkmark$}}
\newcommand{\checkbox}{\makebox[0pt][l]{$\square$}\raisebox{.15ex}{\hspace{0.1em}}\hspace{3mm}}

\usepackage{gensymb}

\begin{document}

\title{Activité -- Écarts de température à Lyon}

\date{}
\author{}

\maketitle{}

\thispagestyle{empty}
\pagestyle{empty}

\section{Rappels sur les symboles $\leq$, $\geq$, $<$ et $>$}

Compléter le tableau suivant :

\begin{center}
  \begin{tabular}{@{}cc@{}}
    \toprule
    Symbole & se lit \\
    \midrule
    $\leq$ & est inférieur ou égal à \\
    $\geq$ & \\
    $<$ & \\
    $>$ & \\
    \bottomrule
  \end{tabular}
\end{center}

Parmi les inégalités suivantes, dire lesquelles sont vraies :

\begin{multicols}{4}
  \begin{enumerate}
    \item[\checkbox] $1\leq 2$
    \item[\checkbox] $3\leq 2$
    \item[\checkbox] $2\leq 2$
    \item[\checkbox] $5\geq 4$ 
    \item[\checkbox] $3\geq 4$
    \item[\checkbox] $4\geq 4$
    \item[\checkbox] $9<10$
    \item[\checkbox] $11<10$
    \item[\checkbox] $10<10$
    \item[\checkbox] $21>20$
    \item[\checkbox] $19>20$
    \item[\checkbox] $20>20$
  \end{enumerate}
\end{multicols}

\section{Activité}

Marama apprend que les températures à Lyon en 2018 ont oscillé entre $-10\degree{}$C et $34\degree$C. Il pose à ses amis Mihiau et Paul la question suivante :

\begin{center}
  \og{}Quelle notation pourrait-on utiliser pour représenter rapidement l'ensemble des nombres compris entre $-10$ (inclus) et $34$ (inclus)?\fg{}
\end{center}

Mihiau propose de tracer, comme ci-dessous, la droite numérique et d'hachurer l'ensemble des nombres compris entre $-10$ et $34$.

\begin{center}
  \psset{xunit=1.0cm,yunit=1.0cm,algebraic=true,dimen=middle,dotstyle=o,dotsize=5pt 0,linewidth=1.6pt,arrowsize=3pt 2,arrowinset=0.25}
  \begin{pspicture*}(-2.,-0.6)(4.,0.6)
    \psaxes[labelFontSize=\scriptstyle,xAxis=true,yAxis=false,labels=y,Dx=10,Dy=1.,ticksize=-2pt 0,subticks=2]{->}(0,0)(-2.,-0.6)(4.,0.6)
    \psline[linewidth=1.pt,linecolor=red](-1.,0.)(3.4,0.)
    \psline[linewidth=1.pt,linecolor=red](-0.9,0.2)(-1.,0.2)
    \psline[linewidth=1.pt,linecolor=red](-1.,0.2)(-1.,-0.2)
    \psline[linewidth=1.pt,linecolor=red](-1.,-0.2)(-0.9,-0.2)
    \psline[linewidth=1.pt,linecolor=red](3.3,0.2)(3.4,0.2)
    \psline[linewidth=1.pt,linecolor=red](3.4,0.2)(3.4,-0.2)
    \psline[linewidth=1.pt,linecolor=red](3.4,-0.2)(3.3,-0.2)
    \uput[d](-1,-0.2){\color{red}$-10$}
    \uput[d](0,0){$0$}
    \uput[d](3.4,-0.2){\color{red}$34$}
    \begin{scriptsize}
      \psdots[dotsize=4pt 0,dotstyle=*,linecolor=darkgray](0.,0.)
    \end{scriptsize}
  \end{pspicture*}  
\end{center}

Paul propose alors d'écrire l'ensemble de ces nombres de la manière suivante : \[\color{red}[-10;34].\] Il l'appelle \textit{intervalle fermé}.

\begin{enumerate}
  \item Marama, satisfait de la proposition de son amie, lui demande : \og{}Comment écrirais-tu alors l'ensemble des nombres réels $x$ tels que $0\leq x\leq 8$ ?\fg{} Que va répondre Paul ?
    \rep{2}
  \item Marama demande : \og{}Paul, comment peut-on noter l'ensemble des nombres $x$ tels que $0\leq x<8$ ? Paul répond : $[0;8]$. Mihiau affirme que sa proposition ne convient pas.
    \begin{enumerate}
      \item Pourquoi Mihiau a-t-elle raison ?\rep{2}
      \item Pourquoi ne pourrait-on pas écrire $[0;7,9]$ ?\rep{2}
      \item Mihiau propose comme réponse : $[0;8[$. Elle explique que le \og{}crochet ouvert\fg{} en $8$ signifie qu'on exclut ce nombre. Comment note-t-on l'ensemble des nombres compris entre $0$ (inclus) et $15$ (exclu) ?\rep{2}
      \item Comment note-t-on l'ensemble des nombres compris entre $0$ (exclu) et $15$ (inclu)?\rep{2}
    \end{enumerate}
  \item Marama demande enfin : \og{}Quelle notation peut-on donner à l'ensemble des nombres supérieurs ou égaux à $2$ ?\fg{}. Paul, qui a déjà rencontré ce cas, propose alors $[2;+\infty[$, le symbole $\infty$ représentant \og{}l'infini\fg{}.
    \begin{enumerate}
      \item Comment note-t-on l'ensemble des nombres supérieurs ou égaux à $10$ ?\rep{2}
      \item Comment note-t-on l'ensemble des nombres strictement supérieurs à $10$ ?\rep{2}
      \item Comment note-t-on l'ensemble des nombres inférieurs ou égaux à $3$ ?\rep{2}
      \item Comment note-t-on l'ensemble des nombres strictement inférieurs à $3$ ?\rep{2}
    \end{enumerate}
\end{enumerate}

\end{document}
