\documentclass[handout]{beamer}

% Lignes réponses
\usepackage{pgffor} % pour la commande \foreach permettant de réaliser une boucle
\newcommand{\pointilles}{{\\\rule{0pt}{1pt}\dotfill\rule{0pt}{1pt}}}
\newcommand{\rep}[1]{\foreach \n in {1,...,#1} {\pointilles}}

% Commandes pour cacher/révéler du texte facilement à l'aide d'un booléen
\usepackage{xstring}
\usepackage{ifthen}

\newboolean{reveal}
\setboolean{reveal}{false}

\newlength{\stextwidth} % une nouvelle longueur

\newcommand\x{6}

\newcommand{\guess}[1]{\ifthenelse{\boolean{reveal}}{{\color{red}#1}}{\settowidth{\stextwidth}{#1}\makebox[\stextwidth]{\dotfill}}}

\newcommand{\guessmath}[1]{\ifthenelse{\boolean{reveal}}{\textcolor{red}{#1}}{\settowidth{\stextwidth}{$#1$}\makebox[1.9\stextwidth]{\dotfill}}}

\newcommand{\guessmathbin}[1]{\ifthenelse{\boolean{reveal}}{\mathbin{\color{red}#1}}{\settowidth{\stextwidth}{$#1$}\makebox[2\stextwidth]{\dotfill}}}

% ========================================================================%

\usetheme{focus}

\usepackage{pgfpages}
\pgfpagesuselayout{4 on 1}[a4paper,landscape]

\usepackage[french]{babel}

\usepackage{xcolor}

\usepackage{pstricks,pst-plot,pst-text,pst-tree,pst-eps,pst-fill,pst-node,pst-math}
\usepackage{pstricks-add,pst-xkey}

\input ../tabvar

\usepackage{multicol}
\usepackage[np]{numprint}

\begin{document}

\title{}

\date{}

\begin{frame}
  \frametitle{Intervalles de $\mathbb{R}$}
  \textbf{Définition. --} Soient $a$ et $b$ deux nombres réels tels que $a<b$. L'intervalle $[a;b]$ est l'ensemble des réels tels que $a\leq x\leq b$. On définit de même les intervalles $[a;b[$, $]a;b]$ et $]a;b[$.

  \bigskip

  \begin{center}
    \renewcommand{\arraystretch}{1.2}
    \begin{tabular}{|>{\centering}m{2cm}|>{\centering}m{4cm}|>{\centering}m{4cm}|}
      \hline
      \textbf{Intervalles} & \textbf{Ensemble des réels $x$ tels que \dots} & \textbf{Représentation graphique}\tabularnewline
      \hline
      $[a;b]$ & $a\leq x\leq b$ & \tabularnewline
      \hline
      $[a;b[$ & & \tabularnewline
      \hline
      $]a;b]$ & & \tabularnewline
      \hline
      $]a;b[$ & & \tabularnewline
      \hline
    \end{tabular}
  \end{center}
\end{frame}

\begin{frame}
  \textit{Exemples. --} Représenter (sur quatre graphiques différents) les intervalles $[0;5]$, $]-2;4[$, $[-3;-1[$ et $]-1;8]$.\rep{3}

  \medskip

  Compléter à l'aide des symboles $\in$ et $\notin$ :
  \begin{multicols}{4}
    \scriptsize
    \begin{enumerate}
      \item [] $0\guessmathbin{\in}[0;5]$\columnbreak
      \item [] $5\guessmathbin{\in}[0;5]$\columnbreak
      \item [] $2\guessmathbin{\in}[0;5]$\columnbreak
      \item [] $7\guessmathbin{\in}[0;5]$
    \end{enumerate}
  \end{multicols}

  \begin{multicols}{4}
    \scriptsize
    \begin{enumerate}
      \item [] $-2\guessmathbin{\notin}]-2;4[$ \columnbreak
      \item [] $4\guessmathbin{\notin}]-2;4[$\columnbreak
      \item [] $1\guessmathbin{\in}]-2;4[$\columnbreak
      \item [] $-3\guessmathbin{\notin}]-2;4[$
    \end{enumerate}
  \end{multicols}

  \begin{multicols}{4}
    \scriptsize
    \begin{enumerate}
      \item [] $-3\guessmathbin{\in}[-3;-1[$\columnbreak
      \item [] $-1\guessmathbin{\notin}[-3;-1[$\columnbreak
      \item [] $0\guessmathbin{\notin}[-3;-1[$\columnbreak
      \item [] $-2\guessmathbin{\in}[-3;-1[$
    \end{enumerate}
  \end{multicols}

  \begin{multicols}{4}
    \scriptsize
    \begin{enumerate}
      \item [] $5\guessmathbin{\in}]-1;8]$\columnbreak
      \item [] $8\guessmathbin{\in}]-1;8]$\columnbreak
      \item [] $-1\guessmathbin{\notin}]-1;8]$\columnbreak
      \item [] $-4\guessmathbin{\notin}]-1;8]$
    \end{enumerate} 
  \end{multicols}
\end{frame}

\begin{frame}
  \textbf{Définition. --} Soit $a$ un nombre réel. L'intervalle $[a;+\infty[$ est l'ensemble des réels tels que $x\geq a$. On définit de la même façon les intervalles $]a;+\infty[$, $]-\infty;a]$ et $]-\infty;a[$.

  \begin{center}
    \renewcommand{\arraystretch}{1.2}
    \begin{tabular}{|>{\centering}m{2cm}|>{\centering}m{4cm}|>{\centering}m{4cm}|}
      \hline
      \textbf{Intervalles} & \textbf{Ensemble des réels $x$ tels que \dots} & \textbf{Représentation graphique}\tabularnewline
      \hline
      $[a;+\infty[$ & $x\geq a$ & \tabularnewline
      \hline
      $]a;+\infty[$ & & \tabularnewline
      \hline
      $]-\infty;a]$ & & \tabularnewline
      \hline
      $]-\infty;a[$ & & \tabularnewline
      \hline
    \end{tabular}
  \end{center}
\end{frame}

\begin{frame}
  \textit{Exemples. -- Représenter (sur quatre graphiques différents) les intervalles $[10;+\infty[$, $]-1;+\infty[$, $]-\infty;-5]$ et $]-\infty;2[$.}\rep{3}\\

  \medskip

  \textit{Compléter à l'aide des symboles $\in$ et $\notin$ :}

  \begin{multicols}{3}
    \scriptsize
    \begin{enumerate}
      \item [] $10\guessmathbin{\in}[10;+\infty[$\columnbreak
      \item [] $12\guessmathbin{\in}[10;+\infty[$\columnbreak
      \item [] $2\guessmathbin{\notin}[10;+\infty[$
    \end{enumerate}
  \end{multicols}

  \begin{multicols}{3}
    \scriptsize
    \begin{enumerate}
      \item [] $-1\guessmathbin{\notin}]-1;+\infty[$ \columnbreak
      \item [] $3\guessmathbin{\in}]-1;+\infty[$ \columnbreak
      \item [] $-3\guessmathbin{\notin}]-1;+\infty[$ 
    \end{enumerate}
  \end{multicols}

  \begin{multicols}{3}
    \scriptsize
    \begin{enumerate}
      \item [] $-5\guessmathbin{\in}]-\infty;-5]$\columnbreak
      \item [] $-3\guessmathbin{\notin}]-\infty;-5]$\columnbreak
      \item [] $-10\guessmathbin{\in}]-\infty;-5]$
    \end{enumerate}
  \end{multicols}

  \begin{multicols}{3}
    \scriptsize
    \begin{enumerate}
      \item [] $2\guessmathbin{\notin}]-\infty;2[$\columnbreak
      \item [] $-1\guessmathbin{\in}]-\infty;2[$\columnbreak
      \item [] $3\guessmathbin{\notin}]-\infty;2[$
    \end{enumerate} 
  \end{multicols}
\end{frame}

\end{document}

%%% Local Variables:
%%% mode: latex
%%% TeX-master: t
%%% End:
