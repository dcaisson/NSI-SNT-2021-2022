\documentclass[handout]{beamer}

% Lignes réponses
\usepackage{pgffor} % pour la commande \foreach permettant de réaliser une boucle
\newcommand{\pointilles}{{\\\rule{0pt}{1pt}\dotfill\rule{0pt}{1pt}}}
\newcommand{\rep}[1]{\foreach \n in {1,...,#1} {\pointilles}}

% Commandes pour cacher/révéler du texte facilement à l'aide d'un booléen
\usepackage{xstring}
\usepackage{ifthen}

\newboolean{reveal}
\setboolean{reveal}{false}

\newlength{\stextwidth} % une nouvelle longueur

\newcommand\x{6}

\newcommand{\guess}[1]{\ifthenelse{\boolean{reveal}}{{\color{red}#1}}{\settowidth{\stextwidth}{#1}\makebox[\stextwidth]{\dotfill}}}

\newcommand{\guessmath}[1]{\ifthenelse{\boolean{reveal}}{\textcolor{red}{#1}}{\settowidth{\stextwidth}{$#1$}\makebox[1.3\stextwidth]{\dotfill}}}

\newcommand{\guessmathbin}[1]{\ifthenelse{\boolean{reveal}}{\mathbin{\color{red}#1}}{\settowidth{\stextwidth}{$#1$}\makebox[2\stextwidth]{\dotfill}}}

% ========================================================================%

\usetheme{focus}

\usepackage{pgfpages}
\pgfpagesuselayout{4 on 1}[a4paper,landscape]

\usepackage[french]{babel}

\usepackage{xcolor}

\usepackage{pstricks,pst-plot,pst-text,pst-tree,pst-eps,pst-fill,pst-node,pst-math}
\usepackage{pstricks-add,pst-xkey}

\input ../tabvar

\usepackage{multicol}
\usepackage[np]{numprint}

\usepackage{booktabs}

\newcommand{\vect}[1]{\overrightarrow{#1}}
\newcommand{\Oij}{\left(O;\vect{i},\vect{j}\right)}
\newcommand{\norm}[1]{\left|\left|#1\right|\right|}

\begin{document}

\title{}

\date{}
\begin{frame}
  \frametitle{1. Proportion et pourcentage}
  \textbf{Définition. --} Une population un ensemble d'éléments appelés les \guess{individus}. Une sous-population est \guess{une partie de la population}.

  \bigskip

  \textit{Exemple. -- L'ensemble des élèves de Seconde du lycée Paul Gauguin est une \guess{population}. Chaque élève de Seconde du lycée est un \guess{individu}. L'ensemble des élèves de la Seconde 1 est une \guess{sous-population}.}
\end{frame}

\begin{frame}
  \frametitle{1.2 Proportion d'une sous-population}
  \textit{Exemple. -- Un sondage sur les habitudes alimentaires est effectué auprès de $\np{1200}$ personnes. $174$ personnes se déclarent vegan et $26\%$ des personnes interrogées disent suivre un régime.}

  \begin{enumerate}
    \item Identifier une population et deux sous-populations.\rep{4}
    \item Quelle est la proportion, exprimée en pourcentage, des personnes interrogées qui sont vegan ?\rep{4}
  \end{enumerate}
\end{frame}

\begin{frame}
  \begin{enumerate}
    \setcounter{enumi}{2}
    \item Combien de personnes interrogées suivent un régime ?\rep{8}
  \end{enumerate}
\end{frame}

\begin{frame}
  \frametitle{1.3 Pourcentage de pourcentage}
  \textit{Exemples. --}
  \begin{enumerate}
    \item $60\%$ des élèves d'un lycée sont des filles. Parmi ces filles, $34\%$ portent des lunettes. Déterminer la proportion de filles portant des lunettes dans ce lycée.\rep{8}
  \end{enumerate}
\end{frame}

\begin{frame}
  \begin{enumerate}
    \setcounter{enumi}{1}
  \item Dans une entreprise, $70\%$ des employés partent en vacances en juillet et, parmi eux, $60\%$ partent au bord de la mer. Quelle proportion des employés de cette entreprise partent en vacances au bord de la mer en juillet ?\rep{8}
  \end{enumerate}
\end{frame}

\begin{frame}
  \frametitle{2. Variations d'une quantité}
  % Schéma à faire
  % V_I et V_F
  % Variation absolue : V_F - V_I
  % Variation relative (ou taux d'évolution) : t = (V_F - V_I) / V_I
  % Coefficient multiplicateur : c = 1 + t = V_F / V_I
\end{frame}

\begin{frame}
  \textit{Exemple. -- Le prix d'une paire de chaussures valant initialement $\np{12000}$ XPF a baissé durant les soldes. Cette paire des chaussures est vendue, soldée, à $\np{8400}$ XPF.}
  \begin{enumerate}
    \item Déterminer la variation absolue puis la variation relative.\rep{4}
    \item Compléter la phrase suivante : \og{}Le prix de cette paire de chaussures a $\hdots\hdots\hdots$ de $\hdots\hdots\hdots$\fg{}
    \item Par quel nombre le prix de la paire de chaussures a-t-il été multiplié ? \rep{3}
  \end{enumerate}
\end{frame}

\begin{frame}
  \textit{Exemple. -- En janvier 2018, l'application Tik Tok comptait $40$ millions d'utilisateurs quotidiens. En six mois, le nombre d'utilisateurs quotidiens a été multiplié par $3,75$.}
  \begin{enumerate}
    \item Déterminer le nombre d'utilisateurs quotidiens de Tik Tok en juillet 2018.\rep{4}
    \item Calculer le taux d'évolution du nombre d'utilisateurs quotidiens de Tik Tok entre janvier et juillet 2018.\rep{4}
  \end{enumerate}
\end{frame}

\begin{frame}
  \textit{Remarques. -- Dans le cas d'une augmentation :
    \begin{itemize}
      \item le taux d'évolution est \guess{positif};
      \item le coefficient multiplicateur est \guess{supérieur à $1$}.
    \end{itemize}
    Dans le cas d'une diminution :
    \begin{itemize}
      \item le taux d'évolution est \guess{négatif};
      \item le coefficient multiplicateur est \guess{inférieur à $1$}.
    \end{itemize}
  }
\end{frame}

\begin{frame}
  \frametitle{3. Évolutions successives}
  \textit{Exemple. -- Le prix d'un litre d'essence a augmenté de $15\%$ entre janvier et juillet, puis de $10\%$ entre juillet et décembre.}
  \begin{enumerate}
    \item Déterminer le coefficient multiplicateur correspondant à chacune des deux évolutions, puis en déduire le coefficient multiplicateur global.\rep{8}
  \end{enumerate}
\end{frame}

\begin{frame}
  \begin{enumerate}
    \setcounter{enumi}{1}
  \item En déduire le pourcentage d'augmentation du prix du litre d'essence entre janvier et décembre.\rep{5}
  \item Le prix du litre d'essence a-t-il augmenté de $25\%$ entre janvier et décembre ?\rep{5}
  \end{enumerate}
\end{frame}

\begin{frame}
  \textit{Exemple. -- La première semaine des soldes, un magasin propose $40\%$ de réduction sur tous les vêtements. Lors de la deuxième démarque, le magasin accorde $20\%$ de remise supplémentaire.}
  \begin{enumerate}
    \item Déterminer le taux d'évolution global des prix des vêtements de ce magasin.\rep{4}
    \item Les prix des vêtements ont-il baissé de $60\%$ ?\rep{4}
  \end{enumerate}

  \bigskip

  \textit{Remarque. -- Dans le cas d'évolutions successives, il ne faut en général pas \guess{ajouter les taux d'évolution}.}

\end{frame}

\begin{frame}
  \frametitle{4. Évolution réciproque}

  \textbf{Définition. --} Deux évolutions sont dites réciproques lorsque le coefficient multiplicateur global de ces deux évolutions \guess{est égal à $1$}.

  \bigskip

  \textit{Exemples. -- Déterminer le taux d'évolution réciproque, arrondi à $0,1\%$ près, des évolutions suivantes : 
    \begin{itemize}
      \item une hausse de $20\%$;\rep{3}
      \item une baisse de $8\%$;\rep{3}
    \end{itemize}
  }
\end{frame}

\begin{frame}
  \textit{
    \begin{itemize}
      \item une augmentation de $12,5\%$;\rep{4}
      \item une diminution de $36\%$.\rep{4}
    \end{itemize}
  }
\end{frame}


\begin{frame}
  \textit{Exemple. -- \og{}Journée noire pour Airbus Group, dont l'action a perdu $10,42\%$ en bourse.\fg{} (source : LesEchos.fr)\\
  Quelle évolution permettra à l'action de retrouver sa valeur initiale ?\rep{8}}
\end{frame}

\begin{frame}
  \textit{Exemple. -- Les ventes d'un article ont baissé de $20\%$ de 2017 à 2018. Quelle évolution des ventes de cet article faudrait-il atteindre de 2018 à 2019 pour revenir à la même quantité qu'en 2017 ?\rep{8}}
\end{frame}

\end{document}

%%% Local Variables:
%%% mode: latex
%%% TeX-master: t
%%% End:
