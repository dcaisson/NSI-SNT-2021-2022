\documentclass[handout]{beamer}

% Lignes réponses
\usepackage{pgffor} % pour la commande \foreach permettant de réaliser une boucle
\newcommand{\pointilles}{{\\\rule{0pt}{1pt}\dotfill\rule{0pt}{1pt}}}
\newcommand{\rep}[1]{\foreach \n in {1,...,#1} {\pointilles}}

% Commandes pour cacher/révéler du texte facilement à l'aide d'un booléen
\usepackage{xstring}
\usepackage{ifthen}

\newboolean{reveal}
\setboolean{reveal}{false}

\newlength{\stextwidth} % une nouvelle longueur

\newcommand\x{6}

\newcommand{\guess}[1]{\ifthenelse{\boolean{reveal}}{{\color{red}#1}}{\settowidth{\stextwidth}{#1}\makebox[\stextwidth]{\dotfill}}}

\newcommand{\guessmath}[1]{\ifthenelse{\boolean{reveal}}{\textcolor{red}{#1}}{\settowidth{\stextwidth}{$#1$}\makebox[1.9\stextwidth]{\dotfill}}}

\newcommand{\guessmathbin}[1]{\ifthenelse{\boolean{reveal}}{\mathbin{\color{red}#1}}{\settowidth{\stextwidth}{$#1$}\makebox[2\stextwidth]{\dotfill}}}

% ========================================================================%

\usetheme{focus}

\usepackage{pgfpages}
\pgfpagesuselayout{4 on 1}[a4paper,landscape]

\usepackage[french]{babel}

\usepackage{xcolor}

\usepackage{pstricks,pst-plot,pst-text,pst-tree,pst-eps,pst-fill,pst-node,pst-math}
\usepackage{pstricks-add,pst-xkey}

\input ../tabvar

\usepackage{multicol}
\usepackage[np]{numprint}

\usepackage{booktabs}

\newcommand{\vect}[1]{\overrightarrow{#1}}
\newcommand{\Oij}{\left(O;\vect{i},\vect{j}\right)}
\newcommand{\norm}[1]{\left|\left|#1\right|\right|}

\begin{document}

\title{}

\date{}

\begin{frame}
  \frametitle{Multiples et diviseurs}
  \textbf{Définition. --} Soient $n$ et $m$ deux entiers. S'il existe un {\color{red}\fbox{entier}} $k$ tel que 
  \[n = m\times k,\]
  alors on dit que :
  \begin{itemize}
    \item $n$ est un \guess{multiple} de $m$ ;
    \item $m$ est un \guess{diviseur} de $n$.
  \end{itemize}

  \bigskip

  \textit{Exemples. -- Compléter les phrases suivantes :
    \begin{itemize}
      \item $6$ est un \guess{multiple} de $2$ car \guess{$6 = 2\times 3$}.
      \item $10$ est un \guess{diviseur} de $100$ car \guess{$100 = 10\times 10$}.
      \item $35$ est un multiple de \guess{$7$} car \guess{$35 = 7\times 5$}.
      \item $18$ est un diviseur de \guess{$90$} car \guess{$90=18\times 5$}.
    \end{itemize}
  }
\end{frame}

\begin{frame}
  \textbf{Proposition. --}Soit $a$ un entier. La somme de deux multiples de $a$ est un multiple de $a$.

  \bigskip

  \textit{Exercice. -- Démontrer la proposition précédente.\rep{9}}
\end{frame}

\begin{frame}
  \frametitle{Nombres premiers}
  \textbf{Définition. --} Un entier \textbf{naturel} $n$ est un nombre premier s'il admet exactement deux diviseurs positifs.

  \bigskip

  \textit{Exercice. -- 
    \begin{itemize}
      \item $2$ est-il un nombre premier ?\dotfill
      \item $45$ est-il un nombre premier ?\dotfill
      \item Donner deux autres exemples de nombres premiers.\rep{2}
      \item Donner deux autres exemples de nombres non premiers.\rep{2}
    \end{itemize}
  }
\end{frame}

\begin{frame}
  \frametitle{Nombres pairs, nombres impairs}
  \textbf{Proposition. --}
  \begin{itemize}
    \item Un nombre entier $n$ est pair si, et seulement s'il existe un entier $k$ tel que \dotfill
    \item Un nombre entier $n$ est impair si, et seulement s'il existe un entier $k$ tel que \dotfill
  \end{itemize}

  \bigskip

  \textbf{Proposition. --} 
  \begin{itemize}
    \item Le carré d'un entier pair est pair.
    \item Le carré d'un entier impair est impair.
  \end{itemize}
\end{frame}

\begin{frame}
  \textit{Exercice. -- Démontrer que le carré d'un nombre impair est impair en complétant le texte ci-dessous.\\
    \medskip
    Soit $n$ un entier impair.\\
    Il existe un entier $k$ tel que $n=\guessmath{2k +1}$. On a donc :
    \[
      \begin{aligned}
	n^2 &= \guessmath{(2k+1)^2}\\
	    &= \guessmath{4k^2+4k+1}\\
	    &= 2\times (\guessmath{\underbrace{2k^2+2k}_{\text{entier}}}) + 1
      \end{aligned}
    \]
    Par conséquent, $n^2$ est \guess{impair}. On a donc démontré que\dotfill\rep{1}
  }

\end{frame}

\begin{frame}
  \textit{Exercice. -- Démontrer que le carré d'un entier pair est pair.\rep{10}}
\end{frame}

\begin{frame}
  \frametitle{Critères de divisibilité}
  \textit{Exercice. -- Rappeler les critères de divisibilité par $2$, $3$, $5$ et $10$ et donner des exemples d'utilisation.\rep{10}}
\end{frame}

\end{document}

%%% Local Variables:
%%% mode: latex
%%% TeX-master: t
%%% End:
