\documentclass[a4paper]{article}

\input ../header
\usepackage[np]{numprint}
\usepackage{xcolor}
\usepackage{booktabs}

\setlength{\multicolsep}{2pt}

\begin{document}

\title{Utiliser les multiples et les diviseurs d'un nombre entier -- Quelques exercices}

\pagestyle{empty}

\date{}
\author{}

\maketitle{}

\exo Donner la liste des diviseurs de $36$.

\bigskip

\exo Donner la liste de tous les multiples de $7$ compris entre $50$ et $100$.

\bigskip

\exo Recopier et compléter les phrases suivantes en utilisant les expressions \og{}est un multiple de\fg{}, \og{}est un diviseur de\fg{}, \og{}est divisible par\fg{} ou \og{}divise\fg{}. Préciser toutes les possibilités lorsqu'il y en a plusieurs.

\begin{enumerate}
  \item $34\hdots\hdots\hdots\hdots\hdots\hdots 17$
  \item $18\hdots\hdots\hdots\hdots\hdots\hdots 9$
  \item $12\hdots\hdots\hdots\hdots\hdots\hdots 144$
  \item $5\hdots\hdots\hdots\hdots\hdots\hdots 125$
\end{enumerate}

\bigskip

\exo Compléter le tableau suivant :

\begin{center}
  \begin{tabular}{@{}ccccccc@{}}
    \toprule
    est divisible par & $2$ & $3$ & $5$ & $10$\\
    \midrule
    $238$ &&&&\\
    $\np{1250}$ &&&&\\
    $\np{1350}$ &&&&\\
    $999$ &&&&\\
    \bottomrule
  \end{tabular}
\end{center}

\bigskip

\exo $729$ élèves sont inscrits à la cantine du lycée. Pour simplifier l'aménagement du réfectoire, on souhaite constituer des tables ayant le même nombre d'élèves. Peut-on les mettre par $5$ ? par $9$ ? par $4$ ?

\bigskip

\exo Une crèche dispose de $60$ dalles carrées en mousse. Elle souhaite les placer de manière à former un rectangle.
\begin{enumerate}
  \item Quelles sont les dimensions possibles de ce rectangle ?
  \item Quel est celui qui a le plus grand périmètre ?
\end{enumerate}

\bigskip

\exo Lors d'un tournoi de pétanque, il y a $80$ hommes et $60$ femmes inscrits. L'organisation veut constituer un maximum d'équipes mixtes contenant toutes le même nombre d'hommes et le même nombre de femmes. Combien d'équipes peuvent être constituées ?

\bigskip

\exo\vspace*{-2mm}
\begin{enumerate}
  \item Compléter les phrases suivantes :
    \begin{itemize}
      \item Un nombre $n$ est pair lorsque \dotfill
      \item Un nombre $n$ est impair lorsque \dotfill
    \end{itemize}
  \item Dire si les propositions suivantes sont vraies ou fausses. Justifier.
    \begin{itemize}
      \item \og{}La somme d'un nombre pair et d'un nombre impair est un nombre pair.\fg{}
      \item \og{}La somme de deux nombres impairs est un nombre pair.\fg{}
    \end{itemize}
\end{enumerate}

\bigskip

\exo\vspace*{-2mm}
\begin{enumerate}
  \item Retrouver dans votre cours la liste des nombres premiers inférieurs ou égaux à $100$ et la réécrire.
  \item Qu'est-ce qu'une conjecture ? L'expliquer oralement à vos camarades.
  \item La conjecture de Goldbach affirme que \og{}tout nombre pair supérieur ou égal à $4$ peut s'écrire comme la somme de deux nombres premiers.\fg{}
    \begin{itemize}
      \item Vérifier cette conjecture pour tous les nombres pairs de l'intervalle $[10;20]$.
      \item Expliquer oralement pourquoi dans la question précédente, on n'a pas démontré la conjecture de Goldbach.
    \end{itemize}
\end{enumerate}

\bigskip

\exo Démontrer que le produit de deux nombres impairs est impair.

\bigskip

\end{document}
