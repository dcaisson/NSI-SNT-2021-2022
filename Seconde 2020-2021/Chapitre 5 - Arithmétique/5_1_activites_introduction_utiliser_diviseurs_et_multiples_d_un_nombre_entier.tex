\documentclass[a4paper]{article}

\input ../header
\usepackage[np]{numprint}
\usepackage{xcolor}
\usepackage{booktabs}

\setlength{\multicolsep}{2pt}

\begin{document}

\title{Activités -- Utiliser les multiples et les diviseurs d'un nombre entier}

\pagestyle{empty}

\date{}
\author{}

\maketitle{}

\exo[Le jeu de Juniper Green] Ce jeu se joue à deux, chaque joueur prenant un stylo d'une couleur différente.\\
On dipose de la grille suivante qui contient les entiers de $1$ à $30$.

\begin{center}
  \Large
  \begin{tabular}{@{}|c|c|c|c|c|@{}}
    \toprule
    $1$ & $2$ & $3$ & $4$ & $5$\\
    $6$ & $7$ & $8$ & $9$ & $10$\\
    $11$ & $12$ & $13$ & $14$ & $15$\\
    $16$ & $17$ & $18$ & $19$ & $20$\\
    $21$ & $22$ & $23$ & $24$ & $25$\\
    $26$ & $27$ & $28$ & $29$ & $30$\\
    \bottomrule
  \end{tabular}
\end{center}

\begin{itemize}
  \item Le premier joueur choisit un nombre entre $1$ et $30$ et le barre sur la grille.
  \item À tour de rôle, chaque joueur choisit un nombre parmi les multiples ou les diviseurs du nombre choisi précédemment par son adversaire, et le barre sur la grille.
  \item Le perdant est le joueur qui ne peut plus jouer, c'est-à-dire qui ne trouve plus de multiples ou de diviseurs communs au nombre précédemment choisi.
\end{itemize}

Jouer à ce jeu avec votre voisin.

\bigskip

\exo[Crible d'Ératosthène] On considère la grille suivante :

\begin{center}
\Large
\begin{tabular}{@{}|c|c|c|c|c|c|c|c|c|c|@{}}
  \toprule
   & $2$ & $3$ & $4$ & $5$ & $6$ & $7$ & $8$ & $9$ & $10$\\
  $11$ & $12$ & $13$ & $14$ & $15$ & $16$ & $17$ & $18$ & $19$ & $20$\\
  $21$ & $22$ & $23$ & $24$ & $25$ & $26$ & $27$ & $28$ & $29$ & $30$\\
  $31$ & $32$ & $33$ & $34$ & $35$ & $36$ & $37$ & $38$ & $39$ & $40$\\
  $41$ & $42$ & $43$ & $44$ & $45$ & $46$ & $47$ & $48$ & $49$ & $50$\\
  $51$ & $52$ & $53$ & $54$ & $55$ & $56$ & $57$ & $58$ & $59$ & $60$\\
  $61$ & $62$ & $63$ & $64$ & $65$ & $66$ & $67$ & $68$ & $69$ & $70$\\
  $71$ & $72$ & $73$ & $74$ & $75$ & $76$ & $77$ & $78$ & $79$ & $80$\\
  $81$ & $82$ & $83$ & $84$ & $85$ & $86$ & $87$ & $88$ & $89$ & $90$\\
  $91$ & $92$ & $93$ & $94$ & $95$ & $96$ & $97$ & $98$ & $99$ & $100$\\
  \bottomrule
\end{tabular}
\end{center}

\begin{enumerate}
  \item Le plus petit nombre non barré est $2$ : l'entourer, puis barrer tous ses multiples.
  \item Quel est maintenant le plus petit nombre non barré ? Entourer ce nombre, puis barrer tous ses multiples.
  \item Recommencer jusqu'à ce que tous les nombres de la grille soient barrés ou entourés.
  \item Que peut-on dire des nombres entourés ?
\end{enumerate}

\end{document}
