\documentclass[a4paper,dvipsnames]{article}

\input ../header
\newcommand{\checkedbox}{\makebox[0pt][l]{$\square$}\raisebox{.15ex}{\hspace{0.1em}$\checkmark$}}
\newcommand{\checkbox}{\makebox[0pt][l]{$\square$}\raisebox{.15ex}{\hspace{0.1em}}\hspace{3mm}}

% Intervalle
\def\intervalleFF(#1,#2){\psline[linecolor=red]{[-]}(#1,0)(#2,0)}
\def\intervalleOO(#1,#2){\psline[linecolor=red]{]-[}(#1,0)(#2,0)}
\def\intervalleFO(#1,#2){\psline[linecolor=red]{[-[}(#1,0)(#2,0)}
\def\intervalleOF(#1,#2){\psline[linecolor=red]{]-]}(#1,0)(#2,0)}
\def\intervalleFpI(#1,#2){\psline[linecolor=red]{[-}(#1,0)(#2,0)}
\def\intervalleOpI(#1,#2){\psline[linecolor=red]{]-}(#1,0)(#2,0)}
\def\intervalleFmI(#1,#2){\psline[linecolor=red]{-]}(#1,0)(#2,0)}
\def\intervalleOmI(#1,#2){\psline[linecolor=red]{-[}(#1,0)(#2,0)}

\begin{document}

\title{Évaluation 12 -- Sujet B}
\author{}
\date{}

\maketitle{}

\pagestyle{empty}
\thispagestyle{empty}


% Résolution graphique d'équations
\exo[3 points] Deux fonctions $f$ et $g$ définies sur $[-5;7]$ sont représentées ci-dessous :

\begin{center}
  \psset{xunit=0.8cm,yunit=0.8cm,algebraic=true,dimen=middle,dotstyle=o,dotsize=5pt 0,linewidth=1.pt,arrowsize=3pt 2,arrowinset=0.25}
  \begin{pspicture*}(-5.,-3.)(7.,4.)
    \multips(0,-3)(0,1.0){8}{\psline[linestyle=dashed,linecap=1,dash=1.5pt 1.5pt,linewidth=0.4pt,linecolor=lightgray]{c-c}(-5.,0)(7.,0)}
    \multips(-5,0)(1.0,0){13}{\psline[linestyle=dashed,linecap=1,dash=1.5pt 1.5pt,linewidth=0.4pt,linecolor=lightgray]{c-c}(0,-3.)(0,4.)}
    \psaxes[labelFontSize=\scriptstyle,xAxis=true,yAxis=true,Dx=1.,Dy=1.,ticksize=-2pt 0]{->}(0,0)(-5.,-3.)(7.,4.)
    \psset{linecolor=red}
    \HermiteDDL(-5,2,1.2,0)(-3.5,3.2,0,0)(-2,2,-2,0)(-1,0,-1.5,0)(1,-2,0,0)(3,0,1,0)(5,2,2,0)(6,4,0,0)(7,3,-1,0)
    \uput[dr](-5,2){\color{red}$\mathcal{C}_f$}
    \psset{linecolor=blue}
    \HermiteDDL(-5,-1,1,0)(-4,0,1,0)(-2,2,0.5,0)(0,3,0,0)(3,0,-0.5,0)(5,-1,-0.5,0)(7,-2,0,0)
    \uput[dr](-5,-1){\color{blue}$\mathcal{C}_g$}
  \end{pspicture*}
\end{center}

\begin{enumerate}
  \item Résoudre graphiquement les équations :
    \begin{multicols}{3}
      \begin{enumerate}
	\item $f(x)=2$
	\item $g(x)=0$
	\item $f(x)=g(x)$
      \end{enumerate} 
    \end{multicols}
    \dotfill\rep{5}
  \item Donner une équation du type $g(x)=k$ n'ayant aucune solution.\rep{3}
\end{enumerate}
\bigskip

% Tracer la courbe d'une fonction vérifiant des conditions
\exo[3 points] 

\begin{multicols}{2}
  Dans le repère ci-dessous, tracer la courbe d'une fonction $h$ définie sur $[-5;4]$ telle que :
  \begin{itemize}
    \item l'équation $h(x)=2$ a quatre solutions ;
    \item les solutions de l'équation $h(x)=-3$ sont $-4$ et $1$ ;
    \item $h(0)=1,5$.
  \end{itemize}
  \vspace*{4.5cm}
  \begin{center}
    \psset{xunit=0.8cm,yunit=0.8cm,algebraic=true,dimen=middle,dotstyle=o,dotsize=5pt 0,linewidth=1.pt,arrowsize=3pt 2,arrowinset=0.25}
    \begin{pspicture*}(-5.,-4.)(5.,4.)
      \multips(0,-4)(0,1.0){9}{\psline[linestyle=dashed,linecap=1,dash=1.5pt 1.5pt,linewidth=0.4pt,linecolor=lightgray]{c-c}(-5.,0)(5.,0)}
      \multips(-5,0)(1.0,0){11}{\psline[linestyle=dashed,linecap=1,dash=1.5pt 1.5pt,linewidth=0.4pt,linecolor=lightgray]{c-c}(0,-4.)(0,4.)}
      \psaxes[labelFontSize=\scriptstyle,xAxis=true,yAxis=true,Dx=1.,Dy=1.,ticksize=-2pt 0]{->}(0,0)(-5.,-4.)(5.,4.)
    \end{pspicture*}
  \end{center}
\end{multicols}

\pagebreak

% Pourcentage (déterminer effectif, proportion)
\exo[2 points] Un smartphone possède une capacité de stockage de $64$ Go.
\begin{enumerate}
  \item Le système d'exploitation occupe $5$ Go. Quel pourcentage de la capacité totale du smartphone est occupée par ce système ?\rep{6}
  \item Les photos, musiques et vidéos sauvegardées occupent $60\%$ de la mémoire restante. Quelle place (en Go) occupent ces fichiers personnels dans le smartphone ?\rep{6}
\end{enumerate}

\bigskip

% Pourcentage de pourcentage
\exo[1 point] Dans une grande surface de Tahiti, $20\%$ des smartphones vendus sont des smartphones de la marque LG. Parmi ceux-ci, $50\%$ possèdent plus de $64$ Go de stockage. Parmi les smartphones vendus, quelle est la proportion de smartphones de la marque LG avec plus de $64$ Go de stockage ?\rep{6}

\bigskip

% Taux d'évolution : déterminer la valeur finale, la valeur initiale
\exo [3 points] Un hôtel proposait $50$ chambres en 2018 et $60$ chambres en 2019. 
\begin{enumerate}
  \item Calculer le pourcentage d'augmentation du nombre de chambres entre 2018 et 2019.\rep{6}
  \item Des travaux ont été réalisés pour augmenter la taille des chambres. À la fin des travaux, en 2020, l'hôtel proposait $10\%$ de chambres en moins. Combien de chambres étaient proposées en 2020, après les travaux ?\rep{6}
\end{enumerate}

\bigskip

% Exercice de réflexion : deux évolutions successives
\exo[2 points] Le prix d'une console de jeux subit une augmentation de $30\%$ puis une réduction de $30\%$. Que peut-on dire du prix de cette console après ces deux évolutions ?\rep{6}

\end{document}
