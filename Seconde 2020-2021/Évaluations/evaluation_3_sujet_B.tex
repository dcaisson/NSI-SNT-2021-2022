\documentclass[a4paper,dvipsnames]{article}

\input ../header
\newcommand{\checkedbox}{\makebox[0pt][l]{$\square$}\raisebox{.15ex}{\hspace{0.1em}$\checkmark$}}
\newcommand{\checkbox}{\makebox[0pt][l]{$\square$}\raisebox{.15ex}{\hspace{0.1em}}\hspace{3mm}}

\begin{document}

\title{Évaluation 3 -- Sujet B}
\author{}
\date{}

\maketitle{}

\pagestyle{empty}

\exo Traduire les égalités et inégalités suivantes à l'aide d'une distance, puis représenter l'ensemble des réels $x$ tels que :

\begin{multicols}{2}
  \begin{enumerate}
    \item $|x-4|=1$\rep{8}
    \item $|x-2|<3$\rep{8}
    \item $|x-9|\leq1$\rep{8}
    \item $|x+2|>8$\rep{8}
  \end{enumerate}
\end{multicols}

\bigskip

\exo 
\begin{enumerate}
\item Compléter chacune des phrases suivantes :
  \begin{enumerate}
    \item L'intervalle $[2;10]$ est l'ensemble des réels $x$ tels que \rep{3}
    \item L'intervalle $]-65;-45[$ est l'ensemble des réels $y$ tels que \rep{3}
  \end{enumerate}
\item Traduire la condition $z\in]38;46[$ à l'aide d'une valeur absolue.\rep{6}
\end{enumerate}

\bigskip

\exo On note $x$ la taille, en m, d'un élève de Seconde 1. On sait que :
\[|x-1,71| = |x-1,5|.\]
Combien mesure cet élève ?

\end{document}
