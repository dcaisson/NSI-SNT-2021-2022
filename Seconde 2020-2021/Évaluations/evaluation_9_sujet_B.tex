\documentclass[a4paper,dvipsnames]{article}

\input ../header
\newcommand{\checkedbox}{\makebox[0pt][l]{$\square$}\raisebox{.15ex}{\hspace{0.1em}$\checkmark$}}
\newcommand{\checkbox}{\makebox[0pt][l]{$\square$}\raisebox{.15ex}{\hspace{0.1em}}\hspace{3mm}}

% Intervalle
\def\intervalleFF(#1,#2){\psline[linecolor=red]{[-]}(#1,0)(#2,0)}
\def\intervalleOO(#1,#2){\psline[linecolor=red]{]-[}(#1,0)(#2,0)}
\def\intervalleFO(#1,#2){\psline[linecolor=red]{[-[}(#1,0)(#2,0)}
\def\intervalleOF(#1,#2){\psline[linecolor=red]{]-]}(#1,0)(#2,0)}
\def\intervalleFpI(#1,#2){\psline[linecolor=red]{[-}(#1,0)(#2,0)}
\def\intervalleOpI(#1,#2){\psline[linecolor=red]{]-}(#1,0)(#2,0)}
\def\intervalleFmI(#1,#2){\psline[linecolor=red]{-]}(#1,0)(#2,0)}
\def\intervalleOmI(#1,#2){\psline[linecolor=red]{-[}(#1,0)(#2,0)}

\begin{document}

\title{Évaluation 9 -- Sujet B}
\author{}
\date{}

\maketitle{}

\pagestyle{empty}
\thispagestyle{empty}

% QCM sur les multiples/diviseurs et les puissances
\exo[3 points] Cet exercice est un QCM (questionnaire à choix multiples). Pour chacune des questions posées, une seule réponse est exacte. Entourer, sur l'énoncé, la lettre correspondant à la réponse exacte. Aucune justification n'est demandée. Une réponse exacte rapporte 1 point ; une réponse fausse, une réponse multiple ou l'absence de réponse ne rapporte ni n'enlève aucun point.

\begin{enumerate}
  \item Le nombre $25$ :
    \vspace{-3mm}
    \begin{multicols}{2}
      \begin{enumerate}
	\item est un diviseur de $75$ ;
	\item est un multiple de $75$ ;
	\item n'est ni un diviseur, ni un multiple de $75$ ;
	\item est l'unique diviseur de $75$.
      \end{enumerate} 
    \end{multicols}
  \item Le nombre $40$ :
    \vspace{-3mm}
    \begin{multicols}{2}
      \begin{enumerate}
	\item est un diviseur de $20$ ;
	\item est un multiple de $20$ ;
	\item n'est ni un diviseur, ni un multiple de $20$ ;
	\item est l'unique multiple de $20$.
      \end{enumerate} 
    \end{multicols}
  \item Parmi les nombres suivants, lequel est un nombre premier ?
    \vspace{-3mm}
    \begin{multicols}{4}
      \begin{enumerate}
	\item $\np{1450}$
	\item $\np{33642}$
	\item $\np{54325}$
	\item $\np{5189}$
      \end{enumerate} 
    \end{multicols}
\end{enumerate}

\bigskip

% Question préliminaire
\exo[2 points] Donner, sans justifier, la liste de tous les diviseurs de $45$.\rep{3}

\bigskip

% Exercice sur les multiples/diviseurs : rechercher le PGCD
\exo[3 points] Pour le $1$\ier{} mai, Marama dispose de $45$ brins de muguet et $75$ roses. Il souhaite utiliser toutes ses fleurs pour constituer le plus grand nombre de bouquets \textbf{identiques}.
\begin{enumerate}
  \item Déterminer le nombre maximal de bouquets que Marama pourra constituer.\rep{8}
  \item Quelle sera la composition de chaque bouquet ?\rep{4}
\end{enumerate}

\bigskip

% Exercice théorique sur les multiples diviseurs
\exo[3 points] Démontrer que la somme de deux nombres impairs est paire.\rep{12}

\bigskip

% Vrai/Faux sur les multiples et diviseurs : contre-exemple
\exo[2 points] Dire si l'affirmation suivante est vraie ou fausse, en justifiant votre réponse :

\begin{center}
  \og{}Le produit d'un nombre pair par un nombre impair est un nombre impair.\fg{}
\end{center}

\dotfill\rep{7}

\bigskip

% Exercice sur les puissances (tableau multiplicativement magique)
\exo[3 points] On considère le tableau suivant :

\begin{center}
  \renewcommand{\arraystretch}{1.2}
  \begin{tabular}{|*{3}{>{\centering}m{1.5cm}|}}
    \hline
    $3\times 7$ & $3^4\times7^2$ & $3$\tabularnewline
    \hline
    $3^2$ & $3^2\times 7$ & $(3\times 7)^2$\tabularnewline
    \hline
    $7^2\times3^3$ & $1$ & $3^3\times 7$\tabularnewline
    \hline
  \end{tabular}
\end{center}

Vérifier que le produit des nombres de la deuxième ligne est égal au produit des nombres de la troisième ligne.\rep{5}

\bigskip

% Exercice de statistiques : comparaison de deux séries
\exo[4 points] Un entraîneur de football a relevé les buts marqués par Tamahana et Kamalani lors des $20$ derniers matchs :

\begin{center}
  \begin{tabular}{@{}cccccc@{}}
    \toprule
    Nombre de buts & $0$ & $1$ & $2$ & $3$ & $4$ \\
    \midrule
    Nombre de matchs (Kamalani) & $3$ & $5$ & $6$ & $3$ & $3$\\
    Nombre de matchs (Tamahana) & $5$ & $3$ & $5$ & $3$ & $4$\\
    \bottomrule
  \end{tabular}
\end{center}

\begin{enumerate}
  \item Déterminer le nombre moyen de buts marqués par match pour chaque joueur.\rep{8}
  \item Que constate-t-on ?\rep{4}
  \item Quel est le joueur le plus régulier ?\rep{8}
\end{enumerate}
\end{document}
