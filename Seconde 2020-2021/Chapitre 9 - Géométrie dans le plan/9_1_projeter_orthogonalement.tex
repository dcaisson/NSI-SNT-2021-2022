\documentclass[handout]{beamer}

% Lignes réponses
\usepackage{pgffor} % pour la commande \foreach permettant de réaliser une boucle
\newcommand{\pointilles}{{\\\rule{0pt}{1pt}\dotfill\rule{0pt}{1pt}}}
\newcommand{\rep}[1]{\foreach \n in {1,...,#1} {\pointilles}}

% Commandes pour cacher/révéler du texte facilement à l'aide d'un booléen
\usepackage{xstring}
\usepackage{ifthen}

\newboolean{reveal}
\setboolean{reveal}{false}

\newlength{\stextwidth} % une nouvelle longueur

\newcommand\x{6}

\newcommand{\guess}[1]{\ifthenelse{\boolean{reveal}}{{\color{red}#1}}{\settowidth{\stextwidth}{#1}\makebox[\stextwidth]{\dotfill}}}

\newcommand{\guessmath}[1]{\ifthenelse{\boolean{reveal}}{\textcolor{red}{#1}}{\settowidth{\stextwidth}{$#1$}\makebox[1.3\stextwidth]{\dotfill}}}

\newcommand{\guessmathbin}[1]{\ifthenelse{\boolean{reveal}}{\mathbin{\color{red}#1}}{\settowidth{\stextwidth}{$#1$}\makebox[2\stextwidth]{\dotfill}}}

% ========================================================================%

\usetheme{focus}

\usepackage{pgfpages}
\pgfpagesuselayout{4 on 1}[a4paper,landscape]

\usepackage[french]{babel}

\usepackage{xcolor}

\usepackage{pstricks,pst-plot,pst-text,pst-tree,pst-eps,pst-fill,pst-node,pst-math}
\usepackage{pstricks-add,pst-xkey}

\input ../tabvar

\usepackage{multicol}
\usepackage[np]{numprint}

\usepackage{booktabs}

\newcommand{\vect}[1]{\overrightarrow{#1}}
\newcommand{\Oij}{\left(O;\vect{i},\vect{j}\right)}
\newcommand{\norm}[1]{\left|\left|#1\right|\right|}

\begin{document}

\title{}

\date{}

\begin{frame}
  \frametitle{1. Définition}
  \textbf{Définition. --} Soient $M$ un point et $\mathcal{D}$ une droite.
    \begin{itemize}
      \item Si le point $M$ n'appartient pas à $\mathcal{D}$, alors le projeté orthogonal du point $M$ sur la droite $\mathcal{D}$ est le point $P$ tel que\\
	\dotfill
      \item Si le point $M$ appartient à $\mathcal{D}$, alors le projeté orthogonal du point $M$ sur la droite $\mathcal{D}$ est le point $M$.
    \end{itemize}

    \medskip

    \textit{Remarque. -- Pour construire le projeté orthogonal d'un point sur une droite, j'ai souvent besoin d'utiliser \dotfill}
\end{frame}

\begin{frame}
  \textit{Exemple. -- Tracer une droite $\mathcal{D}$, placer un point $M$ n'appartenant pas à la droite $\mathcal{D}$ puis construire le projeté orthogonal $M'$ de $M$ sur $\mathcal{D}$.}
  \vspace{7cm}
\end{frame}

\begin{frame}
  \textit{Exemple. -- Soit $ABC$ un triangle rectangle en $B$. Que peut-on dire :
    \begin{itemize}
      \item du projeté orthogonal de $A$ sur la droite $(BC)$ ?
      \item du projeté orthogonal de $C$ sur la droite $(AB)$ ?
    \end{itemize}
    \dotfill\rep{4}
  }
\end{frame}

\begin{frame}
  \frametitle{2. Propriété du projeté orthogonal}
  \textbf{Proposition. --} Le projeté orthogonal d'un point $M$ sur une droite $\mathcal{D}$ est le point de la droite $\mathcal{D}$ \dotfill

  \vspace{6cm}
\end{frame}

\end{document}

%%% Local Variables:
%%% mode: latex
%%% TeX-master: t
%%% End:
