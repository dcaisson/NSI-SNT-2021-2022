\documentclass[a4paper]{article}

\input ../header
\usepackage[np]{numprint}
\usepackage{xcolor}
\usepackage{booktabs}
\usepackage[version=4]{mhchem}

\setlength{\multicolsep}{2pt}

\begin{document}

\title{Quelques exercices sur les puissances}

\pagestyle{empty}

\thispagestyle{empty}
\date{}
\maketitle{}

\exo\vspace{-2mm}
\begin{enumerate}
  \item Vérifier que le tableau ci-dessous est multiplicativement magique, c'est-à-dire que le produit des nombres est le même sur chaque ligne, chaque colonne et chaque diagonale.

    \begin{center}
      \renewcommand{\arraystretch}{1.2}
      \begin{tabular}{|*{3}{>{\centering}m{1.5cm}|}}
	\hline
	$3\times 7$ & $3^4\times7^2$ & $3$\tabularnewline
	\hline
	$3^2$ & $3^2\times 7$ & $(3\times 7)^2$\tabularnewline
	\hline
	$7^2\times3^3$ & $1$ & $3^3\times 7$\tabularnewline
	\hline
      \end{tabular}
    \end{center}

  \item Compléter le tableau ci-dessous pour qu'il soit multiplicativement magique :

    \begin{center}
      \renewcommand{\arraystretch}{1.2}
      \begin{tabular}{|*{3}{>{\centering}m{2cm}|}}
	\hline
	$3^4\times2^2\times 5$ & & \tabularnewline
	\hline
	$3^4\times 5^2 $ & $3^3\times 2^2\times 5^2$ & $(3\times 5)^2\times 2^4$\tabularnewline
	\hline
			 & & \tabularnewline
	\hline
      \end{tabular}
    \end{center}
\end{enumerate}

\bigskip

\exo À la fin d'un calcul, une calculatrice affiche les résultats ci-dessous. Préciser la signification de chaque affichage.

\begin{multicols}{4}
  \begin{enumerate}
    \item \verb|1,020304E-04|
    \item \verb|4,546E-03|
    \item \verb|8,6071E-04|
    \item \verb|7,0045E-02|
  \end{enumerate}
\end{multicols}

\bigskip

\exo\vspace{-2mm}
\begin{enumerate}
  \item Écrire $D=22^6\times\dfrac{33^3}{8\times 6^3}$ sous la forme $11^n$, où $n$ est un entier relatif.
  \item Écrire le nombre $E=15^3\times\dfrac{3^{-2}}{5^2}\times45^{-2}$ sous la forme $3^n\times 5^p$, avec $n$ et $p$ entiers relatifs.
  \item 
    \begin{enumerate}
      \item Écrire $G=4^3\times 9^{-2}$ et $H=6^3\times 18^{-2}$ sous la forme $2^n\times 3^p$, où $n$ et $p$ sont des entiers relatifs.
      \item Écrire $\dfrac{G}{H}$ de la même façon.
    \end{enumerate}
  \item Écrire $I=\dfrac{8\times 10^{15}\times 15\times 10^{-6}}{20\times \left(10^2\right)^5}$ sous forme irréductible.
\end{enumerate}

\bigskip

\exo Soient $a$ un réel non nul et $n$ et $m$ des entiers relatifs. Écrire les expressions suivantes sous la forme d'une seule puissance de $a$:

\begin{multicols}{4}
  \begin{enumerate}
    \item $a^n\times a^{2n}\vphantom{\dfrac{1}{2^2}}$
    \item $a^n\times a^{-2n}\vphantom{\dfrac{1}{2^2}}$
    \item $a^n\times a^{3m}\vphantom{\dfrac{1}{2^2}}$
    \item $\dfrac{a}{a^{-n}}$
  \end{enumerate}
\end{multicols}

\bigskip

\exo Soient $a$ et $b$ deux réels non nuls et $p$ et $q$ des entiers relatifs. Écrire les expressions suivantes sous la forme $a^n\times b^m$ où $n$ et $m$ sont des entiers relatifs :

\begin{multicols}{2}
  \begin{enumerate}
    \item $\left(a^{-5}\times b^{-2}\right)^{10}\vphantom{\left(\dfrac{1}{2^2}\right)^{-2}}$
    \item $\left(\dfrac{a}{b^2}\right)^{-2}\times\left(\dfrac{1}{a^3}\right)^5$
  \end{enumerate}
\end{multicols}

\bigskip

\exo Le noyau d'un atome est constitué de neutron(s) et de proton(s) appelés les nucléons. Autour du noyau gravite(nt) un (ou plusieurs) électron(s). La masse d'un nucléon est $\np{1,672}\times10^{-27}$ kg et celle d'un électron est $\np{9,109}\times10^{-31}$ kg.

\begin{enumerate}
  \item Justifier qu'un électron est environ $\np{1830}$ fois plus léger qu'un nucléon.
  \item Un atome d'aluminium est décrit par l'écriture $\ce{^{27}_{13}Al}$. Combien a-t-il de nucléons ? d'électrons ?
  \item En déduire la masse d'un atome d'aluminium.
\end{enumerate}

\end{document}
