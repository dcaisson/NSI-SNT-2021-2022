\documentclass[handout]{beamer}

% Lignes réponses
\usepackage{pgffor} % pour la commande \foreach permettant de réaliser une boucle
\newcommand{\pointilles}{{\\\rule{0pt}{1pt}\dotfill\rule{0pt}{1pt}}}
\newcommand{\rep}[1]{\foreach \n in {1,...,#1} {\pointilles}}

% Commandes pour cacher/révéler du texte facilement à l'aide d'un booléen
\usepackage{xstring}
\usepackage{ifthen}

\newboolean{reveal}
\setboolean{reveal}{false}

\newlength{\stextwidth} % une nouvelle longueur

\newcommand\x{6}

\newcommand{\guess}[1]{\ifthenelse{\boolean{reveal}}{{\color{red}#1}}{\settowidth{\stextwidth}{#1}\makebox[\stextwidth]{\dotfill}}}

\newcommand{\guessmath}[1]{\ifthenelse{\boolean{reveal}}{\textcolor{red}{#1}}{\settowidth{\stextwidth}{$#1$}\makebox[1.9\stextwidth]{\dotfill}}}

\newcommand{\guessmathbin}[1]{\ifthenelse{\boolean{reveal}}{\mathbin{\color{red}#1}}{\settowidth{\stextwidth}{$#1$}\makebox[2\stextwidth]{\dotfill}}}

% ========================================================================%

\usetheme{focus}

\usepackage{pgfpages}
\pgfpagesuselayout{4 on 1}[a4paper,landscape]

\usepackage[french]{babel}

\usepackage{xcolor}

\usepackage{pstricks,pst-plot,pst-text,pst-tree,pst-eps,pst-fill,pst-node,pst-math}
\usepackage{pstricks-add,pst-xkey}

\input ../tabvar

\usepackage{multicol}
\usepackage[np]{numprint}

\usepackage{minted}

\begin{document}

\title{}

\date{}

\begin{frame}
  \frametitle{1. Vocabulaire}
  \begin{itemize}
    \item Une équation d'inconnue $x$ est une égalité dans laquelle intervient un nombre inconnu $x$.

      \medskip

      \textit{Exemple. -- $2x+3 = 5$ est une équation d'inconnue $x$.}

      \medskip

    \item Une solution d'une équation est une valeur de l'inconnue qui rend l'égalité vraie.

      \medskip

      \textit{Exemples. -- Vérifier que $0$ n'est pas une solution de l'équation précédente, puis que $1$ est une solution de cette équation.\rep{2}}

      \medskip

    \item Résoudre une équation dans un ensemble $\mathcal{E}$ c'est trouver tous les nombres de $\mathcal{E}$ qui sont des solutions de l'équation.
  \end{itemize}
\end{frame}

\begin{frame}
  \frametitle{2. Équivalence de deux équations}
  \begin{itemize}
    \item On dit que deux équations sont équivalentes lorsqu'elles ont les mêmes solutions.

      \medskip

      \textit{Exemple. -- Les équations $2x-1=-4x+12$ et $\guessmath{6}x=\guessmath{13}$ sont équivalentes.}
  \end{itemize}
\end{frame}

\begin{frame}
  \textbf{Proposition. --} Les manipulations suivantes transforment une équation en une équation équivalente :
  \begin{itemize}
    \item Ajouter (ou soustraire) un même nombre aux deux membres d'une équation.
    \item Multiplier (ou diviser) les deux membres d'une équation par un même nombre non nul.
    \item Développer, factoriser, réduire l'un des deux membres de l'équation.
  \end{itemize}
\end{frame}

\begin{frame}
  \frametitle{3. Équation-produit}
  \textbf{Proposition. --} Un produit de deux nombres réels est nul si, et seulement si, \guess{l'un des deux nombres est nul}.

  \medskip

  \textit{Exemple. -- Résoudre l'équation $(x-2)(3x+6)=0$.\rep{7}}
\end{frame}

\begin{frame}
  \frametitle{4. Équation-quotient}
  \textbf{Proposition. --} Un quotient d'un nombre réel $A$ par un nombre réel non nul $B$ est nul si, et seulement si, \guess{$A$ est nul}.

  \medskip

  \textit{Exemple. -- Résoudre l'équation $\dfrac{x-3}{x+2}=0$.\\
  \dotfill\rep{8}}
\end{frame}

\end{document}

%%% Local Variables:
%%% mode: latex
%%% TeX-master: t
%%% End:
