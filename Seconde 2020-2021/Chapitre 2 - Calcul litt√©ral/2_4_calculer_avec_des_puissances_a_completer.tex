\documentclass[handout]{beamer}

% Lignes réponses
\usepackage{pgffor} % pour la commande \foreach permettant de réaliser une boucle
\newcommand{\pointilles}{{\\\rule{0pt}{1pt}\dotfill\rule{0pt}{1pt}}}
\newcommand{\rep}[1]{\foreach \n in {1,...,#1} {\pointilles}}

% Commandes pour cacher/révéler du texte facilement à l'aide d'un booléen
\usepackage{xstring}
\usepackage{ifthen}

\newboolean{reveal}
\setboolean{reveal}{false}

\newlength{\stextwidth} % une nouvelle longueur

\newcommand\x{6}

\newcommand{\guess}[1]{\ifthenelse{\boolean{reveal}}{{\color{red}#1}}{\settowidth{\stextwidth}{#1}\makebox[\stextwidth]{\dotfill}}}

\newcommand{\guessmath}[1]{\ifthenelse{\boolean{reveal}}{\textcolor{red}{#1}}{\settowidth{\stextwidth}{$#1$}\makebox[1.9\stextwidth]{\dotfill}}}

\newcommand{\guessmathbin}[1]{\ifthenelse{\boolean{reveal}}{\mathbin{\color{red}#1}}{\settowidth{\stextwidth}{$#1$}\makebox[2\stextwidth]{\dotfill}}}

% ========================================================================%

\usetheme{focus}

\usepackage{pgfpages}
\pgfpagesuselayout{4 on 1}[a4paper,landscape]

\usepackage[french]{babel}

\usepackage{xcolor}

\usepackage{pstricks,pst-plot,pst-text,pst-tree,pst-eps,pst-fill,pst-node,pst-math}
\usepackage{pstricks-add,pst-xkey}

\input ../tabvar

\usepackage{multicol}
\usepackage[np]{numprint}

\usepackage{minted}

\begin{document}

\title{}

\date{}

\begin{frame}
  \frametitle{Pourquoi utilise-t-on des puissances ?}
  Les dimensions des objets de l'Univers qui nous entourent vont de l'échelle microscopique à l'échelle macroscopique. En physique et en biologie, on utilise surtout les \textbf{puissances de $10$} pour exprimer ces dimensions.

  \begin{itemize}
    \item Le diamètre de la Voie lactée est d'environ $10^{21}$ m.
      \[10^{21}=\underbrace{10\times10\times10\times\hdots\times10}_{\guessmath{21} \text{ facteurs}}\]
    \item Le diamètre d'un atome est d'environ $10^{-10}$ m.
      \[10^{-10}=0,000\,000\,000\,{\color{red}1}.\]
  \end{itemize}
\end{frame}

\begin{frame}
  \frametitle{Calculer avec des puissances}
  Dans toute la suite $a$ désigne un nombre réel, $n$ un entier naturel non nul, $p$ et $q$ deux entiers relatifs.

  \[a^n = \underbrace{a\times a\times\hdots a}_{n \text{ facteurs}}\]

  \[a^1 = \guessmath{a}\]

  \[a^0 = \guessmath{1}\]

  \[a^{-n}=\guessmath{\dfrac{1}{a^n}}\]
\end{frame}

\begin{frame}
  \frametitle{Règles de calcul (1)}
  \[a^p\times a^q = \guessmath{a^{p+q}}\]

  \[\dfrac{a^p}{a^q} = \guessmath{a^{p-q}}\]

  \[\left(a^p\right)^q = \guessmath{a^{p\times q}}\]

  \medskip

  \textit{Exemples. -- Écrire les nombres suivants sous la forme d'une puissance de $7$ :
    \begin{multicols}{4}
      \begin{enumerate}
	\item $7^5\times 7^{-2}\vphantom{\dfrac{1^3}{2^3}}$
	\item $7^{-4}\times 7\vphantom{\dfrac{1^3}{2^3}}$
	\item $\dfrac{7^{34}}{7^{21}}$
	\item $\left(7^3\right)^6\vphantom{\dfrac{1^3}{2^3}}$
      \end{enumerate} 
    \end{multicols}
  }
\end{frame}

\begin{frame}
  \frametitle{Règles de calcul (2)}
  \[a^p \times b^p = \guessmath{(ab)^p}\]

  \[\dfrac{a^p}{b^p} = \guessmath{\left(\dfrac{a}{b}\right)^p}\]

  \medskip

  \textit{Exemples. -- Proposer un exemple d'application de chacune des règles précédentes.\rep{6}}
\end{frame}

\end{document}

%%% Local Variables:
%%% mode: latex
%%% TeX-master: t
%%% End:
