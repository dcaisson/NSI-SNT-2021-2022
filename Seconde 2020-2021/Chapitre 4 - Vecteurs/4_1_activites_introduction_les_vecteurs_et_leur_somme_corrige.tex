\documentclass[a4paper]{article}

\input ../header
\usepackage[np]{numprint}
\usepackage{xcolor}
\usepackage{booktabs}

\setlength{\multicolsep}{2pt}

\begin{document}

\title{Activités -- Somme de deux vecteurs -- Éléments de correction}

\pagestyle{empty}

\date{}
\author{}

\maketitle{}

% Math'x page 282
\exo \vspace{-2mm}
\begin{enumerate}
  \item On considère la figure ci-dessous :
    \begin{center}
      \newrgbcolor{wwzzff}{0.4 0.6 1.}
      \newrgbcolor{qqzzqq}{0. 0.6 0.}
      \NormalCoor
      \psset{xunit=0.7cm,yunit=0.7cm,algebraic=true,dimen=middle,dotstyle=o,dotsize=5pt 0,linewidth=1.pt,arrowsize=3pt 2,arrowinset=0.25}
      \begin{pspicture*}(0.,-3.)(17.,8.)
	\multips(0,-3)(0,1.0){12}{\psline[linestyle=dashed,linecap=1,dash=1.5pt 1.5pt,linewidth=0.4pt,linecolor=lightgray]{c-c}(0.,0)(17.,0)}
	\multips(0,0)(1.0,0){18}{\psline[linestyle=dashed,linecap=1,dash=1.5pt 1.5pt,linewidth=0.4pt,linecolor=lightgray]{c-c}(0,-3.)(0,8.)}
	\pspolygon[linewidth=1.pt,linecolor=wwzzff,fillcolor=wwzzff,fillstyle=solid,opacity=0.1](1.,1.)(3.,1.)(3.,2.)(2.,2.)(2.,4.)(1.,4.)
	\pspolygon[linewidth=1.pt,linecolor=qqzzqq,fillcolor=qqzzqq,fillstyle=solid,opacity=0.1](7.,3.)(9.,3.)(9.,4.)(8.,4.)(8.,6.)(7.,6.)
	\pspolygon[linewidth=1.pt,linecolor=red,fillcolor=red,fillstyle=solid,opacity=0.1](14.,0.)(16.,0.)(16.,1.)(15.,1.)(15.,3.)(14.,3.)
	\pspolygon[linewidth=1.pt,fillcolor=black,fillstyle=solid,opacity=0.1](8.,-2.)(10.,-2.)(10.,-1.)(9.,-1.)(9.,1.)(8.,1.)
	\psline[linewidth=1.pt,linecolor=wwzzff](1.,1.)(3.,1.)
	\psline[linewidth=1.pt,linecolor=wwzzff](3.,1.)(3.,2.)
	\psline[linewidth=1.pt,linecolor=wwzzff](3.,2.)(2.,2.)
	\psline[linewidth=1.pt,linecolor=wwzzff](2.,2.)(2.,4.)
	\psline[linewidth=1.pt,linecolor=wwzzff](2.,4.)(1.,4.)
	\psline[linewidth=1.pt,linecolor=wwzzff](1.,4.)(1.,1.)
	\psline[linewidth=1.pt,linecolor=red]{->}(1.,5.)(7.,7.)
	\psline[linewidth=1.pt,linecolor=red]{->}(4.,5.)(11.,2.)
	\psline[linewidth=1.pt,linecolor=blue,linestyle=dashed]{->}(1.,4.)(7.,6.)
	\psline[linewidth=1.pt,linecolor=blue,linestyle=dashed]{->}(7.,6.)(14.,3.)
	\psline[linewidth=1.pt,linecolor=blue,linestyle=dashed]{->}(1.,4.)(8.,1.)
	\psline[linewidth=1.pt,linecolor=blue]{->}(1,4)(14.,3.)
	\psline[linewidth=1.pt,linecolor=qqzzqq](7.,3.)(9.,3.)
	\psline[linewidth=1.pt,linecolor=qqzzqq](9.,3.)(9.,4.)
	\psline[linewidth=1.pt,linecolor=qqzzqq](9.,4.)(8.,4.)
	\psline[linewidth=1.pt,linecolor=qqzzqq](8.,4.)(8.,6.)
	\psline[linewidth=1.pt,linecolor=qqzzqq](8.,6.)(7.,6.)
	\psline[linewidth=1.pt,linecolor=qqzzqq](7.,6.)(7.,3.)
	\psline[linewidth=1.pt,linecolor=red](14.,0.)(16.,0.)
	\psline[linewidth=1.pt,linecolor=red](16.,0.)(16.,1.)
	\psline[linewidth=1.pt,linecolor=red](16.,1.)(15.,1.)
	\psline[linewidth=1.pt,linecolor=red](15.,1.)(15.,3.)
	\psline[linewidth=1.pt,linecolor=red](15.,3.)(14.,3.)
	\psline[linewidth=1.pt,linecolor=red](14.,3.)(14.,0.)
	\psline[linewidth=1.pt](8.,-2.)(10.,-2.)
	\psline[linewidth=1.pt](10.,-2.)(10.,-1.)
	\psline[linewidth=1.pt](10.,-1.)(9.,-1.)
	\psline[linewidth=1.pt](9.,-1.)(9.,1.)
	\psline[linewidth=1.pt](9.,1.)(8.,1.)
	\psline[linewidth=1.pt](8.,1.)(8.,-2.)
	\uput[u](4,6){\color{red}$\vect{u}$}
	\uput[d](11,2){\color{red}$\vect{v}$}
	\uput[u](4,5){\color{blue}$\vect{u}$}
	\uput[u](10.5,4.5){\color{blue}$\vect{v}$}
	\uput[d](6,3.6){\color{blue}$\vect{w}$}
      \end{pspicture*}
    \end{center}
    \begin{enumerate}
      \item Tracer en vert l'image du motif bleu par la translation de vecteur $\vect{u}$.
      \item Tracer en rouge l'image du motif vert par la translation de vecteur $\vect{v}$.
      \item Peut-on trouver une translation qui transforme directement le motif bleu en le motif rouge ? Si oui, tracer le vecteur $\vect{w}$ associé à cette translation.
	\begin{center}
	  {\color{red}On apprendra que le vecteur $\vect{w}$ de cette translation (celle qui transforme directement le motif bleu en le motif rouge) est la \textbf{somme} des deux vecteurs $\vect{u}$ et $\vect{v}$ : on écrira \fbox{$\vect{w}=\vect{u}+\vect{v}$}.}
	\end{center}
    \end{enumerate}
  \item 
    \begin{enumerate}
      \item Tracer en noir l'image du motif bleu par la translation de vecteur $\vect{v}$.
      \item Quelle est l'image du motif noir par la translation de vecteur $\vect{u}$ ?
	\begin{center}
	  {\color{red}L'image du motif noir par la translation de vecteur $\vect{u}$ est le motif rouge.}
	\end{center}
    \end{enumerate}
  \item Que constate-t-on ?
    \begin{center}
      {\color{red}Lorsqu'on \og{}enchaîne\fg{} les deux translations, l'ordre n'a pas d'importance.}
    \end{center}
\end{enumerate}

\bigskip

% Déclic page 128
\exo Paul et Baptiste tentent d'arracher un poteau en tirant dessus. On donne ci-dessous les positions du poteau $P$, de Paul et de Baptiste.

\medskip

Les vecteurs $\vect{F}_1$ et $\vect{F}_2$ représentent les forces exercées respectivement par Paul et Baptiste sur le poteau.

\begin{center}
  \NormalCoor
  \psset{xunit=1.0cm,yunit=1.0cm,algebraic=true,dimen=middle,dotstyle=o,dotsize=5pt 0,linewidth=1.pt,arrowsize=3pt 2,arrowinset=0.25}
  \begin{pspicture*}(0.,0.)(10.,5.)
    \multips(0,0)(0,1.0){6}{\psline[linestyle=dashed,linecap=1,dash=1.5pt 1.5pt,linewidth=0.4pt,linecolor=lightgray]{c-c}(0.,0)(10.,0)}
    \multips(0,0)(1.0,0){11}{\psline[linestyle=dashed,linecap=1,dash=1.5pt 1.5pt,linewidth=0.4pt,linecolor=lightgray]{c-c}(0,0.)(0,5.)}
    \psline[linecolor=red,linewidth=1.pt]{->}(2.,4.)(5.,4.)
    \psline[linecolor=blue,linewidth=1.pt]{->}(2.,4.)(4.,2.)
    \psline[linewidth=1.pt,linestyle=dashed,dash=1pt 1pt](5.,4.)(8.,4.)
    \psline[linewidth=1.pt,linestyle=dashed,dash=1pt 1pt](4.,2.)(5.,1.)
    \uput[u](3.5,4){\color{red}$\vect{F}_1$}
    \uput[d](3,3){\color{blue}$\vect{F}_2$}
    \psdots[dotstyle=x](8,4)(5,1)
    \uput[u](2,4){$P$}
    \uput[u](8,4){\text{Paul}}
    \uput[d](5,1){\text{Baptiste}}
    \uput[u](5,4){$A$}
    \psline[linecolor=blue,linewidth=1.pt]{->}(5.,4.)(7.,2.)
    \uput[u](6,3){\color{blue}$\vect{F}_2$}
    \uput[r](7,2){$B=F$}
    \psline[linecolor=red,linewidth=1.pt]{->}(4.,2.)(7.,2.)
    \uput[d](5.5,2){\color{red}$\vect{F}_1$}
    \uput[d](4,2){$E$}
  \end{pspicture*} 
\end{center}

\bigskip

\begin{enumerate}
  \item 
    \begin{enumerate}
      \item Construire l'image $A$ du point $P$ par la translation de vecteur $\vect{F}_1$, puis l'image $B$ de $A$ par la translation de vecteur $\vect{F}_2$.
      \item Construire l'image $E$ du point $P$ par la translation de vecteur $\vect{F}_2$, puis l'image $F$ de $E$ par la translation de vecteur $\vect{F}_1$.
      \item Que constate-t-on ?
	\begin{center}
	  {\color{red}Les points $B$ et $F$ sont confondus.}
	\end{center}
      \item Dans quelle direction, le poteau va-t-il tomber ?
	\begin{center}
	  {\color{red}Le poteau va tomber dans la direction de la droite $(PB)$.}
	\end{center}
    \end{enumerate}
  \item Compléter les égalités suivantes :
    \[\vect{PA}+\vect{\color{red}AB}=\vect{P\color{red}B}\]
    \[\vect{PE}+\vect{\color{red}EF}=\vect{\color{red}PF}\]

    \bigskip

    {\color{red}La première égalité peut se comprendre de la façon \og{}intuitive\fg{} suivante :
      \begin{itemize}
	\item si l'on se rend du point $P$ au point $A$ (vecteur $\vect{PA}$),
	\item puis (addition $+$) du point $A$ au point $B$ (vecteur $\vect{AB}$),
      \end{itemize}
      alors on s'est rendu du point $P$ au point $B$ (vecteur $\vect{PB}$).
      \bigskip
      \bigskip
    }
    \begin{multicols}{2}
      {\color{red}Nous verrons plus tard que, peu importe les points $A$, $B$ et $C$ :
	\begin{itemize}
	  \item si l'on se rend du point $A$ au point $B$ (vecteur $\vect{AB}$),
	  \item puis (addition $+$) du point $B$ au point $C$ (vecteur $\vect{BC}$),
	\end{itemize}
	on s'est rendu du point $A$ au point $C$ (vecteur $\vect{AC}$).
	Autrement dit :
	\[\vect{AB}+\vect{BC}=\vect{AC}.\]
	Cette égalité est ce qu'on appelle la relation~de~Chasles.
	\begin{center}
	  \NormalCoor
	  \psset{xunit=1.0cm,yunit=1.0cm,algebraic=true,dimen=middle,dotstyle=o,dotsize=5pt 0,linewidth=1.pt,arrowsize=3pt 2,arrowinset=0.25}
	  \begin{pspicture*}(0.,0.)(6.,4.)
	    \psline[linecolor=blue,linewidth=1.pt]{->}(1.,1.)(3.,3.)
	    \psline[linecolor=blue,linewidth=1.pt]{->}(3.,3.)(5.,2.)
	    \psline[linecolor=blue,linewidth=1.pt]{->}(1.,1.)(5.,2.)
	    \uput[d](1,1){\color{blue}$A$}
	    \uput[u](3,3){\color{blue}$B$}
	    \uput[r](5,2){\color{blue}$C$}
	  \end{pspicture*}
	\end{center}
      }
    \end{multicols}
\end{enumerate}

\end{document}
