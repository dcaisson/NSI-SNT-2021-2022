\documentclass[a4paper,dvipsnames]{article}

\input ../header
\usepackage{gensymb}
\usepackage[np]{numprint}
\usepackage{xcolor}
\usepackage{booktabs}
\usepackage{tikz}
\def\checkmark{\tikz\fill[scale=0.4](0,.35) -- (.25,0) -- (1,.7) -- (.25,.15) -- cycle;}

\setlength{\multicolsep}{2pt}

\begin{document}

\title{Activités -- Produit d'un vecteur par un nombre réel -- Éléments de correction}

\pagestyle{empty}

\date{}
\author{}

\maketitle{}

% Indice page 121
\exo Un vecteur $\vect{u}$ est représenté ci-dessous. Le vecteur opposé de $\vect{u}$, noté $-\vect{u}$, est également représenté.
\begin{center}
  \NormalCoor
  \psset{xunit=1.0cm,yunit=1.0cm,algebraic=true,dimen=middle,dotstyle=o,dotsize=5pt 0,linewidth=1.pt,arrowsize=3pt 2,arrowinset=0.25}
  \begin{pspicture*}(0.,0.)(16.,9.)
    \multips(0,0)(0,1.0){10}{\psline[linestyle=dashed,linecap=1,dash=1.5pt 1.5pt,linewidth=0.4pt,linecolor=lightgray]{c-c}(0.,0)(16.,0)}
    \multips(0,0)(1.0,0){17}{\psline[linestyle=dashed,linecap=1,dash=1.5pt 1.5pt,linewidth=0.4pt,linecolor=lightgray]{c-c}(0,0.)(0,9.)}
    \psline[linewidth=1.pt,linecolor=red]{->}(2.,5.)(4.,6.)
    \uput[u](3,5.5){$\color{red}\vect{u}$}
    \psline[linecolor=CornflowerBlue,linewidth=1.pt]{->}(5.,5.)(9.,7.)
    \uput[u](6.5,6){$\color{CornflowerBlue}\vect{w}=2\vect{u}$}
    \psline[linecolor=ForestGreen,linewidth=1.pt]{->}(9.,5.)(15.,8.)
    \uput[u](11.5,6.5){$\color{ForestGreen}\vect{a}=3\vect{u}$} 
    \psline[linecolor=blue,linewidth=1.pt]{->}(15.,4.)(13.,3.)
    \uput[u](14,3.5){$\color{blue}-\vect{u}$}
    \psline[linecolor=BurntOrange,linewidth=1.pt]{->}(12.,4.)(8.,2.)
    \uput[ul](10,3){\color{BurntOrange}$\vect{b}=-2\vect{u}$}
    \psline[linecolor=Purple,linewidth=1.pt]{->}(7.,4.)(1.,1.)
    \uput[u](3.2,2.5){$\color{Purple}\vect{c}=-3\vect{u}$}
  \end{pspicture*}
\end{center}
\begin{enumerate}
  \item Représenter, sur la figure précédente, le vecteur $\vect{w}=\vect{u}+\vect{u}$. Ce vecteur $\vect{w}$ est noté $2\vect{u}$ :
    \[\vect{w}=2\vect{u}.\]
    Les vecteurs $\vect{w}$ et $\vect{u}$ ont la même direction et le même sens. La longueur de $\vect{w}$ est égale à $2$ fois la longueur de $\vect{u}$.
  \item Représenter le vecteur $\vect{a}=\vect{u}+\vect{u}+\vect{u}$. Ce vecteur $\vect{a}$ est noté ${\color{red}3}\vect{u}$ :
    \[\vect{a}={\color{red}3}\vect{u}.\]
    Les vecteurs $\vect{a}$ et $\vect{u}$ ont {\color{red}la même direction et le même sens.}\\
    La longueur du vecteur $\vect{a}$ est égale à $\color{red}3$ fois celle du vecteur $\vect{u}$.
  \item Représenter le vecteur $\vect{b}=(-\vect{u})+(-\vect{u})$. Ce vecteur $\vect{b}$ est noté $-2\vect{u}$ :
    \[\vect{b}=-2\vect{u}.\]
    Les vecteurs $\vect{b}$ et $\vect{u}$ ont la même direction mais des sens contraires. La longueur de $\vect{b}$ est égale à $2$ fois la longueur de $\vect{u}$.
  \item Représenter le vecteur $\vect{c}=(-\vect{u})+(-\vect{u})+(-\vect{u})$. Ce vecteur $\vect{c}$ est noté ${\color{red}-3}\vect{u}$ :
    \[\vect{c}={\color{red}-3}\vect{u}.\]
    Les vecteurs $\vect{c}$ et $\vect{u}$ ont {\color{red}la même direction mais des sens contraires.} La longueur de $\vect{c}$ est égale à {\color{red}$3$ fois celle de $\vect{u}$.}
\end{enumerate}

\pagebreak

% Suite de Indice page 121
\exo On a représenté ci-dessous plusieurs vecteurs. 
\begin{center}
  \NormalCoor
  \psset{xunit=1.0cm,yunit=1.0cm,algebraic=true,dimen=middle,dotstyle=o,dotsize=5pt 0,linewidth=1.pt,arrowsize=3pt 2,arrowinset=0.25}
  \begin{pspicture*}(0.,0.)(13.,6.)
    \multips(0,0)(0,1.0){7}{\psline[linestyle=dashed,linecap=1,dash=1.5pt 1.5pt,linewidth=0.4pt,linecolor=lightgray]{c-c}(0.,0)(13.,0)}
    \multips(0,0)(1.0,0){14}{\psline[linestyle=dashed,linecap=1,dash=1.5pt 1.5pt,linewidth=0.4pt,linecolor=lightgray]{c-c}(0,0.)(0,6.)}
    \psline[linewidth=1.pt]{->}(5.,5.)(9.,5.)
    \uput[u](7,5){$\vect{u}$}
    \psline[linewidth=1.pt,linecolor=red]{->}(5.,4.)(6.,4.)
    \uput[u](5.5,4){$\color{red}\vect{u}_1$}
    \psline[linewidth=1.pt,linecolor=blue]{->}(8.,3.)(6.,3.)
    \uput[u](7,3){$\color{blue}\vect{u}_2$}
    \psline[linewidth=1.pt,linecolor=ForestGreen]{->}(6.,2.)(12.,2.)
    \uput[u](9,2){$\color{ForestGreen}\vect{u}_3$}
    \psline[linewidth=1.pt,linecolor=Goldenrod]{->}(10.,1.)(2.,1.)
    \uput[u](6,1){$\color{Goldenrod}\vect{u}_4$}
  \end{pspicture*}
\end{center}

Compléter le tableau ci-dessous :

\medskip

\begin{center}
  \begin{tabular}{@{}ccccc@{}}
    \toprule
	  & la même direction & le même sens & des sens contraires & égalité \\
	  \cmidrule(lr){2-4} \cmidrule(lr){5-5}
	  \addlinespace[5pt]
    $\vect{u}$ et $\color{red}\vect{u}_1$ ont: & \checkmark & \checkmark && ${\color{red}\vect{u}_1}={\color{red}\dfrac{1}{4}}\vect{u}$\\
    \addlinespace[5pt]
    $\vect{u}$ et $\color{blue}\vect{u}_2$ ont: & {\color{red}\checkmark} && {\color{red}\checkmark} & ${\color{blue}\vect{u}_2}={\color{red}-\dfrac{1}{2}}\vect{u}$\\
    \addlinespace[5pt]
    $\vect{u}$ et $\color{ForestGreen}\vect{u}_3$ ont: & {\color{red}\checkmark} & {\color{red}\checkmark} && ${\color{ForestGreen}\vect{u}_3}={\color{red}1,5}\vect{u}$\\
    \addlinespace[5pt]
    $\vect{u}$ et $\color{Goldenrod}\vect{u}_4$ ont: & {\color{red}\checkmark} && {\color{red}\checkmark} & ${\color{Goldenrod}\vect{u}_4}={\color{red}-2}\vect{u}$\\
    \bottomrule
  \end{tabular}
\end{center}

\end{document}
