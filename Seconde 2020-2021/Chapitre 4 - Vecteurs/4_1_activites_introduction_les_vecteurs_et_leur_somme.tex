\documentclass[a4paper]{article}

\input ../header
\usepackage{gensymb}
\usepackage[np]{numprint}
\usepackage{xcolor}
\usepackage{booktabs}

\setlength{\multicolsep}{2pt}

\begin{document}

\title{Activités -- Somme de deux vecteurs}
\author{}

\pagestyle{empty}

\date{}

\maketitle{}

% Math'x page 282
\exo \vspace{-2mm}
\begin{enumerate}
  \item On considère la figure ci-dessous :
    \begin{center}
      \newrgbcolor{wwzzff}{0.4 0.6 1.}
      \newrgbcolor{qqzzqq}{0. 0.6 0.}
      \psset{xunit=0.7cm,yunit=0.7cm,algebraic=true,dimen=middle,dotstyle=o,dotsize=5pt 0,linewidth=1.pt,arrowsize=3pt 2,arrowinset=0.25}
      \begin{pspicture*}(0.,-3.)(17.,8.)
	\multips(0,-3)(0,1.0){12}{\psline[linestyle=dashed,linecap=1,dash=1.5pt 1.5pt,linewidth=0.4pt,linecolor=lightgray]{c-c}(0.,0)(17.,0)}
	\multips(0,0)(1.0,0){18}{\psline[linestyle=dashed,linecap=1,dash=1.5pt 1.5pt,linewidth=0.4pt,linecolor=lightgray]{c-c}(0,-3.)(0,8.)}
	\pspolygon[linewidth=1.pt,linecolor=wwzzff,fillcolor=wwzzff,fillstyle=solid,opacity=0.1](1.,1.)(3.,1.)(3.,2.)(2.,2.)(2.,4.)(1.,4.)
	%\pspolygon[linewidth=1.pt,linecolor=qqzzqq,fillcolor=qqzzqq,fillstyle=solid,opacity=0.1](7.,3.)(9.,3.)(9.,4.)(8.,4.)(8.,6.)(7.,6.)
	%\pspolygon[linewidth=1.pt,linecolor=red,fillcolor=red,fillstyle=solid,opacity=0.1](14.,0.)(16.,0.)(16.,1.)(15.,1.)(15.,3.)(14.,3.)
	%\pspolygon[linewidth=1.pt,fillcolor=black,fillstyle=solid,opacity=0.1](8.,-2.)(10.,-2.)(10.,-1.)(9.,-1.)(9.,1.)(8.,1.)
	\psline[linewidth=1.pt,linecolor=wwzzff](1.,1.)(3.,1.)
	\psline[linewidth=1.pt,linecolor=wwzzff](3.,1.)(3.,2.)
	\psline[linewidth=1.pt,linecolor=wwzzff](3.,2.)(2.,2.)
	\psline[linewidth=1.pt,linecolor=wwzzff](2.,2.)(2.,4.)
	\psline[linewidth=1.pt,linecolor=wwzzff](2.,4.)(1.,4.)
	\psline[linewidth=1.pt,linecolor=wwzzff](1.,4.)(1.,1.)
	\psline[linewidth=1.pt,linecolor=red]{->}(1.,5.)(7.,7.)
	\psline[linewidth=1.pt,linecolor=red]{->}(4.,5.)(11.,2.)
	%\psline[linewidth=1.pt,linecolor=qqzzqq](7.,3.)(9.,3.)
	%\psline[linewidth=1.pt,linecolor=qqzzqq](9.,3.)(9.,4.)
	%\psline[linewidth=1.pt,linecolor=qqzzqq](9.,4.)(8.,4.)
	%\psline[linewidth=1.pt,linecolor=qqzzqq](8.,4.)(8.,6.)
	%\psline[linewidth=1.pt,linecolor=qqzzqq](8.,6.)(7.,6.)
	%\psline[linewidth=1.pt,linecolor=qqzzqq](7.,6.)(7.,3.)
	%\psline[linewidth=1.pt,linecolor=red](14.,0.)(16.,0.)
	%\psline[linewidth=1.pt,linecolor=red](16.,0.)(16.,1.)
	%\psline[linewidth=1.pt,linecolor=red](16.,1.)(15.,1.)
	%\psline[linewidth=1.pt,linecolor=red](15.,1.)(15.,3.)
	%\psline[linewidth=1.pt,linecolor=red](15.,3.)(14.,3.)
	%\psline[linewidth=1.pt,linecolor=red](14.,3.)(14.,0.)
	%\psline[linewidth=1.pt](8.,-2.)(10.,-2.)
	%\psline[linewidth=1.pt](10.,-2.)(10.,-1.)
	%\psline[linewidth=1.pt](10.,-1.)(9.,-1.)
	%\psline[linewidth=1.pt](9.,-1.)(9.,1.)
	%\psline[linewidth=1.pt](9.,1.)(8.,1.)
	%\psline[linewidth=1.pt](8.,1.)(8.,-2.)
	\uput[u](4,6){\color{red}$\vect{u}$}
	\uput[u](11,2){\color{red}$\vect{v}$}
      \end{pspicture*}
    \end{center}
    \begin{enumerate}
      \item Tracer en vert l'image du motif bleu par la translation de vecteur $\vect{u}$.
      \item Tracer en rouge l'image du motif vert par la translation de vecteur $\vect{v}$.
      \item Peut-on trouver une translation qui transforme directement le motif bleu en le motif rouge ? Si oui, tracer le vecteur $\vect{w}$ associé à cette translation.\rep{6}
    \end{enumerate}
  \item 
    \begin{enumerate}
      \item Tracer en noir l'image du motif bleu par la translation de vecteur $\vect{v}$.
      \item Quelle est l'image du motif noir par la translation de vecteur $\vect{u}$ ?\rep{6}
    \end{enumerate}
  \item Que constate-t-on ?\rep{6}
\end{enumerate}

\pagebreak

% Déclic page 128
\exo Paul et Baptiste tentent d'arracher un poteau en tirant dessus. On donne ci-dessous les positions du poteau $P$, de Paul et de Baptiste.

\medskip

Les vecteurs $\vect{F}_1$ et $\vect{F}_2$ représentent les forces exercées respectivement par Paul et Baptiste sur le poteau.

\begin{center}
  \psset{xunit=1.0cm,yunit=1.0cm,algebraic=true,dimen=middle,dotstyle=o,dotsize=5pt 0,linewidth=1.pt,arrowsize=3pt 2,arrowinset=0.25}
  \begin{pspicture*}(0.,0.)(10.,5.)
    \multips(0,0)(0,1.0){6}{\psline[linestyle=dashed,linecap=1,dash=1.5pt 1.5pt,linewidth=0.4pt,linecolor=lightgray]{c-c}(0.,0)(10.,0)}
    \multips(0,0)(1.0,0){11}{\psline[linestyle=dashed,linecap=1,dash=1.5pt 1.5pt,linewidth=0.4pt,linecolor=lightgray]{c-c}(0,0.)(0,5.)}
    \psline[linecolor=red,linewidth=1.pt]{->}(2.,4.)(5.,4.)
    \psline[linecolor=blue,linewidth=1.pt]{->}(2.,4.)(4.,2.)
    \psline[linewidth=1.pt,linestyle=dashed,dash=1pt 1pt](5.,4.)(8.,4.)
    \psline[linewidth=1.pt,linestyle=dashed,dash=1pt 1pt](4.,2.)(5.,1.)
    \uput[u](3.5,4){\color{red}$\vect{F}_1$}
    \uput[d](3,3){\color{blue}$\vect{F}_2$}
    \psdots[dotstyle=x](8,4)(5,1)
    \uput[u](2,4){$P$}
    \uput[u](8,4){\text{Paul}}
    \uput[d](5,1){\text{Baptiste}}
  \end{pspicture*} 
\end{center}

\bigskip

\begin{enumerate}
  \item 
    \begin{enumerate}
      \item Construire l'image $A$ du point $P$ par la translation de vecteur $\vect{F}_1$, puis l'image $B$ de $A$ par la translation de vecteur $\vect{F}_2$.
      \item Construire l'image $E$ du point $P$ par la translation de vecteur $\vect{F}_2$, puis l'image $F$ de $E$ par la translation de vecteur $\vect{F}_1$.
      \item Que constate-t-on ?\rep{8}
      \item Dans quelle direction, le poteau va-t-il tomber ?\rep{8}
    \end{enumerate}
  \item Compléter les égalités suivantes :
    \[\vect{PA}+\vect{\vphantom{A}.....}=\vect{P...}\]
    \[\vect{PE}+\vect{\vphantom{A}.....}=\vect{\vphantom{A}.....}\]
\end{enumerate}

\end{document}
