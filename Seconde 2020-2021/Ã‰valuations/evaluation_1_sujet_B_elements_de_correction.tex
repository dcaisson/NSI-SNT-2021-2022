\documentclass[a4paper,dvipsnames]{article}

\input ../header
\newcommand{\checkedbox}{\makebox[0pt][l]{$\square$}\raisebox{.15ex}{\hspace{0.1em}$\checkmark$}}
\newcommand{\checkbox}{\makebox[0pt][l]{$\square$}\raisebox{.15ex}{\hspace{0.1em}}\hspace{3mm}}

\begin{document}

\title{Évaluation 1 -- Sujet B -- Éléments de correction}

\pagestyle{empty}

\date{}
\author{}

\maketitle{}

\setcounter{exercice}{1}

\exo[3 points]
Compléter en utilisant $\in$ ou $\notin$ :
\begin{multicols}{3}
  \begin{enumerate}
    \item $-2\in\mathbb{Z}\vphantom{\dfrac{1}{3}}$
    \item $\dfrac{9}{4}\in\mathbb{D}$
    \item $\dfrac{4}{7}\in\mathbb{Q}$
    \item $\sqrt{25}\in\mathbb{N}\vphantom{\dfrac{9}{4}}$
    \item $\np{1492}\times10^{-8}\in\mathbb{D}\vphantom{\dfrac{1}{3}}$
    \item $\sqrt{7}\notin\mathbb{D}\vphantom{\dfrac{1}{3}}$
  \end{enumerate}
\end{multicols}

\bigskip

\exo[2 points]
Compléter le tableau suivant en utilisant V (vrai) ou F (faux) :
\begin{center}
  \begin{tabular}{@{}cccccc@{}}
    \toprule
    & $\in\mathbb{N}$ & $\in\mathbb{Z}$ & $\in\mathbb{D}$ & $\in\mathbb{Q}$ & $\in\mathbb{R}$\\
    \cmidrule(lr){2-6}
    $\dfrac{10}{5}$ & V & V & V & V & V\\
    \addlinespace[2mm]
    $\sqrt{3}$ & F & F & F & F & V\\
    \addlinespace[1mm]
    $9\times10^{12}$ & V & V & V & V & V\\
    \addlinespace[1mm]
    $\np{1,3333}$ & F & F & V & V & V\\
    \bottomrule
  \end{tabular}
\end{center}

\bigskip

\exo[2 points] Compléter le schéma suivant avec les nombres de votre choix :

\begin{center}
  \psset{xunit=0.7cm,yunit=0.6cm,algebraic=true,dimen=middle,dotstyle=o,dotsize=5pt 0,linewidth=1.2pt,arrowsize=3pt 2,arrowinset=0.25}
  \begin{pspicture*}(-1.,-3.)(17.,11.)
    \rput{0.}(4.,4.){\psellipse[linecolor=red,linewidth=1.2pt](0,0)(1.9898733422917,1.7203476155600472)}
    \rput{0.}(5.,4.){\psellipse[linecolor=blue,linewidth=1.2pt](0,0)(3.6055512754639873,3.)}
    \rput{0.}(6.,4.){\psellipse[linecolor=orange,linewidth=1.2pt](0,0)(5.,4.)}
    \rput{0.}(7.,4.){\psellipse[linecolor=Green,linewidth=1.2pt](0,0)(6.403124237432849,5.)}
    \rput{0.}(8.,4.){\psellipse[linecolor=Fuchsia,linewidth=1.2pt](0,0)(7.810249675906654,6.)}
    \uput[u](5.4,5.4){\color{red}\fbox{$\mathbb{N}$}}
    \uput[u](7.6,6.4){\color{blue}\fbox{$\mathbb{Z}$}}
    \uput[u](9.2,7.3){\color{orange}\fbox{$\mathbb{D}$}}
    \uput[u](10.6,8.2){\color{Green}\fbox{$\mathbb{Q}$}}
    \uput[u](12,9.4){\color{Fuchsia}\fbox{$\mathbb{R}$}}
    \uput[u](4,4){$1$}
    \uput[u](7,4){$-6$}
    \uput[u](6.8,2.6){$-10$}
    \uput[u](9.8,3.6){$0,23$}
    \uput[u](9,2){$1,33$}
    \uput[u](12.2,3.5){$\dfrac{1}{3}$}
    \uput[u](11.5,1.7){$\dfrac{2}{7}$}
    \uput[u](14.3,2.1){$\sqrt{2}$}
  \end{pspicture*}
\end{center}

\bigskip

\exo[1 point] 
Dire, sans justifier, si chacune des phrases suivantes est vraie (V) ou fausse (F) :
\begin{enumerate}
  \item V \checkedbox{} F \checkbox{} Tout entier relatif est un nombre décimal. 
  \item V \checkedbox{} F \checkbox{} Il existe des nombres rationnels qui sont des entiers relatifs.
\end{enumerate}

\end{document}
