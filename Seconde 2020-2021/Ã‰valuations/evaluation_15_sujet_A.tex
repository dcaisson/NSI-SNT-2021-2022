\documentclass[a4paper]{article}

\input ../header
\setlength{\multicolsep}{2pt}

\begin{document}

\title{Évaluation 15 -- Sujet A}

\pagestyle{empty}

\date{}
\author{}

\maketitle{}

\thispagestyle{empty}

\exo[6 points] Cet exercice est un questionnaire à choix multiples (QCM). Pour chaque question posée, une seule des réponses proposées est correcte. Une réponse juste rapporte $1$ point ; une réponse fausse ou l'absence de réponse ne rapporte ni n'enlève de point. Répondre sur le sujet.

\medskip

Dans un repère, $\mathcal{D}$ est la droite d'équation $-7x-2y-10=0$.

\begin{enumerate}
  \item La droite $\mathcal{D}$ passe par le point de coordonnées :
  \begin{multicols}{4}
    \begin{enumerate}
      \item $(5;0)\phantom{\left(\dfrac{10}{7}\right.}$
      \item $(0;5)\phantom{\left(\dfrac{10}{7}\right.}$
      \item $\left(0;-\dfrac{10}{7}\right)$
      \item $\left(-\dfrac{10}{7};0\right)$
    \end{enumerate}
  \end{multicols}
  \item Les coordonnées d'un vecteur directeur de $\mathcal{D}$ sont :
    \begin{multicols}{4}
      \begin{enumerate}
	\item $(-7;-2)$
	\item $(-2;-7)$
	\item $(2;-7)$
	\item $(2;7)$
      \end{enumerate}
    \end{multicols}
  \item L'équation réduite de $\mathcal{D}$ est :
    \begin{multicols}{4}
      \begin{enumerate}
	\item $y=-\dfrac{2}{7}x-\dfrac{10}{7}$
	\item $y=-\dfrac{7}{2}x-10$
	\item $y=-\dfrac{7}{2}x-5$
	\item $2y=-7x-10\phantom{\dfrac{7}{2}}$
      \end{enumerate}
    \end{multicols}
  \item Le coefficient directeur de $\mathcal{D}$ est :
    \begin{multicols}{4}
      \begin{enumerate}
	\item $\dfrac{7}{2}$
	\item $-\dfrac{7}{2}$
	\item $\dfrac{2}{7}$
	\item $-\dfrac{2}{7}$
      \end{enumerate}
    \end{multicols}
  \item Une autre équation cartésienne de $\mathcal{D}$ est :
    \begin{multicols}{4}
      \begin{enumerate}
	\item $7x+2y-10=0$
	\item $14x+4y+20=0$
	\item $-2x-7y-10=0$
	\item $-7x-2y=0$
      \end{enumerate}
    \end{multicols}
  \item La droite d'équation réduite $x=2$ est :
    \begin{multicols}{4}
      \begin{enumerate}
	\item parallèle à l'axe des ordonnées
	\item parallèle à l'axe des abscisses
	\item passe par l'origine du repère
	\item aucune des réponses précédentes
      \end{enumerate} 
    \end{multicols} 
\end{enumerate}

\smallskip

\exo[4 points] Dans un repère orthonormé, $\mathcal{D}$ est la droite de vecteur directeur $\vect{u}(-5;2)$ et qui passe par le point $A(-1;1)$.
\begin{enumerate}
  \item Faire une figure.\rep{14}
  \item Déterminer une équation cartésienne de $\mathcal{D}$.\rep{6}
  \item Déterminer l'équation réduite de $\mathcal{D}$.\rep{6}
  \item Que vaut l'ordonnée à l'origine de $\mathcal{D}$ ?\rep{4}
\end{enumerate}

\smallskip

\exo[5 points] Soit $\mathcal{D}$ la droite d'équation réduite $y=-2x+3$. Les droites $\mathcal{D}_1$ et $\mathcal{D}_2$ ont pour équations réduites respectives $x=3$ et $y=-1$.
\begin{enumerate}
  \item Représenter les droites $\mathcal{D}$, $\mathcal{D}_1$ et $\mathcal{D}_2$.\rep{14}
  \item Donner un vecteur directeur de chacune des droites.\rep{4}
  \item Le point $A(3;-3)$ appartient-il à la droite $\mathcal{D}$ ?\rep{4}
  \item Déterminer deux équations cartésiennes de $\mathcal{D}$.\rep{6}
\end{enumerate}

\smallskip

\exo[5 points] Le plan est muni d'un repère orthonormé. On considère les points $A(-2;4)$ et $B(4;5)$. \textit{Dans cet exercice, aucune figure n'est exigée.}
\begin{enumerate}
  \item Déterminer les coordonnées du vecteur $\vect{AB}$.\rep{4}
  \item Donner un autre vecteur directeur de la droite $(AB)$.\rep{2}
  \item On rappelle que le vecteur $\vect{AB}$ est un vecteur directeur de la droite $(AB)$. Déterminer l'équation réduite de la droite $(AB)$.\rep{6}
  \item Quel est le coefficient directeur de la droite $(AB)$ ?\rep{2}
\end{enumerate}

\end{document}

%%% Local Variables:
%%% mode: latex
%%% TeX-master: t
%%% End:
