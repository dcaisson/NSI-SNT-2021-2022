\documentclass[a4paper]{tufte-handout}

\input header_seconde
\setlength{\multicolsep}{2pt}
\usepackage{booktabs}
\usepackage{hyperref}
\usepackage[np]{numprint}

\begin{document}

\title{Séance 10 du jeudi 7 mai -- Éléments de correction}

\pagestyle{empty}

\date{}

\maketitle{}

\begin{fullwidth}
  {\color{red}On rappelle la relation de Chasles :
    \begin{center}
      Quels que soient les points $A$, $B$ et $C$, \fbox{$\vect{AB}+\vect{BC}=\vect{AC}$}.
    \end{center}
  }
\end{fullwidth}

\bigskip

\begin{fullwidth}
  \exo Soient $A$, $B$ et $C$ trois points. En utilisant la relation de Chasles, compléter chacune des égalités suivantes :
  \begin{multicols}{4}
    \begin{enumerate}
      \item $\vect{BA}+\vect{{\color{red}A} C}=\vect{BC}$
      \item $\vect{AB}+\vect{B{\color{red}C}}=\vect{AC}$
      \item $\vect{C{\color{red}B}}+\vect{BA}=\vect{CA}$
      \item $\vect{{\color{red}B} C}+\vect{{\color{red}C} A}=\vect{BA}$
    \end{enumerate}
  \end{multicols}
\end{fullwidth}

\bigskip

\begin{fullwidth}
  \exo Sur chacune des figures suivantes, construire le vecteur $\vect{u}+\vect{v}$. On fera apparaître sur la construction les vecteurs $\vect{u}$ et $\vect{v}$ représentés \og{}bout à bout\fg{}.

  \medskip

  \begin{multicols}{2}
    \begin{enumerate}
      \item[]
	\NormalCoor
	\psset{xunit=0.7cm,yunit=0.7cm,algebraic=true,dimen=middle,dotstyle=o,dotsize=5pt 0,linewidth=1.pt,arrowsize=3pt 2,arrowinset=0.25}
	\begin{pspicture*}(0.,0.)(10.,10.)
	  \multips(0,0)(0,1.0){11}{\psline[linestyle=dashed,linecap=1,dash=1.5pt 1.5pt,linewidth=0.4pt,linecolor=lightgray]{c-c}(0.,0)(10.,0)}
	  \multips(0,0)(1.0,0){11}{\psline[linestyle=dashed,linecap=1,dash=1.5pt 1.5pt,linewidth=0.4pt,linecolor=lightgray]{c-c}(0,0.)(0,10.)}
	  \psline[linecolor=blue]{->}(1,1)(4,2)
	  \psline[linecolor=blue]{->}(6,2)(9,3)
	  \uput[d](7.5,2.5){$\color{blue}\vect{v}$}
	  \psline[linecolor=red]{->}(3,5)(6,2)
	  \psline[linecolor=ForestGreen]{->}(3,5)(9,3)
	  \uput[u](6,4){$\color{ForestGreen}\vect{u}+\vect{v}$}
	  \uput[d](2.5,1.5){$\color{blue}\vect{v}$}
	  \uput[dl](4.5,3.5){$\color{red}\vect{u}$}
	\end{pspicture*}
      \item[]
	\NormalCoor
	\psset{xunit=0.7cm,yunit=0.7cm,algebraic=true,dimen=middle,dotstyle=o,dotsize=5pt 0,linewidth=1.pt,arrowsize=3pt 2,arrowinset=0.25}
	\begin{pspicture*}(0.,0.)(10.,10.)
	  \multips(0,0)(0,1.0){11}{\psline[linestyle=dashed,linecap=1,dash=1.5pt 1.5pt,linewidth=0.4pt,linecolor=lightgray]{c-c}(0.,0)(10.,0)}
	  \multips(0,0)(1.0,0){11}{\psline[linestyle=dashed,linecap=1,dash=1.5pt 1.5pt,linewidth=0.4pt,linecolor=lightgray]{c-c}(0,0.)(0,10.)}
	  \psline[linecolor=blue]{->}(1,1)(2,6)
	  \psline[linecolor=red]{->}(4,4)(7,4)
	  \psline[linecolor=red]{->}(2,6)(5,6)
	  \uput[u](3.5,6){$\color{red}\vect{u}$}
	  \psline[linecolor=red]{->}(2,6)(5,6)
	  \psline[linecolor=ForestGreen]{->}(1,1)(5,6)
	  \uput[r](3,3.5){$\color{ForestGreen}\vect{u}+\vect{v}$}
	  \uput[l](1.5,3.5){$\color{blue}\vect{v}$}
	  \uput[u](5.5,4){$\color{red}\vect{u}$}
	\end{pspicture*}
      \item[]
	\NormalCoor
	\psset{xunit=0.7cm,yunit=0.7cm,algebraic=true,dimen=middle,dotstyle=o,dotsize=5pt 0,linewidth=1.pt,arrowsize=3pt 2,arrowinset=0.25}
	\begin{pspicture*}(0.,0.)(10.,10.)
	  \multips(0,0)(0,1.0){11}{\psline[linestyle=dashed,linecap=1,dash=1.5pt 1.5pt,linewidth=0.4pt,linecolor=lightgray]{c-c}(0.,0)(10.,0)}
	  \multips(0,0)(1.0,0){11}{\psline[linestyle=dashed,linecap=1,dash=1.5pt 1.5pt,linewidth=0.4pt,linecolor=lightgray]{c-c}(0,0.)(0,10.)}
	  \psline[linecolor=blue]{->}(3,8)(4,2)
	  \psline[linecolor=blue]{->}(6,10)(7,4)
	  \psline[linecolor=red]{->}(7,4)(1,2)
	  \psline[linecolor=ForestGreen]{->}(6,10)(1,2)
	  \uput[l](3,8){$\color{blue}\vect{v}$}
	  \uput[r](3.5,6){$\color{ForestGreen}\vect{u}+\vect{v}$}
	  \uput[r](6.5,7){$\color{blue}\vect{v}$}
	  \uput[d](5,3.33){$\color{red}\vect{u}$}
	\end{pspicture*}
      \item[]
	\NormalCoor
	\psset{xunit=0.7cm,yunit=0.7cm,algebraic=true,dimen=middle,dotstyle=o,dotsize=5pt 0,linewidth=1.pt,arrowsize=3pt 2,arrowinset=0.25}
	\begin{pspicture*}(0.,0.)(10.,10.)
	  \multips(0,0)(0,1.0){11}{\psline[linestyle=dashed,linecap=1,dash=1.5pt 1.5pt,linewidth=0.4pt,linecolor=lightgray]{c-c}(0.,0)(10.,0)}
	  \multips(0,0)(1.0,0){11}{\psline[linestyle=dashed,linecap=1,dash=1.5pt 1.5pt,linewidth=0.4pt,linecolor=lightgray]{c-c}(0,0.)(0,10.)}
	  \psline[linecolor=blue]{->}(3,3)(2,5)
	  \psline[linecolor=blue]{->}(4,2)(3,4)
	  \psline[linecolor=red]{->}(8,7)(4,2)
	  \psline[linecolor=ForestGreen]{->}(8,7)(3,4)
	  \uput[l](2.5,4){$\color{blue}\vect{v}$}
	  \uput[u](5,5.5){$\color{ForestGreen}\vect{u}+\vect{v}$}
	  \uput[r](3.5,3){$\color{blue}\vect{v}$}
	  \uput[l](6,4.5){$\color{red}\vect{u}$}
	\end{pspicture*}
    \end{enumerate}
  \end{multicols} 
\end{fullwidth}

\pagebreak

\begin{fullwidth}
  \exo En physique, la vitesse d'un solide à un instant $t$ est représentée par un vecteur $\vect{v}$ qui indique sa direction, son sens et sa norme (par exemple exprimée en mètres par seconde). On considère un yacht remorqué par deux bateaux et on note $\vect{v}_1$ et $\vect{v}_2$ les vecteurs vitesses de chaque remorqueur. On considère que ces vitesses sont constantes.

  \begin{center}
    \includegraphics[width=12cm]{seance_10_yacht_corrige.png} 
  \end{center}

  Représenter le vecteur vitesse $\vect{v}$ du yacht sachant que $\vect{v}$ est la somme des vecteurs $\vect{v}_1$ et $\vect{v}_2$.
\end{fullwidth}

\bigskip

\begin{fullwidth}
  \exo Un nageur part d'un point $A$ et nage vers la berge opposée. On note :
  \begin{itemize}
    \item $\vect{v}_1$ le vecteur vitesse instantanée du nageur ;
    \item $\vect{v}_2$ le vecteur vitesse instantanée du courant.
  \end{itemize}
  On considère que ces deux vitesses sont constantes.

  \begin{center}
    \includegraphics[width=12cm]{seance_10_nageur_corrige.png}  
  \end{center}

  Déterminer graphiquement le point sur la berge opposée où arrivera le nageur.
\end{fullwidth}


\end{document}
