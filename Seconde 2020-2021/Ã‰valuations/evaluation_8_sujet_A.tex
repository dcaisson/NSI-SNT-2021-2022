\documentclass[a4paper,dvipsnames]{article}

\input ../header
\newcommand{\checkedbox}{\makebox[0pt][l]{$\square$}\raisebox{.15ex}{\hspace{0.1em}$\checkmark$}}
\newcommand{\checkbox}{\makebox[0pt][l]{$\square$}\raisebox{.15ex}{\hspace{0.1em}}\hspace{3mm}}

% Intervalle
\def\intervalleFF(#1,#2){\psline[linecolor=red]{[-]}(#1,0)(#2,0)}
\def\intervalleOO(#1,#2){\psline[linecolor=red]{]-[}(#1,0)(#2,0)}
\def\intervalleFO(#1,#2){\psline[linecolor=red]{[-[}(#1,0)(#2,0)}
\def\intervalleOF(#1,#2){\psline[linecolor=red]{]-]}(#1,0)(#2,0)}
\def\intervalleFpI(#1,#2){\psline[linecolor=red]{[-}(#1,0)(#2,0)}
\def\intervalleOpI(#1,#2){\psline[linecolor=red]{]-}(#1,0)(#2,0)}
\def\intervalleFmI(#1,#2){\psline[linecolor=red]{-]}(#1,0)(#2,0)}
\def\intervalleOmI(#1,#2){\psline[linecolor=red]{-[}(#1,0)(#2,0)}

\begin{document}

\title{Évaluation 8 -- Sujet A}
\author{}
\date{}

\maketitle{}

\pagestyle{empty}
\thispagestyle{empty}

% Exercice sur les ensembles de nombres
\exo[2 points] 
Compléter les phrases suivantes :
\begin{enumerate}
  \item L'ensemble $\mathbb{N}$ est l'ensemble des \dotfill
  \item L'ensemble $\hdots\hdots\hdots$ est l'ensemble des nombres rationnels.
\end{enumerate}

\bigskip

% Exercice sur les ensembles de nombres
\exo[2 points] Compléter le schéma suivant avec les nombres de votre choix :

\begin{center}
  \psset{xunit=0.7cm,yunit=0.6cm,algebraic=true,dimen=middle,dotstyle=o,dotsize=5pt 0,linewidth=1.2pt,arrowsize=3pt 2,arrowinset=0.25}
  \begin{pspicture*}(-1.,-3.)(17.,11.)
    \rput{0.}(4.,4.){\psellipse[linecolor=red,linewidth=1.2pt](0,0)(1.9898733422917,1.7203476155600472)}
    \rput{0.}(5.,4.){\psellipse[linecolor=blue,linewidth=1.2pt](0,0)(3.6055512754639873,3.)}
    \rput{0.}(6.,4.){\psellipse[linecolor=orange,linewidth=1.2pt](0,0)(5.,4.)}
    \rput{0.}(7.,4.){\psellipse[linecolor=Green,linewidth=1.2pt](0,0)(6.403124237432849,5.)}
    \rput{0.}(8.,4.){\psellipse[linecolor=Fuchsia,linewidth=1.2pt](0,0)(7.810249675906654,6.)}
    \uput[u](5.4,5.4){\color{red}\fbox{$\mathbb{N}$}}
    \uput[u](7.6,6.4){\color{blue}\fbox{$\mathbb{Z}$}}
    \uput[u](9.2,7.3){\color{orange}\fbox{$\mathbb{D}$}}
    \uput[u](10.6,8.2){\color{Green}\fbox{$\mathbb{Q}$}}
    \uput[u](12,9.4){\color{Fuchsia}\fbox{$\mathbb{R}$}}
    \uput[u](4,4){$\hdots$}
    \uput[u](7,4){$\hdots$}
%    \uput[u](6.8,2.6){$\hdots$}
    \uput[u](9.8,3.6){$\hdots$}
%    \uput[u](9,2){$\hdots$}
    \uput[u](12.2,3.5){$\hdots$}
%    \uput[u](11.5,1.7){$\hdots$}
    \uput[u](14.3,2.1){$\hdots$}
  \end{pspicture*}
\end{center} 

\bigskip

% Exercice sur la valeur absolue, les encadrements et les intervalles
\exo[3 points] Compléter le tableau suivant :
\begin{center}
  \begin{tabular}{@{}>{\centering}p{3.8cm}cp{4.5cm}@{}}
    \toprule
    Encadrement / Inégalité & Représentation & Appartenance à un intervalle\\
    \midrule
    \addlinespace[2mm]
    \vspace*{-6mm}$-10<x\leq21$ & \psset{xunit=0.5cm,yunit=0.5cm,algebraic=true,dimen=middle,dotstyle=o,dotsize=5pt 0,linewidth=1.2pt,arrowsize=3pt 2,arrowinset=0.25}
    \begin{pspicture*}(1.,-1)(8.,.5)
      \psaxes[labelFontSize=\scriptstyle,xAxis=true,yAxis=true,Dx=10.,Dy=1.,ticksize=-2pt 0]{->}(0,0)(1.,-1.)(8.,1.)
      \intervalleOF(3,6)
      \uput[d](3,-0.1){$\color{red}-10$}
      \uput[d](6,-0.1){$\color{red}21$}
    \end{pspicture*} & \vspace*{-6mm}\centering{}$x\in]-10;21]$\tabularnewline
    \addlinespace[2mm]
				& \psset{xunit=0.5cm,yunit=0.5cm,algebraic=true,dimen=middle,dotstyle=o,dotsize=5pt 0,linewidth=1.2pt,arrowsize=3pt 2,arrowinset=0.25}
    \begin{pspicture*}(1.,-1)(8.,.5)
      \psaxes[labelFontSize=\scriptstyle,xAxis=true,yAxis=true,Dx=10.,Dy=1.,ticksize=-2pt 0]{->}(0,0)(1.,-1.)(8.,1.)
      \intervalleFF(3,6)
      \uput[d](3,-0.1){$\color{red}-3$}
      \uput[d](6,-0.1){$\color{red}1$}
    \end{pspicture*} &\tabularnewline
%    \addlinespace[2mm]
%    $1<x<7$ &&\tabularnewline
    \addlinespace[4mm]
	    && \centering{}$x\in]42;+\infty[$\tabularnewline
%    \addlinespace[4mm]
%	    & \psset{xunit=0.5cm,yunit=0.5cm,algebraic=true,dimen=middle,dotstyle=o,dotsize=5pt 0,linewidth=1.2pt,arrowsize=3pt 2,arrowinset=0.25}
%    \begin{pspicture*}(1.,-1)(8.,.5)
%      \psaxes[labelFontSize=\scriptstyle,xAxis=true,yAxis=true,Dx=10.,Dy=1.,ticksize=-2pt 0]{->}(0,0)(1.,-1.)(8.,1.)
%      \intervalleOmI(1,6)
%      \uput[d](6,-0.1){$\color{red}-2$}
%    \end{pspicture*} &\tabularnewline
    \bottomrule
  \end{tabular}
\end{center}

\pagebreak

% Exercice sur la valeur absolue
\exo[3 points] Traduire les égalités et inégalités suivantes à l'aide d'une distance, puis représenter l'ensemble des réels $x$ tels que :

\begin{multicols}{2}
  \begin{enumerate}
    \item $|x-5|=2$\rep{8}
    \item $|x-12|<1$\rep{8}
  \end{enumerate}
\end{multicols}

\bigskip

\exo[3 points] \vspace{-2mm} 
\begin{enumerate}
  \item Écrire $\sqrt{300}$ sous la forme $a\sqrt{b}$, où $a$ est un entier et $b$ l'entier naturel le plus petit possible.\rep{6}
  \item Écrire $\dfrac{5}{\sqrt{6}-1}$ sans racine carrée au dénominateur.\rep{6}
  \item Donner une valeur arrondie de $\dfrac{3}{\sqrt{7}}$ à $10^{-4}$ près.\rep{4}
\end{enumerate}

\bigskip

% Exercice de probabilités
\exo[3 points] Vaitiare lance un dé cubique équilibré un peu particulier : le dé comporte deux faces avec le numéro $1$, une face avec le $4$, deux faces avec le $5$ et une face avec le $6$. Vaitiare note le numéro obtenu.
\begin{enumerate}
  \item Modéliser cette expérience aléatoire.\rep{8}
  \item Teiki affirme qu'avec un dé cubique \og{}classique\fg{} (les faces sont numérotées de $1$ à $6$), non truqué, il a plus de chance d'obtenir un numéro supérieur ou égal à $5$. Que penser de son affirmation ?\rep{8}
\end{enumerate}

\pagebreak

% Exercice Python : fonction et if, Black Friday
\exo[4 points] Aujourd'hui, c'est Black Friday ! Henua S. réunit toutes ses économies et se rend dans son magasin de surf préféré. La remise effectuée dépend du prix de l'article acheté :
\begin{itemize}
  \item pour un article coûtant moins de \np{7500} XPF, une remise de $40\%$ est effectuée ;
  \item pour un article coûtant \np{7500} XPF ou plus, une remise de $30\%$ est effectuée.
\end{itemize}
\begin{enumerate}
  \item Compléter la fonction Python suivante afin qu'elle renvoie le montant de la remise effectuée pour un article dont le prix est donné en paramètre de la fonction :
    \begin{minted}{python}
def calculer_remise(P):
    if ...............:
        remise = ...........
    else:
        .................
    return ..........
    \end{minted}
  \item Quel appel permet de déterminer le montant de la remise pour un article coûtant $\np{5000}$ XPF ?\rep{4}
  \item Déterminer ce montant.\rep{6}
\end{enumerate}

\end{document}
