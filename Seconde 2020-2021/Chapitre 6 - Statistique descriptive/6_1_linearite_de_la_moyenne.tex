\documentclass[a4paper]{article}

\input ../header
\usepackage[np]{numprint}
\usepackage{xcolor}
\usepackage{booktabs}

\setlength{\multicolsep}{2pt}

\begin{document}

\title{Activités -- Découvrir certaines propriétés de la moyenne}

\pagestyle{empty}

\date{}
\author{}

\maketitle{}

Un magasin de jouets vend des trottinettes électriques dont les prix sont répartis de la manière suivante :

\begin{center}
  \begin{tabular}{@{}ccccc@{}}
    \toprule
    Prix (en XPF) & $\np{30000}$ & $\np{40000}$ & $\np{50000}$ & $\np{60000}$\\
    \midrule
    Quantité & $150$ & $250$ & $400$ & $200$\\
    \bottomrule
  \end{tabular}
\end{center}

\begin{enumerate}
  \item Calculer le prix moyen d'une trottinette électrique dans ce stock.
  \item Afin de liquider son stock, l'enseigne propose de multiplier tous les prix par $0,80$.
    \begin{enumerate}
      \item (Bonus) Quelle réduction est ainsi appliquée ?
      \item Compléter le tableau ci-dessous avec les prix soldés :

	\begin{center}
	  \begin{tabular}{@{}ccccc@{}}
	    \toprule
	    Prix (en XPF) & $\hdots\hdots\hdots$ & $\hdots\hdots\hdots$ & $\hdots\hdots\hdots$ & $\hdots\hdots\hdots$\\
	    \midrule
	    Quantité & $150$ & $250$ & $400$ & $200$\\
	    \bottomrule
	  \end{tabular}
	\end{center}
      \item Calculer le nouveau prix moyen.
      \item La moyenne a-t-elle aussi été multipliée par $0,80$ ?
    \end{enumerate}
  \item Quel sera le prix moyen d'une trottinette si le magasin propose une seconde remise de $\np{5000}$ XPF sur chaque trottinette ?
\end{enumerate}

\vspace{1cm}

\textbf{À retenir}

\begin{itemize}
  \item Si on multiplie toutes les valeurs d'une série statistique par un nombre $a$, alors la moyenne de cette série est \dotfill{}
  \item Si on ajoute un même nombre $b$ à toutes les valeurs d'une série statistique, \dotfill\\
    $\hdots$\dotfill
\end{itemize}


\end{document}
