\documentclass[handout]{beamer}

% Lignes réponses
\usepackage{pgffor} % pour la commande \foreach permettant de réaliser une boucle
\newcommand{\pointilles}{{\\\rule{0pt}{1pt}\dotfill\rule{0pt}{1pt}}}
\newcommand{\rep}[1]{\foreach \n in {1,...,#1} {\pointilles}}

% Commandes pour cacher/révéler du texte facilement à l'aide d'un booléen
\usepackage{xstring}
\usepackage{ifthen}

\newboolean{reveal}
\setboolean{reveal}{true}

\newlength{\stextwidth} % une nouvelle longueur

\newcommand\x{6}

\newcommand{\guess}[1]{\ifthenelse{\boolean{reveal}}{{\color{red}#1}}{\settowidth{\stextwidth}{#1}\makebox[\stextwidth]{\dotfill}}}

\newcommand{\guessmath}[1]{\ifthenelse{\boolean{reveal}}{\textcolor{red}{#1}}{\settowidth{\stextwidth}{$#1$}\makebox[1\stextwidth]{\dotfill}}}

\newcommand{\guessmathbin}[1]{\ifthenelse{\boolean{reveal}}{\mathbin{\color{red}#1}}{\settowidth{\stextwidth}{$#1$}\makebox[2\stextwidth]{\dotfill}}}

% ========================================================================%

\usetheme{focus}

\usepackage{pgfpages}
\pgfpagesuselayout{4 on 1}[a4paper,landscape]

\usepackage[french]{babel}

\usepackage{xcolor}

\usepackage{pstricks,pst-plot,pst-text,pst-tree,pst-eps,pst-fill,pst-node,pst-math}
\usepackage{pstricks-add,pst-xkey}

\input ../tabvar

\usepackage{multicol}
\usepackage[np]{numprint}

\usepackage{booktabs}

\newcommand{\vect}[1]{\overrightarrow{#1}}
\newcommand{\Oij}{\left(O;\vect{i},\vect{j}\right)}
\newcommand{\norm}[1]{\left|\left|#1\right|\right|}

\begin{document}

\title{}

\date{}

\begin{frame}
  \frametitle{1. Moyenne d'une série statistique}
  \textbf{Définition. --}   La \guess{moyenne} pondérée de la série statistique 

  \begin{center}
    \begin{tabular}{@{}ccccc@{}}
      \toprule
      \text{Valeur} & $x_1$ & $x_2$ & $\hdots$ & $x_p$\\
      \midrule
      \text{Effectif} & $n_1$ & $n_2$ & $\hdots$ & $n_p$\\
      \bottomrule
    \end{tabular}
  \end{center}

  est le réel $\bar{x}$ tel que:\pause{} \[\bar{x}=\guessmath{\dfrac{n_1\times x_1 + n_2\times x_2 + \hdots + n_p\times x_p}{N}},\]\pause{}
  où $N$ est \guess{l'effectif total}. 
\end{frame}

\begin{frame}
  \textit{Exemple. -- Le tableau ci-dessous donne la répartition des magasins d'une enseigne de prêt-à-porter en fonction de leur nombre d'employés. 
    \begin{center}
      \begin{tabular}{@{}cccccccc@{}}
	\toprule
	Nombre d'employés & $1$ & $2$ & $3$ & $4$ & $5$ & $6$ & $7$\\
	\midrule
	Effectif & $2$ & $10$ & $48$ & $90$ & $54$ & $14$ & $4$\\
	\bottomrule
      \end{tabular}
    \end{center}
    \begin{enumerate}
      \item Interpréter la valeur $90$ présente dans le tableau.\rep{2}
      \item Déterminer la moyenne de cette série statistique (on donnera la valeur exacte puis une valeur arrondie à $0,01$ près).\rep{2}
    \end{enumerate}
  }
\end{frame}

\begin{frame}
  \textbf{Propriété. --} On peut calculer la moyenne $\bar{x}$ à partir de la distribution des fréquences :

  \begin{center}
    \begin{tabular}{@{}ccccc@{}}
      \toprule
      \text{Valeur} & $x_1$ & $x_2$ & $\hdots$ & $x_p$\\
      \midrule
      \text{Effectif} & $f_1$ & $f_2$ & $\hdots$ & $f_p$\\
      \bottomrule
    \end{tabular}
  \end{center}

  \[\bar{x}=\guessmath{n_1\times f_1 + n_2\times f_2 + \hdots + n_p\times f_p}.\]
\end{frame}

\begin{frame}
  \textit{Exemple. -- On a soumis une liste de $10$ questions à un groupe de candidats à un jeu télévisé. Voici les résultats :
    \begin{center}
      \begin{tabular}{@{}cccccccc@{}}
	\toprule
	Réponses justes & $4$ & $5$ & $6$ & $7$ & $8$ & $9$ & $10$\\
	\midrule
	Fréquence & $0,05$ & $0$ & $0,175$ & $0,35$ & $0,275$ & $0,125$ & $0,025$\\
	\bottomrule
      \end{tabular}
    \end{center}
    \begin{enumerate}
      \item Interpréter la valeur $0,175$ présente dans le tableau précédent.\rep{2}
      \item Déterminer le nombre moyen de bonnes réponses données.\rep{2}
    \end{enumerate}
  }
\end{frame}

\begin{frame}
  \textit{Remarque. -- La moyenne permet de résumer une série statistique à l'aide d'un seul nombre mais elle ne donne aucune information sur la \guess{répartition} des valeurs.}

  \bigskip

  \textbf{Proposition (linéarité de la moyenne). --} 
  \begin{itemize}
    \item Si on multiplie toutes les valeurs d'une série statistique par un nombre $a$, alors la moyenne de cette série \guess{est multipliée par $a$}.
    \item Si on ajoute un même nombre $b$ à toutes les valeurs d'une série statistique, \guess{alors la moyenne de cette série augmente de $b$}.
  \end{itemize}
\end{frame}

\begin{frame}
  \textit{Exemple. -- Dans la classe de Seconde 3, la moyenne au dernier devoir commun de mathématiques est catastrophique : 6,25/20 ! Dans la classe de Seconde 1, les élèves sont bien meilleurs et la moyenne de cette classe est 12,25/20.\\
  Le professeur de la Seconde 3 décide alors de multiplier toutes les notes par 2, tandis que le professeur de la Seconde 1 décide d'ajouter 1 point à chaque élève. Après modification des notes, quelle classe aura la meilleure moyenne ?\rep{4}}
\end{frame}

\begin{frame}
  \frametitle{2. Variance et écart-type}
  \textbf{Définition. --} Avec les notations du paragraphe 1., la variance $V$ de la série statistique est calculée grâce à :
  \[V=\guessmath{\dfrac{n_1\times x_1^2 + n_2\times x_2^2 + \hdots + n_p\times x_p^2}{N} - \bar{x}^2}.\]
  L'écart-type de la série statistique est le nombre $\sigma$ défini par :
  \[\sigma=\guessmath{\sqrt{V}}.\]
\end{frame}

\begin{frame}
  \textit{Exemple. -- On a relevé l'âge des participants à une compétition inter académique de judo. On a obtenu les résultats ci-dessous :
    \begin{center}
      \begin{tabular}{@{}ccccc@{}}
	\toprule
	Âge & $15$ & $16$ & $17$ & $18$\\
	\midrule
	Nombre de judokas & $24$ & $29$ & $35$ & $22$\\
	\bottomrule
      \end{tabular}
    \end{center}
    Déterminer l'écart-type de cette série à $0,01$ près.\rep{8}
  }
\end{frame}

\end{document}

%%% Local Variables:
%%% mode: latex
%%% TeX-master: t
%%% End:
