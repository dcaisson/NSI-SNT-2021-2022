\documentclass[handout]{beamer}

% Lignes réponses
\usepackage{pgffor} % pour la commande \foreach permettant de réaliser une boucle
\newcommand{\pointilles}{{\\\rule{0pt}{1pt}\dotfill\rule{0pt}{1pt}}}
\newcommand{\rep}[1]{\foreach \n in {1,...,#1} {\pointilles}}

% Commandes pour cacher/révéler du texte facilement à l'aide d'un booléen
\usepackage{xstring}
\usepackage{ifthen}

\newboolean{reveal}
\setboolean{reveal}{false}

\newlength{\stextwidth} % une nouvelle longueur

\newcommand\x{6}

\newcommand{\guess}[1]{\ifthenelse{\boolean{reveal}}{{\color{red}#1}}{\settowidth{\stextwidth}{#1}\makebox[\stextwidth]{\dotfill}}}

\newcommand{\guessmath}[1]{\ifthenelse{\boolean{reveal}}{\textcolor{red}{#1}}{\settowidth{\stextwidth}{$#1$}\makebox[1\stextwidth]{\dotfill}}}

\newcommand{\guessmathbin}[1]{\ifthenelse{\boolean{reveal}}{\mathbin{\color{red}#1}}{\settowidth{\stextwidth}{$#1$}\makebox[2\stextwidth]{\dotfill}}}

% ========================================================================%

\usetheme{focus}

\usepackage{pgfpages}
\pgfpagesuselayout{4 on 1}[a4paper,landscape]

\usepackage[french]{babel}

\usepackage{xcolor}

\usepackage{pstricks,pst-plot,pst-text,pst-tree,pst-eps,pst-fill,pst-node,pst-math}
\usepackage{pstricks-add,pst-xkey}

\input ../tabvar

\usepackage{multicol}
\usepackage[np]{numprint}

\usepackage{booktabs}

\newcommand{\vect}[1]{\overrightarrow{#1}}
\newcommand{\Oij}{\left(O;\vect{i},\vect{j}\right)}
\newcommand{\norm}[1]{\left|\left|#1\right|\right|}

\begin{document}

\title{}

\date{}

\begin{frame}
  \frametitle{1. Médiane, quartiles et écart interquartile}

  \textbf{Définition. --} La médiane $Me$ d'une série statistique de $n$ valeurs ordonnées est :
  \begin{itemize}
    \item la valeur centrale si $n$ est impair ;
    \item la demi-somme des deux valeurs situées \og{}au milieu\fg{} si $n$ est pair.
  \end{itemize}

  \medskip

  \textit{Exemple. -- Monsieur C., le professeur préféré de Henua, demande à quelques élèves de Seconde 1 combien de téléphones ils ont eu dans leur vie. Voici les résultats obtenus :
    \begin{center}
      $3$, $2$, $1$, $0$, $4$, $2$, $3$, $2$, $2$
    \end{center}
  Quelle est la médiane de cette série statistique ?}
\end{frame}

\begin{frame}
  \textit{Exemple. -- Calculer la médiane de la série statistique suivante :
  \begin{center}
    $5$, $1$, $0$, $1$, $3$, $10$, $12$, $11$, $4$, $2$
  \end{center}
Calculer la médiane de la série statistique obtenue.}
\end{frame}

\begin{frame}
  \textit{Exemple. -- Josh compte le nombre de \og{}like\fg{} obtenus chaque jour pendant un mois sur son compte Instagram. Il obtient les résultats suivants :
    \begin{center}
      \renewcommand{\arraystretch}{1.2}
      \begin{tabular}{|>{\centering}m{3.5cm}|*{6}{>{\centering}m{7mm}|}}
	\hline
	Nombre de \og{}like\fg{} & $0$ & $1$ & $2$ & $3$ & $4$ & $5$\tabularnewline
	\hline
	Nombre de jours & $3$ & $10$ & $7$ & $7$ & $2$ & $2$\tabularnewline
	\hline
      \end{tabular} 
    \end{center}
    Moe fait de même :
    \begin{center}
      \renewcommand{\arraystretch}{1.2}
      \begin{tabular}{|>{\centering}m{3.5cm}|*{6}{>{\centering}m{7mm}|}}
	\hline
	Nombre de \og{}like\fg{} & $0$ & $1$ & $2$ & $3$ & $4$ & $5$\tabularnewline
	\hline
	Nombre de jours & $5$ & $4$ & $6$ & $10$ & $2$ & $3$\tabularnewline
	\hline
      \end{tabular} 
    \end{center}
    Déterminer la moyenne puis la médiane de chacune des séries précédentes.
  }
\end{frame}

\begin{frame}
  \textbf{Définition. --} Le premier quartile, noté $Q_1$, est la plus petite valeur de la série telle \dotfill\\\dotfill

  \smallskip

  Le troisième quartile, noté $Q_3$, est la plus petite valeur de la série telle
  \dotfill\\\dotfill

  \bigskip
  
  \textit{Exemple. -- Déterminer le premier et le troisième quartile des deux séries précédentes.}
\end{frame}

\begin{frame}
  \textbf{Définition. --} L'écart interquartile est la différence $\hdots\hdots\hdots\hdots$. L'intervalle interquartile est l'intervalle $\hdots\hdots\hdots\hdots$

  \bigskip

  \textit{Exemple. -- Déterminer l'écart interquartile puis l'intervalle interquartile des deux séries de l'exemple précédent.}
\end{frame}

\end{document}

%%% Local Variables:
%%% mode: latex
%%% TeX-master: t
%%% End:
