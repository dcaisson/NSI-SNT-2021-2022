\documentclass[handout]{beamer}

% Lignes réponses
\usepackage{pgffor} % pour la commande \foreach permettant de réaliser une boucle
\newcommand{\pointilles}{{\\\rule{0pt}{1pt}\dotfill\rule{0pt}{1pt}}}
\newcommand{\rep}[1]{\foreach \n in {1,...,#1} {\pointilles}}

% Commandes pour cacher/révéler du texte facilement à l'aide d'un booléen
\usepackage{xstring}
\usepackage{ifthen}

\newboolean{reveal}
\setboolean{reveal}{true}

\newlength{\stextwidth} % une nouvelle longueur

\newcommand\x{6}

\newcommand{\guess}[1]{\ifthenelse{\boolean{reveal}}{{\color{red}#1}}{\settowidth{\stextwidth}{#1}\makebox[\stextwidth]{\dotfill}}}

\newcommand{\guessmath}[1]{\ifthenelse{\boolean{reveal}}{\textcolor{red}{#1}}{\settowidth{\stextwidth}{$#1$}\makebox[1.3\stextwidth]{\dotfill}}}

\newcommand{\guessmathbin}[1]{\ifthenelse{\boolean{reveal}}{\mathbin{\color{red}#1}}{\settowidth{\stextwidth}{$#1$}\makebox[2\stextwidth]{\dotfill}}}

% ========================================================================%

\usetheme{focus}

\usepackage{pgfpages}
\pgfpagesuselayout{4 on 1}[a4paper,landscape]

\usepackage[french]{babel}

\usepackage{xcolor}

\usepackage{pstricks,pst-plot,pst-text,pst-tree,pst-eps,pst-fill,pst-node,pst-math}
\usepackage{pstricks-add,pst-xkey}

\input ../tabvar

\usepackage{multicol}
\usepackage[np]{numprint}

\usepackage{booktabs}

\newcommand{\vect}[1]{\overrightarrow{#1}}
\newcommand{\Oij}{\left(O;\vect{i},\vect{j}\right)}
\newcommand{\norm}[1]{\left|\left|#1\right|\right|}

\begin{document}

\title{}

\date{}
\begin{frame}
  \frametitle{Résoudre graphiquement une équation}
  \textbf{Définition. --} Soient $f$ une fonction définie sur un ensemble $\mathcal{D}$ et $k$ un réel. Résoudre graphiquement l'équation $f(x)=k$ consiste à déterminer les réels $x$ (appartenant à $\mathcal{D}$) qui ont pour \guess{image} $k$ par $f$.\\
  Il s'agit aussi de déterminer les \guess{antécédents éventuels} de \guess{$k$} par \guess{$f$}.

  \bigskip

  \textit{Exemple. -- On a représenté ci-contre une fonction $h$ définie sur $[-3;3]$.
  Pour résoudre l'équation $h(x)=3$, on a tracé la droite rouge.} 
\end{frame}

\begin{frame}
  \begin{center}
    \newrgbcolor{ffvvqq}{1. 0.3333333333333333 0.}
    \newrgbcolor{ccqqqq}{0.8 0. 0.}
    \psset{xunit=1.0cm,yunit=0.7cm,algebraic=true,dimen=middle,dotstyle=o,dotsize=5pt 0,linewidth=1.pt,arrowsize=3pt 2,arrowinset=0.25}
    \begin{pspicture*}(-4.,-2.)(4.,6.)
      \multips(0,-2)(0,1.0){9}{\psline[linestyle=dashed,linecap=1,dash=1.5pt 1.5pt,linewidth=0.4pt,linecolor=lightgray]{c-c}(-4.,0)(4.,0)}
      \multips(-4,0)(1.0,0){9}{\psline[linestyle=dashed,linecap=1,dash=1.5pt 1.5pt,linewidth=0.4pt,linecolor=lightgray]{c-c}(0,-2.)(0,6.)}
      \psaxes[labelFontSize=\scriptstyle,xAxis=true,yAxis=true,Dx=1.,Dy=1.,ticksize=-2pt 0]{->}(0,0)(-4.,-2.)(4.,6.)
      \psplot[linewidth=1.pt,linecolor=ffvvqq,plotpoints=200]{-4.0}{4.0}{-0.5*x^(3.0)+3.0*x+2.0}
      \psplot[linewidth=1.pt,linecolor=ccqqqq]{-4.}{4.}{(--3.-0.*x)/1.}
      \psline[linewidth=1.pt,linestyle=dotted](-2.6016791318831545,3.)(-2.6016791318831545,0.)
      \psline[linewidth=1.pt,linestyle=dotted](0.3398768866231824,3.)(0.3398768866231824,0.)
      \psline[linewidth=1.pt,linestyle=dotted](2.2618022452599713,3.)(2.2618022452599713,0.)
      \begin{scriptsize}
	\psdots[dotsize=2pt 0,dotstyle=*,linecolor=darkgray](-2.6016791318831545,3.)
	\psdots[dotsize=2pt 0,dotstyle=*,linecolor=darkgray](0.3398768866231824,3.)
	\psdots[dotsize=2pt 0,dotstyle=*,linecolor=darkgray](2.2618022452599713,3.)
      \end{scriptsize}
    \end{pspicture*}
  \end{center}
  \textit{On constate alors que l'équation $h(x)=3$ a \guess{trois} solutions dont des valeurs approchées sont \guess{$-2,6$}, \guess{$0,3$} et \guess{$2,3$}.}
\end{frame}

\begin{frame}
  \textit{Cas particulier des équations du type $f(x)=g(x)$ -- Exemple\\
    Ci-dessous sont représentées deux fonctions $f$ et $g$, toutes deux définies sur l'intervalle $[-3;3]$.
    \begin{center}
      \newrgbcolor{ccqqqq}{0.8 0. 0.}
      \psset{xunit=1.0cm,yunit=0.3cm,algebraic=true,dimen=middle,dotstyle=o,dotsize=5pt 0,linewidth=1.6pt,arrowsize=3pt 2,arrowinset=0.25}
      \begin{pspicture*}(-4.,-6.)(5.,10.)
	\multips(0,-6)(0,2.0){9}{\psline[linestyle=dashed,linecap=1,dash=1.5pt 1.5pt,linewidth=0.4pt,linecolor=lightgray]{c-c}(-4.,0)(5.,0)}
	\multips(-4,0)(1.0,0){10}{\psline[linestyle=dashed,linecap=1,dash=1.5pt 1.5pt,linewidth=0.4pt,linecolor=lightgray]{c-c}(0,-6.)(0,10.)}
	\psaxes[labelFontSize=\scriptstyle,xAxis=true,yAxis=true,Dx=1.,Dy=2.,ticksize=-2pt 0]{->}(0,0)(-4.,-6.)(5.,10.)
	\psplot[linewidth=1.pt,linecolor=ccqqqq,plotpoints=200]{-3}{4}{x^(2.0)-3.0*x+2.0}
	\psplot[linewidth=1.pt,linecolor=blue,plotpoints=200]{-3}{4}{-x^(2.0)+x+8.0}
	\begin{scriptsize}
	  \psdots[dotsize=2pt 0,dotstyle=*,linecolor=darkgray](-1,6)
	  \psdots[dotsize=2pt 0,dotstyle=*,linecolor=darkgray](3,2)
	\end{scriptsize}
	\psline[linewidth=1.pt,linestyle=dotted](-1,6)(-1,0)
	\psline[linewidth=1.pt,linestyle=dotted](3,2)(3,0)
	\uput[u](4,6){\color{ccqqqq}$\mathcal{C}_f$}
	\uput[r](4,-4){\color{blue}$\mathcal{C}_g$}
      \end{pspicture*} 
    \end{center}
  }
  L'équation $f(x)=g(x)$ a \guess{deux} solutions dont des valeurs approchées sont \guess{$-1$} et \guess{$3$}.
\end{frame}

\begin{frame}
  \textit{Remarque. -- Lorsqu'on résout graphiquement une équation, on obtient une valeur \guess{approchée} de chaque solution éventuelle.}
\end{frame}

\end{document}

%%% Local Variables:
%%% mode: latex
%%% TeX-master: t
%%% End:
