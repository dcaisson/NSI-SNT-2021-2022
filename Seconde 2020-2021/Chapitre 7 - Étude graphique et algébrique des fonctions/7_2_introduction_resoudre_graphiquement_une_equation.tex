\documentclass[a4paper]{article}

\input ../header
\usepackage[np]{numprint}
\usepackage{xcolor}
\usepackage{booktabs}
\usepackage{gensymb}

\setlength{\multicolsep}{2pt}

\begin{document}

\title{Activité d'introduction -- Résoudre graphiquement une équation}

\pagestyle{empty}

\date{}
\author{}

\maketitle{}
\thispagestyle{empty}

Dans un club de modélisme, des amis essaient un petit avion télécommandé. Ils le posent sur un support situé à un mètre du sol et le font décoller. Au bout de quelques secondes, l'avion heurte un mur puis tombe au sol.

\medskip

La courbe ci-dessous donne la hauteur $h(t)$ de l'avion, exprimée en m, en fonction du temps $t$, exprimé en s, jusqu'à ce que celui-ci heurte le mur. L'instant $t=0$ correspond au moment où l'avion s'envole.

\bigskip

\begin{center}
  \newrgbcolor{qqwuqq}{0. 0.39215686274509803 0.}
  \psset{xunit=1.0cm,yunit=1.0cm,algebraic=true,dimen=middle,dotstyle=o,dotsize=5pt 0,linewidth=1.pt,arrowsize=3pt 2,arrowinset=0.25}
  \begin{pspicture*}(-1.,-1.)(12.,7.)
    \multips(0,-1)(0,1.0){9}{\psline[linestyle=dashed,linecap=1,dash=1.5pt 1.5pt,linewidth=0.4pt,linecolor=lightgray]{c-c}(-1.,0)(12.,0)}
    \multips(-1,0)(1.0,0){14}{\psline[linestyle=dashed,linecap=1,dash=1.5pt 1.5pt,linewidth=0.4pt,linecolor=lightgray]{c-c}(0,-1.)(0,7.)}
    \psaxes[labelFontSize=\scriptstyle,xAxis=true,yAxis=true,Dx=1.,Dy=1.,ticksize=-2pt 0]{->}(0,0)(-1.,-1.)(12.,7.)
    \psplot[linewidth=1.pt,linecolor=qqwuqq,plotpoints=200]{0}{9}{-1.0/5.0*x^(2.0)+2.0*x+1.0}
    \uput[r](0,6.5){Hauteur $h(t)$ (en m)}
    \uput[u](10.5,0){Temps $t$ (en s)}
  \end{pspicture*}
\end{center}

\bigskip

\begin{enumerate}
  \item Lire graphiquement (on donnera des valeurs approchées à $0,2$ près) :
    \begin{enumerate}
      \item la hauteur de l'avion au bout de $2$ s;
      \item la hauteur de l'avion quand il heurte le mur ;
      \item le (ou les) instant(s) où l'avion a une hauteur égale à $5,80$ m.
    \end{enumerate}
  \item Reformuler chacune des réponses précédentes en utilisant les mots \og{}image\fg{}, \og{}antécédent\fg{}, \og{}équation\fg{}.
  \item La hauteur $h(t)$ (en m) de l'avion en fonction du temps (en s) est donnée par : 
    \[h(t)=-\dfrac{1}{5}t^2+2t+1.\]
    \begin{enumerate}
      \item Déterminer $h(9)$ par le calcul.
      \item Déterminer $h(4)$ et $h(6)$.
      \item Quels résultats obtenus précédemment sont confirmés ici ?
    \end{enumerate}
\end{enumerate}

\end{document}
