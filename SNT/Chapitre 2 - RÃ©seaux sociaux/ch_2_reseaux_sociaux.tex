\documentclass[a4paper, dvipsnames]{article}

\input ../header
\usepackage{lscape}

\title{Chapitre 2 -- Les réseaux sociaux}

\author{}
\date{}

\begin{document}

\renewcommand{\contentsname}{}

\pagestyle{fancy}

\begin{tcolorbox}[colframe=blue!75, colback=blue!45, valign=center, height=1.5cm, top=5mm]
  \maketitle
\end{tcolorbox}

\tableofcontents

\vspace{1cm}

\thispagestyle{fancy}

\section{Une brève présentation des réseaux sociaux}

Les premiers réseaux sociaux sur Internet sont apparus d'abord sous la forme de clubs fermés. Leur véritable essor sous la forme que nous connaissons, étendue à la planète et sans limites, ne date que de la fin des années \np{2000}. En dix ans, les réseaux sociaux ont pourtant transformé nos vies et ils ont l'ambition d'aller beaucoup plus loin.

\bigskip

L'objectif de cette activité est de créer une frise chronologique replaçant la création des principaux réseaux sociaux dans un contexte historique.
\bigskip

\begin{activite}{}{}
  \begin{enumerate}
    \item Compléter la deuxième colonne du tableau donné en annexe.
      %    \begin{center}
      %      \renewcommand{\arraystretch}{1.2}
      %      \begin{tabular}{|>{\centering}m{2cm}|>{\centering}m{1.2cm}|>{\centering}m{3cm}|>{\centering}m{3.5cm}|>{\centering}m{5cm}|}
      %	\hline
      %	\textbf{Réseau social} & \textbf{Année} & \textbf{Utilisateurs} & \textbf{Principe} & \textbf{Fonctionnalités}\tabularnewline
      %	\hline
      %	Facebook &&&& \tabularnewline
      %	\hline
      %	YouTube &&&&\tabularnewline
      %	\hline
      %	Instagram &&&&\tabularnewline
      %	\hline
      %	WhatsApp &&&&\tabularnewline
      %	\hline
      %	Snapchat &&&&\tabularnewline
      %	\hline
      %	Twitter &&&&\tabularnewline
      %	\hline
      %      \end{tabular}
      %    \end{center}
    \item Récupérer la frise chronologique à compléter sur Moodle. Il s'agit du fichier \verb|ch_2_frise_a_completer.bin|.
    \item Se rendre sur le site \href{http://www.frisechronos.fr/DojoMain.htm}{http://www.frisechronos.fr/DojoMain.htm} et ouvrir le fichier précédent.
    \item Choisir de présenter la frise sur $2$ pages horizontales.
    \item Compléter la frise en y ajoutant la création de chacun des réseaux sociaux de la question 1. Utiliser une image pour chaque réseau social.
    \item Générer un fichier \verb|pdf| puis le renommer sous la forme \verb|Nom_Prenom_Classe.pdf| avant de le déposer sur Moodle.
    \item Générer un fichier \verb|png| puis le renommer sous la forme \verb|Nom_Prenom_Classe.png| avant de le déposer sur Moodle.
  \end{enumerate} 
\end{activite}


%\begin{enumerate}
%  \item Compléter la deuxième colonne du tableau donné en annexe.
%%    \begin{center}
%%      \renewcommand{\arraystretch}{1.2}
%%      \begin{tabular}{|>{\centering}m{2cm}|>{\centering}m{1.2cm}|>{\centering}m{3cm}|>{\centering}m{3.5cm}|>{\centering}m{5cm}|}
%%	\hline
%%	\textbf{Réseau social} & \textbf{Année} & \textbf{Utilisateurs} & \textbf{Principe} & \textbf{Fonctionnalités}\tabularnewline
%%	\hline
%%	Facebook &&&& \tabularnewline
%%	\hline
%%	YouTube &&&&\tabularnewline
%%	\hline
%%	Instagram &&&&\tabularnewline
%%	\hline
%%	WhatsApp &&&&\tabularnewline
%%	\hline
%%	Snapchat &&&&\tabularnewline
%%	\hline
%%	Twitter &&&&\tabularnewline
%%	\hline
%%      \end{tabular}
%%    \end{center}
%  \item Récupérer la frise chronologique à compléter sur Moodle. Il s'agit du fichier \verb|ch_2_frise_a_completer.bin|.
%  \item Se rendre sur le site \href{http://www.frisechronos.fr/DojoMain.htm}{http://www.frisechronos.fr/DojoMain.htm} et ouvrir le fichier précédent.
%  \item Choisir de présenter la frise sur $2$ pages horizontales.
%  \item Compléter la frise en y ajoutant la création de chacun des réseaux sociaux de la question 1. Utiliser une image pour chaque réseau social.
%  \item Générer un fichier \verb|pdf| puis le renommer sous la forme \verb|Nom_Prenom_Classe.pdf| avant de le déposer sur Moodle.
%  \item Générer un fichier \verb|png| puis le renommer sous la forme \verb|Nom_Prenom_Classe.png| avant de le déposer sur Moodle.
%\end{enumerate}

\bigskip

\section{Quelques chiffres}

\begin{activite}[breakable]{Tableau à compléter}{}
  \begin{enumerate}
    \item Compléter la colonne \textbf{Utilisateurs} du tableau.
    \item Compléter la colonne \textbf{Principe} à l'aide des propositions suivantes :
      \begin{multicols}{2}
	\begin{itemize}
	  \item Partager des photos et des vidéos de manière éphémère
	  \item Poster et regarder des vidéos
	  \item Rester en contact et échanger avec son entourage\columnbreak
	  \item Dialoguer et échanger des photos et des vidéos instantanément
	  \item Envoyer des messages courts pour commenter l'actualité ou échanger avec des internautes
	  \item Partager des photos et des vidéos
	\end{itemize}
      \end{multicols}
    \item Compléter la colonne \textbf{Fonctionnalités} à l'aide des propositions suivantes :
      \begin{multicols}{2}
	\begin{itemize}
	  \item Créer des stories : ensemble de photos/vidéos de $10$ secondes maximum, prises en direct, disponibles pendant $24$ heures et consultables par ses contacts
	  \item Regarder des vidéos
	  \item Suivre les tweets de ses abonnements
	  \item Suivre et interagir avec les autres utilisateurs
	  \item Créer des pages ou des groupes
	  \item Créer des stories (photos, vidéos) \textit{à utiliser deux fois}
	  \item Suivre la chaîne d'autres personnes
	  \item Discuter en ligne (Messenger)
	  \item Partager des statuts, des tweets (messages courts de $280$ caractères), avec ses abonnés (followers)
	  \item Créer une chaîne
	  \item Publier des contenus sur son compte
	  \item Créer des groupes de discussion \textit{à utiliser deux fois}
	  \item Partager des photos et des vidéos avec ses contacts mais via des groupes
	  \item Publier des contenus avec des amis (photos, vidéos, textes etc)
	\end{itemize} 
      \end{multicols}
  \end{enumerate} 
\end{activite}

\section{Réseaux sociaux et graphes}


\begin{activite}{Amitiés}{}
  On considère un groupe d'amis :
  \begin{itemize}
    \item Léa est amie avec Mathias, Pearl, Emie et Léo.
    \item Mathias est ami avec Léa, Maxence, Emie, Léo et Emma.
    \item Emma est amie avec Mathias, Quentin et Léo.
    \item Léo est ami avec Maxence, Léa, Mathias et Emma.
    \item Emie est amie avec Léa et Mathias.
    \item Pearl est amie avec Léa.
    \item Maxence est ami avec Léo et Mathias.
    \item Quentin est ami avec Emma.
  \end{itemize}
  Proposer deux moyens de représenter ces amitiés.
\end{activite}

\bigskip

\begin{activite}[breakable]{Réseaux sociaux et graphes}{}
  \begin{itemize}
    \item Un graphe non orienté est défini par ses sommets (ou noeuds) et des arêtes reliant entre eux des sommets.
    \item Une chaîne est une suite de noeuds reliés par des arêtes dans un graphe non orienté.
    \item La longueur d'une chaîne est le nombre d'arêtes de la chaîne (un de moins que le nombre de sommets).
    \item La distance entre $2$ sommets est égale à la longueur de la plus petite chaîne qui les relie.
    \item L'écartement (ou l'excentricité) d'un sommet est la distance maximale entre ce sommet et les autres sommets du graphe.
    \item Le diamètre d'un graphe non orienté est la distance maximale entre deux sommets de ce graphe.
    \item Le centre d'un graphe est l'ensemble des sommets d'écartement minimal.
    \item Le rayon d'un graphe est l'écartement d'un des sommets du centre du graphe.
  \end{itemize} 

  \begin{enumerate}
    \pagebreak
    \item Remplir le tableau suivant avec la distance entre chacun des sommets du graphe de l'activité 3 :
      \vspace*{-2mm}
      \begin{center}
	\renewcommand{\arraystretch}{1.2}
	\begin{tabular}{|*{9}{>{\centering}m{1.1cm}|}}
	  \hline
	& Léa & Mathias & Emma & Léo & Emie & Pearl & Maxence & Quentin\tabularnewline
	\hline
	  Léa &&&& &&&&\tabularnewline
	  \hline
	  Mathias &&&& &&&&\tabularnewline
	  \hline
	  Emma &&&& &&&&\tabularnewline
	  \hline
	  Léo &&&& &&&&\tabularnewline
	  \hline
	  Emie &&&& &&&&\tabularnewline
	  \hline
	  Pearl &&&& &&&&\tabularnewline
	  \hline
	  Maxence &&&& &&&&\tabularnewline
	  \hline
	  Quentin &&&& &&&&\tabularnewline
	  \hline
	\end{tabular}
      \end{center}
    \item Remplir le tableau suivant :
      \vspace*{-4mm}
      \begin{center}
	\renewcommand{\arraystretch}{1.2}
	\begin{tabular}{|>{\centering}m{1.6cm}|*{8}{>{\centering}m{1.1cm}|}}
	  \hline
	  Sommet & Léa & Mathias & Emma & Léo & Emie & Pearl & Maxence & Quentin\tabularnewline
	  \hline
	  Écartement &&&& &&&&\tabularnewline
	  \hline
	\end{tabular}
      \end{center}
    \item Quel est le diamètre du graphe ?
    \item Déterminer le centre du graphe.
    \item Quel est le rayon du graphe ?
  \end{enumerate}
\end{activite}

\bigskip

\begin{activite}[breakable]{Limiter les risques pour sa vie privée}{}
  Sur un réseau social, il est souvent difficile de distinguer ce qui est public de ce qui reste privé.

  \bigskip

  \fbox{
    \begin{minipage}{0.92\linewidth}
      Voici quelques conseils pour éviter les mauvaises surprises:
      \begin{itemize}
	\item[$\bullet$] Soyez méfiants et gardez certaines informations absolument confidentielles : ne dévoilez pas trop de votre privée sur internet, ne partagez pas vos informations avec n'importe qui.

	  \smallskip
	\item[$\bullet$] Prenez le temps de régler les paramètres de confidentialité de votre profil : empêchez les moteurs de recherche d'indexer votre profil ou limitez sa visibilité complète à vos contacts. 

	\item[$\bullet$] Classez vos amis en plusieurs catégories (amis proches, connaissances, famille, etc.) : vous pourrez ainsi plus facilement déterminer ce que vous voulez montrer à tel ou tel groupe.
	  \smallskip
	\item[$\bullet$] Sécurisez vos profils : créer des mots de passe complexes (au moins 8 caractères mélangeant majuscules, minuscules, chiffres, caractères spéciaux).
	  \smallskip
	\item[$\bullet$] Masquez si possible votre identifiant de connexion des informations générales du profil (votre adresse électronique par exemple). 
	  \smallskip
	\item[$\bullet$] N'utilisez pas le même mot de passe que celui de votre messagerie électronique ; effacez régulièrement vos données de navigation et vos cookies sur votre ordinateur.
	  \smallskip
	\item[$\bullet$] Maîtrisez votre présence sur Internet : ne restez pas connecté quand vous quittez un site, pensez à cliquer sur "Déconnexion". Supprimer les profils que vous n'utilisez plus depuis plusieurs années.
      \end{itemize}
    \end{minipage}
  }

  \smallskip

  \hfill {\small Source : \href{https://www.cnil.fr/fr/cnil-direct/question/reseaux-sociaux-comment-limiter-les-risques-pour-ma-vie-privee}{\color{blue}https://www.cnil.fr}}\hspace{0.5cm}

  \bigskip

  \begin{enumerate}
    \item Que se passe-t-il lorsque notre profil est indexé par les moteurs de recherches ? \rep{2}
    \item Quel est l'intérêt de classer ses amis en plusieurs catégories ? Quels risques court-on si on ne le fait pas ?\rep{2}
    \item Pourquoi faut-il créer des mots de passe complexes et différents pour chaque service en-ligne ? Quels risques court-on si on ne le fait pas ?\rep{2}
    \item Pourquoi faut-il systématiquement se déconnecter des services en-ligne en fin de navigation ? Quels risques court-on si on ne le fait pas ?\rep{2}
    \item À la maison, connectez-vous sur chacun des réseaux sociaux que vous utilisez et appliquer ces recommandations. 
  \end{enumerate}
\end{activite}

\bigskip

\begin{activite}{Le harcèlement en ligne}{}
  Répondre aux questions suivantes après avoir regardé la vidéo disponible sur Moodle.

  \begin{enumerate}
    \item Qu'est-ce que le cyber-harcèlement ?\rep{2}
    \item Quelles peuvent être les conséquences du cyber-harcèlement ?\rep{2}
    \item Comment stopper ces violences ?\rep{2}
    \item Que risque l'auteur de cyber-harcèlement ?\rep{2}
  \end{enumerate}
\end{activite}

\begin{landscape}
  \renewcommand{\arraystretch}{1.3}
  \vspace*{1cm}
  \hspace*{1cm}\begin{tabular}{|>{\centering}m{4cm}|>{\centering}m{1.2cm}|>{\centering}m{3cm}|>{\centering}m{5cm}|>{\centering}m{10cm}|}
    \hline
    \textbf{Réseau social} & \textbf{Année} & \textbf{Utilisateurs} & \textbf{Principe} & \textbf{Fonctionnalités}\tabularnewline
    \hline
			   &&&&\tabularnewline
    Facebook\bigskip &&&& \begin{itemize}
      \item[] 
      \item[]
      \item[]
      \item[]
    \end{itemize}\tabularnewline
    \hline
		     &&&&\tabularnewline
      YouTube\bigskip &&&&\begin{itemize}
	\item[]
	\item[]
	\item[]
      \end{itemize}\tabularnewline
      \hline
		      &&&&\tabularnewline
	Instagram\bigskip &&&&\begin{itemize}
	  \item[]
	  \item[]
	  \item[]
	\end{itemize}\tabularnewline
	\hline
			  &&&&\tabularnewline
	  WhatsApp\bigskip &&&&\begin{itemize}
	    \item[]
	    \item[]
	  \end{itemize}\tabularnewline
	  \hline
			   &&&&\tabularnewline
	    Snapchat\bigskip &&&&\begin{itemize}
	      \item[]
	    \end{itemize}\tabularnewline
	    \hline
			     &&&&\tabularnewline
	      Twitter\bigskip &&&&\begin{itemize}
		\item[]
		\item[]
	      \end{itemize}\tabularnewline
	      \hline
	      \end{tabular}
	    \end{landscape}
\end{document}
