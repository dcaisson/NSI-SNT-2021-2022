\documentclass[a4paper]{article}

\input ../header
\usepackage{minted}
\usepackage[np]{numprint}
\usepackage{lscape}
\usepackage{afterpage}
\usepackage{hyperref}
\usepackage{gensymb}

\setlength{\multicolsep}{2pt}

% Commandes pour cacher/révéler du texte facilement à l'aide d'un booléen
\usepackage{xstring}
\usepackage{ifthen}

\newboolean{reveal}
\setboolean{reveal}{false}

\newlength{\stextwidth} % une nouvelle longueur

\newcommand\x{6}

\newcommand{\guess}[1]{\ifthenelse{\boolean{reveal}}{{\color{red}#1}}{\settowidth{\stextwidth}{#1}\makebox[\stextwidth]{\dotfill}}}

\newcommand{\guessmath}[1]{\ifthenelse{\boolean{reveal}}{\textcolor{red}{#1}}{\settowidth{\stextwidth}{$#1$}\makebox[1.9\stextwidth]{\dotfill}}}

\newcommand{\guessmathbin}[1]{\ifthenelse{\boolean{reveal}}{\mathbin{\color{red}#1}}{\settowidth{\stextwidth}{$#1$}\makebox[2\stextwidth]{\dotfill}}}

\begin{document}

\title{Chapitre 6 -- Géolocalisation}

\pagestyle{empty}

\date{}
\author{}

\maketitle{}

\thispagestyle{empty}
\noindent\textbf{Activité 5}\hfill{}\textbf{Algorithme de Dijkstra}
\smallskip
\hrule
\medskip

On considère le graphe ci-dessous, et on s'intéresse aux chemins d'origine A :

\medskip

\begin{center}
  \scalebox{0.7}{$
    \begin{psmatrix}[mnode=circle]
 &  & A &  &  & B \\ 
 &  &  &  &  &  &  & G \\ 
      D &  &  &  & F \\ 
	&  & C \\ 
	&  & E &  &  &  & H 
  \end{psmatrix}$}
  \psset{arrows=-,shortput=nab}
  \ncline{1,3}{1,6}^{12}
  \ncline{1,3}{3,1}^{14}
  \ncline{1,6}{3,5}^{9}
  \ncline{1,6}{2,8}^{16}
  \ncline{1,6}{5,7}^{21}
  \ncline{4,3}{5,3}^{13}
  \ncline{4,3}{3,5}^{10}
  \ncline{3,1}{5,3}^{10}
  \ncline{5,3}{5,7}^{10}
  \ncline{3,5}{5,7}^{11}
  \ncline{2,8}{5,7}^{11}
\end{center}

\medskip

\begin{enumerate}
  \item Le chemin A\,--\,D\,--\,E\,--\,C est un chemin d'origine A et d'extrémité C, de longueur 37. S'agit-il du plus court chemin de A vers C ?
  \item Déterminer le plus court chemin de A vers H.
  \item On considère maintenant l'algorithme suivant :

    \medskip

    \fbox{
      \begin{minipage}{0.5\textwidth}
	  Étape 1 : on commence en notant $0$ sur le sommet de départ et $\infty$ sur les autres sommets.\\
	  \smallskip
	  Étape 2 : parmi les sommets, on {\color{red}choisit} le sommet \og{}le plus petit\fg{} et on le {\color{red}traite} :
	    \begin{itemize}
	      \item on ajoute la distance notée sur ce sommet à la distance vers un autre sommet accessible ;
	      \item si le résultat est inférieur à la distance notée sur le sommet d'arrivée, on modifie cette distance ;
	      \item on marque le sommet comme traité.
	    \end{itemize}
	  Étape 3 : on recommence l'étape 2 tant que tous les sommets n'ont pas été traités.
      \end{minipage}
    }\hspace*{5mm}
    \begin{tabular}{@{}cccccccc@{}}
      \toprule
      A & B & C & D & E & F & G & H\\
      \midrule
      \fbox{$0$} & $\infty$ & $\infty$ & $\infty$ & $\infty$ & $\infty$ & $\infty$ & $\infty$\\
      $/$ & \fbox{$12_A$} & $\infty$ & $14_A$ & $\infty$ & $\infty$ & $\infty$ & $\infty$\\
      $/$ & $/$ & $\infty$ & $14_A$ & $\infty$ & $21_B$ & $28_B$ & $33_B$\\
      $/$ & $/$ & & & & & & \\
      $/$ & $/$ & & & & & & \\
      $/$ & $/$ & & & & & & \\
      $/$ & $/$ & & & & & & \\
      $/$ & $/$ & & & & & & \\

      %$/$ & $/$ & $\infty$ & $ /$ & $24_D$ & \fbox{$21_B$} & $28_B$ & $33_B$\\
      %$/$ & $/$ & $31_F$ & $/$ & \fbox{$24_D$} & $/$ & $28_B$ & $32_F$\\
      %$/$ & $/$ & $31_F$ & $/$ & $/$ & $/$ & \fbox{$28_B$} & $32_F$\\
      %$/$ & $/$ & \fbox{$31_F$} & $/$ & $/$ & $/$ & $/$ & $32_F$\\
      %$/$ & $/$ & $/$ & $/$ & $/$ & $/$ & $/$ & \fbox{$32_F$}\\
      \bottomrule
    \end{tabular}

    \medskip

    On choisit le sommet A comme sommet de départ. Le début du déroulement de l'algorithme est détaillé ci-dessous et résumé dans le tableau précédent :
    \textit{
      \begin{itemize}
	\item On choisit le sommet A puis on considère les sommets accessibles à partir de A :
	  \begin{itemize}
	    \item le sommet B : $0+12=12$ qui est inférieur à $\infty$ donc on modifie la distance dans la colonne de B ;
	    \item le sommet D : $0+14=14$ qui est inférieur à $\infty$ donc on modifie la distance dans la colonne de~D ;
	    \item on marque le sommet A comme traité.
	  \end{itemize}
	\item On choisit le sommet B puis on considère les sommets accessibles à partir de B :
	  \begin{itemize}
	    \item le sommet F : $12+9=21$ qui est inférieur à $\infty$ donc on modifie la distance dans la colonne de F ;
	    \item le sommet G : $12+16=28$ qui est inférieur à $\infty$ donc on modifie la distance dans la colonne de~G ;
	    \item le sommet H : $12+21=33$ qui est inférieur à $\infty$ donc on modifie la distance dans la colonne de~H ;
	    \item on marque le sommet B comme traité.
	  \end{itemize}
      \end{itemize}
    }
    Quel est le prochain sommet choisi ?
  \item Terminer le déroulement de l'algorithme et compléter le tableau.
  \item Que signifie le dernier nombre de chacune des colonnes ?
  \item Comment retrouver les plus courts chemins de A vers C et de A vers H à l'aide du tableau ?
  \item L'algorithme présenté s'appelle l'\href{https://fr.wikipedia.org/wiki/Algorithme_de_Dijkstra}{algorithme} de \href{https://fr.wikipedia.org/wiki/Edsger_Dijkstra}{Dijkstra}. Utiliser à nouveau l'algorithme de Dijkstra afin de déterminer les plus courts chemins d'origine A du graphe ci-dessous :

    \medskip

    \begin{center}
      \scalebox{0.7}{$
	\begin{psmatrix}[mnode=circle]
 &  &  &  & A \\ 
 &  & B &  &  &  & C \\ 
	  D &  &  &  &  &  &  & F \\ 
	    &  &  & E \\ 
	    & G &  &  &  &  & I \\ 
	    &  &  & H \\ 
	    &  &  & K &  &  & J 
	\end{psmatrix}
      $}
      \psset{arrows=-,shortput=nab}
      \ncline{1,5}{2,3}^{18}
      \ncline{1,5}{2,7}^{22}
      \ncline{2,3}{2,7}^{31}
      \ncline{2,3}{3,1}^{12}
      \ncline{2,3}{4,4}^{26}
      \ncline{2,7}{3,8}^{17}
      \ncline{3,1}{5,2}^{24}
      \ncline{4,4}{3,8}^{12}
      \ncline{4,4}{5,2}^{9}
      \ncline{3,8}{5,7}^{13}
      \ncline{5,2}{6,4}^{12}
      \ncline{6,4}{5,7}^{7}
      \ncline{6,4}{7,4}^{24}
      \ncline{5,7}{7,7}^{12}
      \ncline{7,7}{7,4}^{18}
    \end{center}
\end{enumerate}
\end{document}
