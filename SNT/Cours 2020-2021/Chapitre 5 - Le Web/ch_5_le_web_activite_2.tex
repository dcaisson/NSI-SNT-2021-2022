\documentclass[a4paper]{article}

\input ../header
\usepackage{minted}
\usepackage[np]{numprint}
\usepackage{lscape}
\usepackage{afterpage}
\usepackage{hyperref}

\setlength{\multicolsep}{2pt}

% Commandes pour cacher/révéler du texte facilement à l'aide d'un booléen
\usepackage{xstring}
\usepackage{ifthen}

\newboolean{reveal}
\setboolean{reveal}{false}

\newlength{\stextwidth} % une nouvelle longueur

\newcommand\x{6}

\newcommand{\guess}[1]{\ifthenelse{\boolean{reveal}}{{\color{red}#1}}{\settowidth{\stextwidth}{#1}\makebox[\stextwidth]{\dotfill}}}

\newcommand{\guessmath}[1]{\ifthenelse{\boolean{reveal}}{\textcolor{red}{#1}}{\settowidth{\stextwidth}{$#1$}\makebox[1.9\stextwidth]{\dotfill}}}

\newcommand{\guessmathbin}[1]{\ifthenelse{\boolean{reveal}}{\mathbin{\color{red}#1}}{\settowidth{\stextwidth}{$#1$}\makebox[2\stextwidth]{\dotfill}}}

\begin{document}

\title{Le Web -- Activité 2}

\pagestyle{empty}

\date{}
\author{}

\maketitle{}

\thispagestyle{empty}
\noindent\textbf{Activité 2}\hfill{}\textbf{Repères historiques}
\smallskip
\hrule
\medskip

Voici quelques repères historiques importants :

\begin{itemize}
  \item 1965 : apparition du terme \og{}hypertexte\fg{} attribué au sociologue américain Ted Nelson ;
  \item 1981 : développement d'un programme de stockage d'informations incluant des hypertextes au CERN ;
  \item 1989 : invention du \textit{World Wide Web} par Tim Berners-Lee ;
  \item 1993 : création du navigateur Mosaic qui va populariser le Web ;
  \item 1994 : création d'annuaires (Yahoo) et de moteurs de recherche (Altavista) qui permettent l'indexation du Web ;
  \item 1994 : création du W3C qui vise à uniformiser les standards du Web ;
  \item 1995 : les pages deviennent dynamiques grâce à deux langages : Javascript et PHP ;
  \item 1998 : standardisation des pages internet grâce au DOM (\textit{Document Object Model});
  \item 1998 : naissance du moteur de recherche Google ;
  \item 2004 : naissance du navigateur Firefox ;
  \item 2005 : arrivée de la vidéo sur le Web (Vimeo, Dailymotion, YouTube) ;
  \item 2010 : mise à disposition de technologies pour le développement d'applications mobiles ;
\end{itemize}

Réaliser une frise chronologique, puis la déposer au format PDF sur Moodle. Le nom du fichier déposé sera au format: 

\begin{center}
  \verb|[Classe]-[NOM]_[Prénom]_frise-web.pdf|,
\end{center} 

par exemple:

\begin{center}
  \verb|Seconde-15_RICARD_Louis_frise-web.pdf|.
\end{center}

Illustrer au moins la moitié des dates à l'aide d'une image en faisant attention aux droits d'utilisation de chaque image.

\end{document}
