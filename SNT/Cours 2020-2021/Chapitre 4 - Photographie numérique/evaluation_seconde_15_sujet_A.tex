\documentclass[a4paper]{article}

\input ../header
\usepackage[np]{numprint}

\setlength{\multicolsep}{2pt}

\begin{document}

\title{Sujet A -- La photographie numérique}

\pagestyle{empty}

\date{}
\author{}

\maketitle{}
\thispagestyle{empty}

% QCM histoire de la photographie numérique
\exo[4 points] Cet exercice est un QCM. Pour chaque question posée, une seule des réponses proposées est correcte. Entourer sur l'énoncé la réponse que vous pensez être correcte.\\
Une bonne réponse rapporte 1 point, une mauvaise réponse ou l'absence de réponse ne rapporte  ni n'enlève aucun point. Le fait d'entourer plusieurs réponses pour une question donnée rapporte 0 point.

\begin{enumerate}
  \item Un photosite, c'est :
    \begin{multicols}{4}
      \begin{enumerate}
	\item un site de partage de photos
	\item un composant élémentaire d'une image
	\item un composant élémentaire d'un capteur
	\item un site de production photovoltaïque
      \end{enumerate}
    \end{multicols}
  \item Dans une image numérique, un pixel :
    \begin{multicols}{4}
      \begin{enumerate}
	\item a une taille fixe sur tous les écrans
	\item a une taille fixe sur toutes les imprimantes
	\item n'a pas de taille prédéfinie
	\item a une taille variable selon la couleur de l'image
      \end{enumerate}
    \end{multicols}
  \item Le nombre de pixels par pouce s'appelle :
    \begin{multicols}{4}
      \begin{enumerate}
	\item la résolution de l'image
	\item la résolution du capteur
	\item la définition de l'image
	\item la définition du capteur
      \end{enumerate}
    \end{multicols}
  \item Je suis un physicien et mathématicien écossais. J'ai présenté la première photographie en vraie couleur. Je suis :
    \begin{multicols}{4}
      \begin{enumerate}
	\item Nicéphore Niepce
	\item Thomas Sutton
	\item James Clerk Maxwell
	\item Louis Daguerre
      \end{enumerate}
    \end{multicols}
\end{enumerate}

\bigskip

\exo[1 point] Compléter la phrase ci-dessous :

\begin{center}
  \og{}Je suis un composé chimique ouvrant la voie à la photographie argentique. Je suis \dotfill\fg{}
\end{center}

\bigskip

\exo[3 points] Expliquer les similitudes entre l'oeil humain et un appareil photo numérique.\rep{8}

\bigskip

% Calcul de la définition
\exo[3 points] Luca souhaite imprimer une photo aux dimensions $63,5$ cm $\times$ $38,1$ cm avec une résolution de $300$ dpi. Quelle définition minimale devra avoir le fichier fourni par Luca ?\rep{10}

\pagebreak

% Calcul de la résolution
\exo[3 points] Voici une publicité pour le Samsung Galaxy S21 :

\medskip

\begin{multicols}{2}
  \begin{enumerate}
    \item [] \includegraphics[width=6cm]{evaluation_samsung_galaxy_S21_1.png}
    \item [] \includegraphics[width=6cm]{evaluation_samsung_galaxy_S21_2.png}
  \end{enumerate}
  Retrouver, par le calcul, la résolution annoncée.\rep{10}
\end{multicols}

\medskip


\bigskip

% Étude de documents sur le droit à l'image
\exo[3 points] Lire les documents suivants :

\begin{center}
  \includegraphics[width=16cm]{evaluation_droit_a_l_oubli.png}
\end{center}

\begin{enumerate}
  \item Compléter le texte suivant :

    \medskip

    Le $\hdots\hdots\hdots\hdots\hdots\hdots$ des données à caractère personnel permet à toute personne de faire $\hdots\hdots\hdots\hdots\hdots\hdots$ une image ou des informations afin de protéger $\hdots\hdots\hdots\hdots\hdots\hdots$ Mais tôt ou tard, ces $\hdots\hdots\hdots\hdots\hdots\hdots$ peuvent ressurgir ce qui re $\hdots\hdots\hdots\hdots\hdots\hdots$ de cette loi très difficile.
  \item Proposer un argument en faveur du droit à l'oubli et un argument contre son application systématique.
    \begin{multicols}{2}
      \begin{enumerate}
	\item Pour\rep{6}
	\item Contre\rep{6}
      \end{enumerate} 
    \end{multicols}
\end{enumerate}

\end{document}
