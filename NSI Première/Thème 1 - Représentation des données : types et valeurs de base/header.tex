% ======================================================================
% =                              langue                                =
% ======================================================================

\usepackage[french]{babel}

% ======================================================================
% =                      identation et marges                          =
% ======================================================================

\setlength{\parindent}{0mm}
\usepackage[left=2.5cm,right=2.5cm,top=2cm,bottom=2cm]{geometry}

% ======================================================================
% =                                AMS                                 =
% ======================================================================

\usepackage{amsmath, amsfonts, amsthm}

% ======================================================================
% =                  couleur des sections et sous-sections             =
% ======================================================================

\usepackage{sectsty}
\sectionfont{\color{Red}}
\subsectionfont{\color{Red!50}}

% ======================================================================
% =                          tcolorbox                                 =
% ======================================================================

\usepackage[most]{tcolorbox}
\tcbuselibrary{theorems}

%%% Environnements théorème et théorème admis
%%% Les environnements se partagent le même compteur

\newcounter{mythm}

\newtcbtheorem[use counter=mythm]{theorem}{Théorème}{
  enhanced,
  attach boxed title to top left={
    xshift=-1mm,
    yshift=-2.5mm
  },
  top=1em,
  colback=gray!5,
  colframe=red!45,
  boxrule=1pt,
  fonttitle=\bfseries,
  boxed title style={
    %sharp corners,
    size=small,
    colback=red!45,
    colframe=red!45,
  } 
}{thm}

\newtcbtheorem[use counter=mythm]{theorem-admis}{Théorème}{
  enhanced,
  attach boxed title to top left={
    xshift=-1mm,
    yshift=-2.5mm
  },
  top=1em,
  after title={\ (admis)},
  colback=gray!5,
  colframe=red!45,
  boxrule=1pt,
  fonttitle=\bfseries,
  boxed title style={
    %sharp corners,
    size=small,
    colback=red!45,
    colframe=red!45,
  } 
}{thma}

%%% Environnements proposition et proposition admise
%%% Les environnements se partagent le même compteur

\newcounter{myprop}

\newtcbtheorem[use counter=myprop]{proposition}{Proposition}{
  enhanced,
  attach boxed title to top left={
    xshift=-1mm,
    yshift=-2.5mm
  },
  top=1em,
  colback=gray!5,
  colframe=red!45,
  boxrule=1pt,
  fonttitle=\bfseries,
  boxed title style={
    %sharp corners,
    size=small,
    colback=red!45,
    colframe=red!45,
  } 
}{prop}

\newtcbtheorem[use counter=myprop]{proposition-admise}{Proposition}{
  enhanced,
  attach boxed title to top left={
    xshift=-1mm,
    yshift=-2.5mm
  },
  after title={\ (admise)},
  top=1em,
  colback=gray!5,
  colframe=red!45,
  boxrule=1pt,
  fonttitle=\bfseries,
  boxed title style={
    %sharp corners,
    size=small,
    colback=red!45,
    colframe=red!45,
  } 
}{propa}

%%% Environnement exemple

\newtcbtheorem[auto counter]{exemple}{Exemple}{
  enhanced,
  attach boxed title to top left={
    xshift=-1mm,
    yshift=-2.5mm
  },
  top=1em,
  colback=gray!5,
  colframe=ForestGreen!45,
  boxrule=1pt,
  fonttitle=\bfseries,
  boxed title style={
    %sharp corners,
    size=small,
    colback=ForestGreen!45,
    colframe=ForestGreen!45,
  } 
}{expl}

%%% Environnement exercice

\newtcbtheorem[auto counter]{exercice}{Exercice}{
  enhanced,
  attach boxed title to top left={
    xshift=-1mm,
    yshift=-2.5mm
  },
  top=1em,
  colback=gray!5,
  colframe=ForestGreen!45,
  boxrule=1pt,
  fonttitle=\bfseries,
  boxed title style={
    %sharp corners,
    size=small,
    colback=ForestGreen!45,
    colframe=ForestGreen!45,
  } 
}{exo}

%%% Environnement application

\newtcbtheorem[auto counter]{appl}{Application}{
  enhanced,
  attach boxed title to top left={
    xshift=-1mm,
    yshift=-2.5mm
  },
  top=1em,
  colback=gray!5,
  colframe=ForestGreen!65,
  boxrule=1pt,
  fonttitle=\bfseries,
  boxed title style={
    %sharp corners,
    size=small,
    colback=ForestGreen!65,
    colframe=ForestGreen!65,
  } 
}{appl}

%%% Environnement travailler l'oral

\newtcbtheorem[auto counter]{oral}{Travailler l'oral}{
  enhanced,
  attach boxed title to top left={
    xshift=-1mm,
    yshift=-2.5mm
  },
  top=1em,
  colback=gray!5,
  boxrule=1pt,
  colframe=RoyalBlue!45,
  fonttitle=\bfseries,
  boxed title style={
    %sharp corners,
    size=small,
    colback=RoyalBlue!45,
    colframe=RoyalBlue!45,
  } 
}{oral}

%%% Environnement remarque

\newtcbtheorem[auto counter]{remarque}{Remarque}{
  enhanced,
  attach boxed title to top left={
    xshift=-1mm,
    yshift=-2.5mm
  },
  top=1em,
  colback=gray!5,
  boxrule=1pt,
  colframe=Cerulean!45,
  fonttitle=\bfseries,
  boxed title style={
    %sharp corners,
    size=small,
    colback=Cerulean!45,
    colframe=Cerulean!45,
  } 
}{rq}

%%% Environnement notation

\newtcbtheorem[auto counter]{notation}{Notation}{
  enhanced,
  attach boxed title to top left={
    xshift=-1mm,
    yshift=-2.5mm
  },
  top=1em,
  colback=gray!5,
  boxrule=1pt,
  colframe=Cerulean!45,
  fonttitle=\bfseries,
  boxed title style={
    %sharp corners,
    size=small,
    colback=Cerulean!45,
    colframe=Cerulean!45,
  } 
}{nota}

%%% Environnement définition

\newtcbtheorem[auto counter]{definition}{Définition}{
  enhanced,
  attach boxed title to top left={
    xshift=-1mm,
    yshift=-2.5mm
  },
  top=1em,
  colback=gray!5,
  colframe=Peach!45,
  boxrule=1pt,
  fonttitle=\bfseries,
  boxed title style={
    %sharp corners,
    size=small,
    colback=Peach!45,
    colframe=Peach!45,
  } 
}{defn}

%%% Environnement méthode

\newtcbtheorem[auto counter]{methode}{Méthode}{
  enhanced,
  attach boxed title to top left={
    xshift=-1mm,
    yshift=-2.5mm
  },
  top=1em,
  colback=gray!5,
  colframe=CadetBlue!45,
  boxrule=1pt,
  fonttitle=\bfseries,
  boxed title style={
    %sharp corners,
    size=small,
    colback=CadetBlue!45,
    colframe=CadetBlue!45,
  } 
}{mthd}


%%% Environnement activite

\newtcbtheorem[auto counter]{activite}{Activité}{
  enhanced,
  attach boxed title to top left={
    xshift=-1mm,
    yshift=-2.5mm
  },
  top=1em,
  colback=gray!5,
  colframe=ForestGreen!65,
  boxrule=1pt,
  fonttitle=\bfseries,
  boxed title style={
    %sharp corners,
    size=small,
    colback=ForestGreen!65,
    colframe=ForestGreen!65,
  } 
}{activite}

% ======================================================================
% =                      tableaux de variations                        =
% ======================================================================

\usepackage{tkz-tab}

% ======================================================================
% =                           symbole euro                             =
% ======================================================================

\usepackage{eurosym}

% ======================================================================
% =                     continuer une énumération                      =
% ======================================================================

\usepackage{enumitem}

% ======================================================================
% =                       tableaux plus élaborés                       =
% ======================================================================

\usepackage{array}

% ======================================================================
% =                       insérer des graphiques                       =
% ======================================================================

\usepackage{graphicx}

% ======================================================================
% =                    écrire sur plusieurs colonnes                   =
% ======================================================================

\usepackage{multicol}

% ======================================================================
% =                fusionner plusieurs lignes d'un tableau             =
% ======================================================================

\usepackage{multirow}

% ======================================================================
% =                             PSTricks                               =
% ======================================================================

\usepackage{pstricks,pst-plot,pst-text,pst-tree,pst-eps,pst-fill,pst-node,pst-math}
\usepackage{pstricks-add,pst-xkey}

% ======================================================================
% =                        pour de beaux tableaux                      =
% ======================================================================

\usepackage{booktabs}

% ======================================================================
% =                    griser des cellules d'un tableau                =
% ======================================================================

\usepackage{colortbl}

% ======================================================================
% =                            les liens                               =
% ======================================================================

\usepackage{hyperref}
\hypersetup{
  colorlinks=true,
  linkcolor=CornflowerBlue,
  linktoc=all
}

% ======================================================================
% =                      coloration syntaxique                         =
% ======================================================================

\usepackage{minted}
\renewcommand{\theFancyVerbLine}{\sffamily\scriptsize\arabic{FancyVerbLine}}

% ======================================================================
% =                      tableaux de variations                        =
% ======================================================================

%\input ../tabvar

% ======================================================================
% =                     interpolation de Hermite                       =
% ======================================================================

%\input ../HermiteDDL

% ======================================================================
% =                     symbole virage dangereux                       =
% ======================================================================

\usepackage{manfnt}

% ======================================================================
% =                       écriture des nombres                         =
% ======================================================================
\usepackage[np]{numprint}

% ======================================================================
% =                       polices de caractères                        =
% ======================================================================

\usepackage{fontspec}
\usepackage{eulervm}
\setmainfont{CMU Concrete}

% ======================================================================
% =                        macros personnelles                         =
% ======================================================================

\newcommand{\vect}[1]{\overrightarrow{#1}}
\newcommand{\Oij}{\left(O;\vect{i},\vect{j}\right)}
\newcommand{\norm}[1]{\left|\left|#1\right|\right|}

\usepackage{pgffor} % pour la commande \foreach permettant de réaliser une boucle
\newcommand{\pointilles}{{\\\rule{0pt}{1pt}\dotfill\rule{0pt}{1pt}}}
\newcommand{\rep}[1]{\foreach \n in {1,...,#1} {\pointilles}}
\newcommand{\points}[1]{\foreach \n in {1,...,#1} {\dots}}


% ======================================================================
% =                      en-tête et pied de page                       =
% ======================================================================

\usepackage{lastpage}
\usepackage{fancyhdr}

\renewcommand{\headrulewidth}{0pt}
\renewcommand{\footrulewidth}{0pt}

\lhead{}
\chead{}
\rhead{}
\lfoot{}
\cfoot{Page \thepage\ sur\ \pageref{LastPage}}
\rfoot{}

\pagestyle{fancy}

% ======================================================================
% =                      algorithmes pseudo-code                       =
% ======================================================================

\usepackage{algpseudocode}
\usepackage{eqparbox}
\algrenewcommand\algorithmicfor{\textbf{Pour}}
\algrenewcommand\algorithmicdo{}
\algrenewcommand\algorithmicend{\textbf{Fin}}
\algrenewcommand\algorithmicwhile{\textbf{Tant que}}
\algrenewcommand\algorithmicif{\textbf{Si}}
%\algrenewcommand\algorithmicelsif{\textbf{Sinon Si}}
\algrenewcommand\algorithmicelse{\textbf{Sinon}}
\algrenewcommand\algorithmicthen{\textbf{alors}}
\algrenewcommand\algorithmiccomment[1]{\hfill\eqparbox{COMMENT}{$\rightarrow$ #1}}
