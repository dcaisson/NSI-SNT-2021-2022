\documentclass[a4paper,dvipsnames]{article}

\input ../../header

\newcommand{\un}{\left(u_n\right)_{n\in\mathbb{N}}}
\newcommand{\uns}{\left(u_n\right)_{n\in\mathbb{N}^\ast}}
\newcommand{\vn}{\left(v_n\right)_{n\in\mathbb{N}}}
\newcommand{\vns}{\left(v_n\right)_{n\in\mathbb{N}^\ast}}
\newcommand{\wn}{\left(w_n\right)_{n\in\mathbb{N}}}
\newcommand{\wns}{\left(w_n\right)_{n\in\mathbb{N}^\ast}}
\newcommand{\tn}{\left(t_n\right)_{n\in\mathbb{N}}}
\newcommand{\tns}{\left(t_n\right)_{n\in\mathbb{N}^\ast}}
\newcommand{\qn}{\left(q_n\right)_{n\in\mathbb{N}}}
\newcommand{\qns}{\left(q_n\right)_{n\in\mathbb{N}^\ast}}
\newcommand{\N}{\mathbb{N}}

\usepackage{dirtree}

\title{Devoir maison 3}

\author{}
\date{}

\begin{document}

\renewcommand{\contentsname}{}

\pagestyle{fancy}

\begin{tcolorbox}[colframe=blue!75, colback=blue!45, valign=center, height=1.5cm, top=5mm]
  \maketitle
\end{tcolorbox}

%\tableofcontents

\vspace{1cm}

\thispagestyle{fancy}

\begin{exercice}{}{}
  \begin{enumerate}
    \item Écrire une fonction \mintinline{python}{maximum} qui prend en paramètre une liste de nombres et qui renvoie la valeur maximale de ces nombres.

      \begin{minted}[xleftmargin=1cm,xrightmargin=1cm,frame=single]{python}
assert maximum([3, 5, -9, 2, 5]) == 5
      \end{minted}

    \item Écrire une fonction \mintinline{python}{position_maximum} qui prend en paramètre une liste de nombres et qui renvoie la position de la valeur maximale de ces nombres.

      \begin{minted}[xleftmargin=1cm,xrightmargin=1cm,frame=single]{python}
assert position_maximum([3, 5, -9, 2, 5]) == 2
      \end{minted}

    \item Écrire une fonction \mintinline{python}{positions_maximum} qui prend en paramètre une liste de nombres et qui renvoie une liste contenant les positions de la valeur maximale de ces nombres.

      \begin{minted}[xleftmargin=1cm,xrightmargin=1cm,frame=single]{python}
assert positions_maximum([3, 5, -9, 2, 5]) == [1, 4]
      \end{minted}
  \end{enumerate}
\end{exercice}

\medskip

\begin{exercice}{}{}
  Écrire une fonction \mintinline{python}{occurrences} qui prend en paramètre une liste d'entiers et un entier et qui renvoie le nombre de fois où ce dernier apparaît dans la liste.

      \begin{minted}[xleftmargin=1cm,xrightmargin=1cm,frame=single]{python}
assert occurrences([3, 5, -9, 5, 4], 5) == 2
assert occurrences([3, 5, -9, 5, 4], 1) == 0
      \end{minted}

\end{exercice}

\medskip

\begin{exercice}{}{}
  Écrire une fonction \mintinline{python}{entremele} qui prend en paramètres deux listes de même longueur et qui renvoie une liste contenant alternativement un élément de la première liste suivi d'un élément de la seconde.

      \begin{minted}[xleftmargin=1cm,xrightmargin=1cm,frame=single]{python}
assert entremele([1, 2, 3], [5, 6, 7]) == [1, 5, 2, 6, 3, 7]
      \end{minted}

\end{exercice}

\medskip

\begin{exercice}{}{}
  Les sorciers ne comptent pas en euros. Ce serait trop facile. Les sorciers fortunés comptent en gallions. Un gallion vaut $17$ mornilles. Et une mornille vaut $29$ noises.

  Sirius a laissé une belle somme à Harry. C'est un nombre entier de gallions, mais qui ne dépasse tout de même pas le million.

  Cette somme a une particularité : qu'on l'exprime en gallions, en mornilles ou en noises, dans les trois cas, on a besoin exactement des mêmes chiffres (utilisés éventuellement plusieurs fois) pour l'écrire.

  \begin{enumerate}
    \item La somme de $125$ gallions s'écrit avec les chiffres $1$, $2$ et $5$. À combien de mornilles équivaut cette somme ?
    \item À combien de noises équivaut-elle ?
    \item Expliquer alors pourquoi la somme laissée par Sirius n'est pas égale à $125$ gallions.
    \item En utilisant Python, retrouver la somme donnée par Sirius à Harry (en gallions).
  \end{enumerate}
\end{exercice}
\end{document}
