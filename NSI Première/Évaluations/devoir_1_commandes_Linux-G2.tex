\documentclass[a4paper,dvipsnames]{article}

\input ../header

\newcommand{\un}{\left(u_n\right)_{n\in\mathbb{N}}}
\newcommand{\uns}{\left(u_n\right)_{n\in\mathbb{N}^\ast}}
\newcommand{\vn}{\left(v_n\right)_{n\in\mathbb{N}}}
\newcommand{\vns}{\left(v_n\right)_{n\in\mathbb{N}^\ast}}
\newcommand{\wn}{\left(w_n\right)_{n\in\mathbb{N}}}
\newcommand{\wns}{\left(w_n\right)_{n\in\mathbb{N}^\ast}}
\newcommand{\tn}{\left(t_n\right)_{n\in\mathbb{N}}}
\newcommand{\tns}{\left(t_n\right)_{n\in\mathbb{N}^\ast}}
\newcommand{\qn}{\left(q_n\right)_{n\in\mathbb{N}}}
\newcommand{\qns}{\left(q_n\right)_{n\in\mathbb{N}^\ast}}
\newcommand{\N}{\mathbb{N}}

\usepackage{dirtree}

\title{Devoir 1 -- Commandes Linux}

\author{}
\date{}

\begin{document}

\renewcommand{\contentsname}{}

\pagestyle{fancy}

\begin{tcolorbox}[colframe=blue!75, colback=blue!45, valign=center, height=1.5cm, top=5mm]
  \maketitle
\end{tcolorbox}

%\tableofcontents

\vspace{1cm}

\thispagestyle{fancy}

\begin{exercice}{}{}
  On considère l'arborescence suivante :

  \bigskip

  \dirtree{%
    .1 \textbf{Téléchargements}.
    .2 \textbf{Documents}.
    .3 linux.pdf.
    .3 outils\_python.ipynb.
    .2 \textbf{Vidéos}.
    .3 \textbf{Vacances}.
    .4 sky\_tower.mp4.
    .4 casino.mp4.
    .2 \textbf{Musique}.
    .3 \textbf{Beatles}.
    .3 \textbf{Clapton}.
    .4 wonderful\_tonight.mp3.
    .4 tears\_in\_heaven.mp3.
    .3 \textbf{ABBA}.
    .4 \textbf{Best\_of}.
    .5 dancing\_queen.mp3.
  }

  \bigskip

  Le chemin absolu vers le dossier \mintinline{bash}{Téléchargements} est \mintinline{bash}{/home/alice/Téléchargements}.

  \begin{enumerate}
    \item Donner le chemin absolu vers le fichier \mintinline{bash}{casino.mp4}.
    \item On se trouve dans le dossier \mintinline{bash}{Téléchargements}. Donner le chemin relatif vers le fichier \mintinline{bash}{sky_tower.mp4}.
    \item On se trouve dans le dossier \mintinline{bash}{Clapton}. Donner le chemin relatif vers le fichier \mintinline{bash}{dancing_queen.mp3}.
  \end{enumerate}
\end{exercice}

\smallskip

\begin{exercice}{}{}
  \begin{enumerate}
    \item Un élève se trouve dans le répertoire \mintinline{bash}{Documents}. Il entre les commandes suivantes dans un terminal :

      \begin{minted}{bash}
$ mkdir Dossier_1
$ mkdir Dossier_2
$ cd Dossier_2
$ touch fichier_1
$ touch fichier_2
      \end{minted}

      Préciser le contenu du répertoire \mintinline{bash}{Documents} après ces commandes.
    \item Les commandes suivantes sont entrées :

      \begin{minted}{bash}
$ cd ..
$ mkdir Dossier_4
$ cp Dossier_2/fichier_1 Dossier_4/fichier_1
$ mkdir Dossier_3
$ mv Dossier_2/fichier_2 Dossier_3/fichier_3
      \end{minted}

      Préciser le contenu du répertoire \mintinline{bash}{Documents} après ces commandes.
    \item Écrire une commande permettant d'effacer le dossier \mintinline{bash}{Dossier_2}.
  \end{enumerate}
\end{exercice}

\smallskip

\begin{exercice}{}{}
  On se trouve dans un répertoire vide \mintinline{bash}{Travail}. Écrire des commandes permettant d'obtenir l'arborescence suivante :
  
  \bigskip

  \dirtree{%
    .1 \textbf{Travail}.
    .2 \textbf{NSI}.
    .3 commandes\_linux.
    .3 aide-memoire\_python.
    .2 \textbf{Mathématiques}.
    .3 \textbf{Devoirs}.
    .4 \textbf{Suites}.
    .5 cours.pdf.
    .4 dm1.
    .4 dm2.
  }

\end{exercice}
\end{document}
