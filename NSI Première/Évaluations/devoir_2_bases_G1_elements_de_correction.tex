\documentclass[a4paper,dvipsnames]{article}

\input ../header

\newcommand{\un}{\left(u_n\right)_{n\in\mathbb{N}}}
\newcommand{\uns}{\left(u_n\right)_{n\in\mathbb{N}^\ast}}
\newcommand{\vn}{\left(v_n\right)_{n\in\mathbb{N}}}
\newcommand{\vns}{\left(v_n\right)_{n\in\mathbb{N}^\ast}}
\newcommand{\wn}{\left(w_n\right)_{n\in\mathbb{N}}}
\newcommand{\wns}{\left(w_n\right)_{n\in\mathbb{N}^\ast}}
\newcommand{\tn}{\left(t_n\right)_{n\in\mathbb{N}}}
\newcommand{\tns}{\left(t_n\right)_{n\in\mathbb{N}^\ast}}
\newcommand{\qn}{\left(q_n\right)_{n\in\mathbb{N}}}
\newcommand{\qns}{\left(q_n\right)_{n\in\mathbb{N}^\ast}}
\newcommand{\N}{\mathbb{N}}

\usepackage{dirtree}

\title{Devoir 2 -- Représentation des entiers -- Éléments de correction}

\author{}
\date{}

\begin{document}

\renewcommand{\contentsname}{}

\pagestyle{fancy}

\begin{tcolorbox}[colframe=blue!75, colback=blue!45, valign=center, height=1.5cm, top=5mm]
  \maketitle
\end{tcolorbox}

%\tableofcontents

\vspace{1cm}

\thispagestyle{fancy}

\begin{exercice}{2 points}{}
 \begin{enumerate}
   \item Combien de valeurs différentes peut-on représenter à l'aide de $8$ bits ?

     \begin{correction}
       À l'aide de $8$ bits, on peut représenter $2^8$ valeurs différentes.
     \end{correction}
     
   \item Quel est le plus grand entier naturel qu'on peut représenter à l'aide de $16$ bits ?

     \begin{correction}
       Le plus grand entier naturel qu'on peut représenter à l'aide de $16$ bits est $2^{16}-1$.
     \end{correction}
     
 \end{enumerate} 
\end{exercice}

\medskip

\begin{exercice}{4 points}{}
  Voici un court extrait de la biographie de Margaret Hamilton :

  \begin{center}
    \begin{tabular}{|p{12cm}}
      Margaret Heafield Hamilton, née Margaret Heafield le 17 août 1936, est une informaticienne, ingénieure système et cheffe d'entreprise américaine. Elle était directrice du département génie logiciel (\og{}software engineering\fg{}, terme de son invention) au sein du MIT Instrumentation Laboratory qui conçut le système embarqué du programme spatial Apollo. En 1986, elle fonde la société Hamilton Technologies, Inc. à partir de ses travaux entrepris au MIT.  
    \end{tabular}

    \smallskip

    \begin{minipage}{12cm}
      \flushright\textit{Source : Wikipédia}
    \end{minipage}
  \end{center}

  \begin{enumerate}
    % base 10 vers la base 2
    \item Écrire le jour de naissance de Margaret Hamilton en base $2$.

      \begin{correction}
	$17=\overline{1\,0001}^2$
      \end{correction}
      
      % réponse : 10001
    % base 10 vers la base 16
    \item Écrire l'année de la fondation de la société Hamilton Technologies, Inc. en base $16$.

      \begin{correction}
	$1986=\overline{7\text{C}2}^{16}$
      \end{correction}
      
      % réponse : 7c2
    % base 2 vers la base 10
    \item Margaret Hamilton reçoit la médaille présidentielle de la Liberté en $\overline{111\,1110\,0000}^2$ de la part de Barack Obama. Il s'agit de la plus haute distinction civile aux États-Unis. Écrire cette année en base $10$.

      \begin{correction}
        $\overline{111\,1110\,0000}^2=2016$
      \end{correction}
      
      % réponse : 2016
    % base 2 vers la base 10
    \item En 2003, Margaret Hamilton reçoit le \og{}NASA Exceptional Space Act Award for scientific and technical contributions\fg{}. Le montant de la récompense qu'elle reçoit est $\overline{9150}^{16}$. Préciser cette récompense en base $10$.

      \begin{correction}
	$\overline{9150}^{16}=\np{37200}$
      \end{correction}
      
      % réponse : 37 200 dollars
  \end{enumerate}
\end{exercice}

\medskip

\begin{exercice}{1 point}{}
 \begin{enumerate}
   \item Convertir $\overline{1\,0101\,1100}^2$ en base $16$.

     \begin{correction}
       $\overline{1\,0101\,1100}^2=\overline{15\text{C}}^{16}$
     \end{correction}
     
   \item Écrire $\overline{7\text{A}}$ en base $2$.

     \begin{correction}
       $\overline{7\text{A}}=\overline{0111\,1010}^2$
     \end{correction}
     
 \end{enumerate} 
\end{exercice}

\medskip

\begin{exercice}{2 points}{}
 On utilise la méthode du complément à 2 sur un octet :
 \begin{enumerate}
   \item Représenter $59$.

     \begin{correction}
       $59=\overline{0011\,1011}^2$
     \end{correction}
     
   \item Représenter $-93$. 

     \begin{correction}
       $-93=\overline{1010\,0011}^2$ 
     \end{correction}
     
   \item Quel entier est représenté par $\overline{1001\,0100}^2$ ?

     \begin{correction}
       $\overline{1001\,0100}^2=-108$
     \end{correction}
     
 \end{enumerate}
\end{exercice}

\end{document}
