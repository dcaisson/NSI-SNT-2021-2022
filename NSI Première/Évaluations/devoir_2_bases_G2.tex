\documentclass[a4paper,dvipsnames]{article}

\input ../header

\newcommand{\un}{\left(u_n\right)_{n\in\mathbb{N}}}
\newcommand{\uns}{\left(u_n\right)_{n\in\mathbb{N}^\ast}}
\newcommand{\vn}{\left(v_n\right)_{n\in\mathbb{N}}}
\newcommand{\vns}{\left(v_n\right)_{n\in\mathbb{N}^\ast}}
\newcommand{\wn}{\left(w_n\right)_{n\in\mathbb{N}}}
\newcommand{\wns}{\left(w_n\right)_{n\in\mathbb{N}^\ast}}
\newcommand{\tn}{\left(t_n\right)_{n\in\mathbb{N}}}
\newcommand{\tns}{\left(t_n\right)_{n\in\mathbb{N}^\ast}}
\newcommand{\qn}{\left(q_n\right)_{n\in\mathbb{N}}}
\newcommand{\qns}{\left(q_n\right)_{n\in\mathbb{N}^\ast}}
\newcommand{\N}{\mathbb{N}}

\usepackage{dirtree}

\title{Devoir 2 -- Représentation des entiers}

\author{}
\date{}

\begin{document}

\renewcommand{\contentsname}{}

\pagestyle{fancy}

\begin{tcolorbox}[colframe=blue!75, colback=blue!45, valign=center, height=1.5cm, top=5mm]
  \maketitle
\end{tcolorbox}

%\tableofcontents

\vspace{1cm}

\thispagestyle{fancy}

\begin{exercice}{2 points}{}
 \begin{enumerate}
   \item Combien de valeurs différentes peut-on représenter à l'aide de $16$ bits ?
   \item Quel est le plus grand entier naturel qu'on peut représenter à l'aide de $32$ bits ?
 \end{enumerate} 
\end{exercice}

\medskip

\begin{exercice}{4 points}{}
  Voici un court extrait de la biographie de Grace Hopper:

  \begin{center}
    \begin{tabular}{|p{12cm}}
      Grace Murray Hopper, née le 9 décembre 1906 à New York et morte le 1\ier{} janvier 1992 dans le comté d'Arlington, est une informaticienne américaine et Rear admiral (lower half) de la marine américaine. Elle est la conceptrice du premier compilateur en 1951 (A-0 System) et du langage Cobol en 1959.
    \end{tabular}

    \smallskip

    \begin{minipage}{12cm}
      \flushright\textit{Source : Wikipédia}
    \end{minipage}
  \end{center}

  \begin{enumerate}
    % base 10 vers la base 2
    \item Écrire le mois de naissance de Grace Hopper en base $2$.
      % réponse : 1100
    % base 10 vers la base 16
    \item Écrire l'année de l'invention du langage Cobol en base $16$.
      % réponse : 7a7
    % base 2 vers la base 10
    \item Le 22 novembre $\overline{11111100000}^2$, Barack Obama lui décerne à titre posthume la médaille présidentielle de la Liberté. Écrire cette année en base $10$.
      % réponse : 2016
    % base 2 vers la base 10
    \item En $\overline{7\text{b}5}$, Grace Hopper est nommée \og{}membre émérite\fg{} (\textit{distinguished fellow}) de la British Computer Society. Écrire  cette année en base $10$.
      % réponse : 1973
  \end{enumerate}
\end{exercice}

\medskip

\begin{exercice}{1 point}{}
 \begin{enumerate}
   \item Convertir $\overline{1\,1011\,0100}^2$ en base $16$.
   \item Écrire $\overline{\text{F}\text{D}}$ en base $2$.
 \end{enumerate} 
\end{exercice}

\medskip

\begin{exercice}{2 points}{}
 On utilise la méthode du complément à 2 sur un octet :
 \begin{enumerate}
   \item Représenter $49$.
   \item Représenter $-73$. 
   \item Quel entier est représenté par $\overline{1011\,1100}^2$ ?
 \end{enumerate}
\end{exercice}

\end{document}
