\documentclass[a4paper,dvipsnames]{article}

\input ../header

\newcommand{\un}{\left(u_n\right)_{n\in\mathbb{N}}}
\newcommand{\uns}{\left(u_n\right)_{n\in\mathbb{N}^\ast}}
\newcommand{\vn}{\left(v_n\right)_{n\in\mathbb{N}}}
\newcommand{\vns}{\left(v_n\right)_{n\in\mathbb{N}^\ast}}
\newcommand{\wn}{\left(w_n\right)_{n\in\mathbb{N}}}
\newcommand{\wns}{\left(w_n\right)_{n\in\mathbb{N}^\ast}}
\newcommand{\tn}{\left(t_n\right)_{n\in\mathbb{N}}}
\newcommand{\tns}{\left(t_n\right)_{n\in\mathbb{N}^\ast}}
\newcommand{\qn}{\left(q_n\right)_{n\in\mathbb{N}}}
\newcommand{\qns}{\left(q_n\right)_{n\in\mathbb{N}^\ast}}
\newcommand{\N}{\mathbb{N}}

\usepackage{dirtree}

\title{Devoir 3 -- Complément à 2, programmation Python}

\author{}
\date{}

\begin{document}

\renewcommand{\contentsname}{}

\pagestyle{fancy}

\begin{tcolorbox}[colframe=blue!75, colback=blue!45, valign=center, height=1.5cm, top=5mm]
  \maketitle
\end{tcolorbox}

%\tableofcontents

\vspace{1cm}

\thispagestyle{fancy}

\begin{exercice}{4 points}{}
  Cet exercice est un QCM. Pour chaque question, une seule des réponses proposées est correcte. Une mauvaise réponse, l'absence de réponse ou choisir plusieurs réponses ne rapporte, ni n'enlève aucun point.
  Écrire sur votre copie le numéro de la question ainsi que la réponse choisie.
  \begin{enumerate}
    \item On considère les instructions suivantes :
      \begin{minted}[xleftmargin=1cm,xrightmargin=1cm]{python}
ch = "Muad'Dib"
c = ch[4]
      \end{minted}
      Que contient la variable \mintinline{python}{c} ?
      \vspace*{-2mm}
      \begin{multicols}{4}
        \begin{enumerate}
	  \item \mintinline{python}{"a"}
	  \item \mintinline{python}{"d"}
	  \item \mintinline{python}{"'"}
	  \item \mintinline{python}{"D"}
        \end{enumerate}
      \end{multicols}
    \item On considère l'instruction \mintinline{python}{liste_mystere = [k + 3 for i in range(4)]}. Que contient la variable \mintinline{python}{liste_mystere} ?
      \vspace*{-2mm}
      \begin{multicols}{2}
        \begin{enumerate}
	  \item \mintinline{python}{[0, 1, 2, 3, 4]}
	  \item \mintinline{python}{[0, 1, 2, 3]}
	  \item \mintinline{python}{[3, 4, 5, 6, 7]}
	  \item \mintinline{python}{[3, 4, 5, 6]}
        \end{enumerate}
      \end{multicols}
    \item On considère les instructions suivantes :
      \begin{minted}[xleftmargin=1cm,xrightmargin=1cm]{python}
ch = "Les cymek sont des cyborgs."
c = ch[-1]
      \end{minted}
      Que contient la variable \mintinline{python}{c} ?
      \vspace*{-2mm}
      \begin{multicols}{2}
        \begin{enumerate}
	  \item \mintinline{python}{"L"}
	  \item \mintinline{python}{"."}\columnbreak
	  \item \mintinline{python}{"s"}
	  \item Rien, le code donné produit une erreur.
        \end{enumerate}
      \end{multicols}
    \item On considère les instructions suivantes :
      \begin{minted}[xleftmargin=1cm,xrightmargin=1cm]{python}
prenom = ["Distille", "Épice", "Feydakin", "Gemmone"]
p = prenom[3][2]
      \end{minted}
      Que contient la variable \mintinline{python}{p} ?
      \vspace*{-2mm}
      \begin{multicols}{4}
        \begin{enumerate}
	  \item \mintinline{python}{"d"}
	  \item \mintinline{python}{"m"}
	  \item \mintinline{python}{"e"}
	  \item \mintinline{python}{"i"}
        \end{enumerate}
      \end{multicols}
  \end{enumerate}
\end{exercice}

\medskip

\begin{exercice}{3 points}{}
 On utilise la méthode du complément à 2 sur un octet :
 \begin{enumerate}
   \item Représenter $39$.
   \item Représenter $-37$. 
   \item Quel entier est représenté par $\overline{1000\,1100}^2$ ?
 \end{enumerate}
\end{exercice}

\medskip

\begin{exercice}{3 points}{}
  On souhaite écrire une fonction \mintinline{python}{alea(a, b, n)} qui renvoie une liste de taille \mintinline{python}{n} contenant des entiers tirés au hasard entre deux entiers \mintinline{python}{a} et \mintinline{python}{b} (ces deux entiers étant inclus).
  \begin{enumerate}
    \item Donner un exemple de liste renvoyée par l'appel \mintinline{python}{alea(12, 20, 4)}.
    \item Écrire la fonction \mintinline{python}{alea(a, b, n)}.
  \end{enumerate}
\end{exercice}

\medskip

\begin{exercice}{3,5 points}{}
  Écrire une fonction \mintinline{python}{somme(L)} qui renvoie la somme des éléments d'une liste d'entiers \mintinline{python}{L} donnée en paramètre.
  \begin{enumerate}
    \item Proposer trois tests pour la fonction \mintinline{python}{somme(L)} en utilisant \mintinline{python}{assert}.
    \item Écrire la fonction \mintinline{python}{somme}.
    \item En utilisant la fonction \mintinline{python}{somme}, écrire une fonction \mintinline{python}{moyenne} qui prend en paramètre une liste d'entiers \mintinline{python}{L} et qui renvoie la moyenne de ces entiers.
  \end{enumerate}
\end{exercice}

\medskip

\begin{exercice}{3,5 points}{}
  Écrire une fonction \mintinline{python}{choix(L)} qui prend en paramètre une liste de longueur supérieure ou égale à $3$, et qui renvoie une liste de deux éléments (distincts) de \mintinline{python}{L} choisis au hasard. Par exemple :

  \begin{center}
    \mintinline{python}{choix(["Naib", "Powindah", "Sardaukar", "Sayyadina"])} pourrait renvoyer la liste \mintinline{python}{["Sayyadina", "Naib"]}.
  \end{center}

  Modifier la fonction en une fonction \mintinline{python}{choix(L, n)} afin qu'elle renvoie une liste de \mintinline{python}{n} éléments (distincts) de \mintinline{python}{L} choisis au hasard, où \mintinline{python}{n} est un entier non nul inférieur ou égal à la longueur de la liste \mintinline{python}{L}. Par exemple :

  \begin{center}
    \mintinline{python}{choix(["Naib", "Sietch", "Powindah", "Sardaukar", "Sayyadina"], 4)} pourrait renvoyer la liste \mintinline{python}{["Sayyadina", "Naib", "Sietch", "Powindah"]}.
  \end{center}
\end{exercice}

\medskip

\begin{exercice}{3 points}{}
  Écrire une fonction \mintinline{python}{recherche(chaine, c)} qui prend en paramètre une chaîne de caractères non vide \mintinline{python}{chaine} et un caractère \mintinline{python}{c}, et qui renvoie l'\textbf{indice} de la dernière occurrence de \mintinline{python}{c}. Si le caractère n'est pas présent dans la chaîne de caractères, la fonction renvoie la longueur de la chaîne.

  \tcblower

  Quelques exemples :

  \begin{itemize}
    \item \mintinline{python}{recherche("Moissonneuse", "o")} renvoie \mintinline{python}{5} ;
    \item \mintinline{python}{recherche("Moissonneuse", "e")} renvoie \mintinline{python}{11} ;
    \item \mintinline{python}{recherche("Moissonneuse", "l")} renvoie \mintinline{python}{11}.
  \end{itemize}
\end{exercice}

\end{document}
