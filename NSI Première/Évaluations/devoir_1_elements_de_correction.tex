\documentclass[a4paper,dvipsnames]{article}

\input ../header

\newcommand{\un}{\left(u_n\right)_{n\in\mathbb{N}}}
\newcommand{\uns}{\left(u_n\right)_{n\in\mathbb{N}^\ast}}
\newcommand{\vn}{\left(v_n\right)_{n\in\mathbb{N}}}
\newcommand{\vns}{\left(v_n\right)_{n\in\mathbb{N}^\ast}}
\newcommand{\wn}{\left(w_n\right)_{n\in\mathbb{N}}}
\newcommand{\wns}{\left(w_n\right)_{n\in\mathbb{N}^\ast}}
\newcommand{\tn}{\left(t_n\right)_{n\in\mathbb{N}}}
\newcommand{\tns}{\left(t_n\right)_{n\in\mathbb{N}^\ast}}
\newcommand{\qn}{\left(q_n\right)_{n\in\mathbb{N}}}
\newcommand{\qns}{\left(q_n\right)_{n\in\mathbb{N}^\ast}}
\newcommand{\N}{\mathbb{N}}

\newenvironment{correction}{\color{blue}}{}

\title{Devoir à rendre 1}

\author{}
\date{}

\begin{document}

\renewcommand{\contentsname}{}

\pagestyle{fancy}

\begin{tcolorbox}[colframe=blue!75, colback=blue!45, valign=center, height=1.5cm, top=5mm]
  \maketitle
\end{tcolorbox}

%\tableofcontents

\vspace{1cm}

\thispagestyle{fancy}

\begin{exercice}{}{}
  Soient $\un$ et $\vns$ les deux suites définies par :
  \begin{itemize}
    \item pour tout $n$ appartenant à $\N$, $u_n=10n^2-3n-7$ ;
    \item $v_1=6$ et, pour tout $n$ appartenant à $\N$, $v_{n+1}=-3v_n+10$.
  \end{itemize}
  \begin{enumerate}
    \item De quelle façon est définie la suite $\un$ ? la suite $\vns$ ?

      \begin{correction}
	La suite $\un$ est définie de façon explicite tandis que la suite $\vns$ est définie par récurrence.
      \end{correction}

    \item Calculer $u_5$ et $v_4$.

      \begin{correction}
	\begin{itemize}
	  \item On a $u_5=10\times5^2-3\times5-7=228$.
	  \item On trouve successivement $v_2=-3v_1+10=-3\times6+10=-8$, $v_3=34$ et $v_4=-92$.
	\end{itemize}
      \end{correction}
  \end{enumerate}
\end{exercice}

\bigskip

\begin{exercice}[breakable]{}{}
  Soit $\wns$ la suite définie par $w_n=\dfrac{n^2+1}{2n^2}$ pour tout $n$ appartenant à $\N^\ast$.
  \begin{enumerate}
    \item Étudier le sens de variation de la suite $\wns$.

      \begin{correction}
	On utilise la méthode vue en cours. Soit $n$ un entier naturel non nul.  
	\[
	  \begin{aligned}
	    w_{n+1}-w_n &= \dfrac{(n+1)^2+1}{2(n+1)^2}-\dfrac{n^2+1}{2n^2}\\
			&= \dfrac{n^2+2n+2}{2(n+1)^2}-\dfrac{n^2+1}{2n^2}\\
			&= \dfrac{(n^2+2n+2){\color{red}\times n^2}}{2(n+1)^2{\color{red}\times n^2}}-\dfrac{(n^2+1){\color{red}\times(n+1)^2}}{2n^2{\color{red}\times(n+1)^2}}\\
			&= \dfrac{(n^2+2n+2)\times n^2-(n^2+1)\times(n+1)^2}{2(n+1)^2\times n^2}\\
			&= \dfrac{n^4+2n^3+2n^2-(n^2+1)(n^2+2n+1)}{2(n+1)^2\times n^2}\\
			&= \dfrac{n^4+2n^3+2n^2-(n^4+2n^3+n^2+n^2+2n+1)}{2(n+1)^2\times n^2}\\
			&= \dfrac{n^4+2n^3+2n^2-n^4-2n^3-n^2-n^2-2n-1}{2(n+1)^2\times n^2}\\
			&= \dfrac{-2n-1}{2(n+1)^2\times n^2}\\
			&= \dfrac{-(2n+1)}{2(n+1)^2\times n^2}
	  \end{aligned}
	\]
	On remarque que :
        \begin{itemize}
          \item $-1<0$ ;
	  \item $2n+1$ est la somme de deux nombres positifs donc $2n+1>0$ ;
	  \item $2>0$ ;
	  \item $(n+1)^2>0$;
	  \item $\color{blue}n^2>0$,
        \end{itemize}

	\vphantom{1}\color{blue}donc $\dfrac{-(2n+1)}{2(n+1)^2\times(2n^2)}<0$.\\
	Pour tout entier naturel non nul $n$, $w_{n+1}-w_n<0$ donc la suite $\wns$ est strictement décroissante.
      \end{correction}
      
    \item (bonus) Montrer que pour tout entier naturel non nul, $w_n\leq 1$.

      \textit{Indication : on pourra étudier le signe de $w_n-1$ (où $n\in\N^\ast$).}

      \begin{correction}
	Soit $n$ un entier naturel non nul.
	\[
	  \begin{aligned}
	    w_n - 1 &= \dfrac{n^2+1}{2n^2}-1\\
		    &= \dfrac{n^2+1}{2n^2}-\dfrac{2n^2}{2n^2}\\
		    &= \dfrac{n^2+1-2n^2}{2n^2}\\
		    &= \dfrac{1-n^2}{2n^2}\\
		    &= \dfrac{(1-n)(1+n)}{2n^2}
	  \end{aligned}
	\]
	On remarque que :
	\begin{itemize}
	  \item $n\geq 1$ donc $1-n\leq 0$;
	  \item $n+1$ est la somme de deux nombres positifs donc $n+1\geq 0$;
	  \item $2>0$ ;
	  \item $n^2\geq 0$,
	\end{itemize}
	donc $\dfrac{(1-n)(1+n)}{2n^2}\leq 0$. Par conséquent $w_n-1\leq 0$, et on en déduit $w_n\leq 1$.

	On a donc montré que, pour tout entier naturel $n$ non nul, $w_n\leq 1$.
      \end{correction}
  \end{enumerate}
\end{exercice}

\end{document}
