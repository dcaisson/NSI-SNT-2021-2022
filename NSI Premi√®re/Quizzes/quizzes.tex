\documentclass{beamer}

\usetheme[numbering=none]{metropolis}

\input ../../header_beamer

\newcommand{\N}{\mathbb{N}}
\newcommand{\Ns}{\mathbb{N}^\ast}
\newcommand{\un}{(u_n)_{n\in\N}}
\newcommand{\uns}{(u_n)_{n\in\Ns}}
\newcommand{\vn}{(v_n)_{n\in\N}}
\newcommand{\vns}{(v_n)_{n\in\Ns}}
\newcommand{\wn}{(w_n)_{n\in\N}}
\newcommand{\wns}{(w_n)_{n\in\Ns}}

\setsansfont[BoldFont={Fira Sans SemiBold}]{Fira Sans Book}

\usepackage{xcolor}
%\definecolor{bleudane}{RGB}{5,100,160}
%\definecolor{bleudane2}{RGB}{25,134,193}
\definecolor{bleu}{RGB}{101,126,228}
\setbeamercolor{frametitle}{bg=bleu}
\setbeamercolor{normal text}{fg=black}

\metroset{titleformat=smallcaps}

\newenvironment{correction}{\pause{}\color{blue}}{\pause{}}

\begin{document}

\section{Quizz 1}

\begin{frame}
  \frametitle{Quizz 1}
  \begin{enumerate}
    \item On considère la liste \mintinline{python}{L = [3, 4, 1, 2, 5]}. Que vaut \mintinline{python}{L[-1]} ?
    \item En utilisant une boucle, créer une liste contenant $10$ zéros.
    \item Écrire une fonction \mintinline{python}{minimum()} qui prend en paramètre une liste de nombres et qui renvoie la valeur minimal de ces nombres. On n'utilisera pas la fonction \mintinline{python}{min()} de Python.
    \item Écrire une fonction \mintinline{python}{position_minimum()} qui prend en paramètre une liste de nombres et qui renvoie la position de la valeur minimale de ces nombres.
  \end{enumerate}
\end{frame}

\section{Quizz 2}

\begin{frame}
  \frametitle{Quizz 2}
  \begin{enumerate}
    \item En utilisant une boucle, créer la liste des premiers cubes jusqu'à $125$ inclus : \mintinline{python}{[0, 1, 8, 27, 64, 125]}.\pause{}
    \item Créer la même liste par compréhension.\pause{}
    \item Écrire une fonction \mintinline{python}{moyenne_ponderee(notes, coefficients)} qui prend en paramètres :
      \begin{itemize}
	\item une liste de notes \mintinline{python}{notes} ;
	\item une liste de coefficients \mintinline{python}{coefficients},
      \end{itemize}
      et qui renvoie la moyenne pondérée des notes pondérées par les coefficients.
  \end{enumerate}
\end{frame}

\section{Quizz 3}

\begin{frame}[fragile]
  \frametitle{Quizz 3}
  \begin{enumerate}
    \item On considère la liste ci-dessous :
      \begin{minted}{python}
liste_1 = [
      ["Maryam", "Alexandre", "Jacques"],
      ["Joseph", "Michael", "Jean-Pierre"],
      ["Pierre", "Leonhard", "Henri"],
      ["David", "Paul", "Pierre-Simon"]
     ]
      \end{minted}
      \begin{enumerate}
	\item[--] Que vaut \mintinline{python}{liste_1[3][2]} ?
	\item[--] Comment peut-on obtenir, de manière similaire, \mintinline{python}{"Paul"} ?
      \end{enumerate}
  \end{enumerate}
\end{frame}

\begin{frame}[fragile]
  \begin{enumerate}
    \setcounter{enumi}{1}
    \item En utilisant une ou plusieurs boucles, créer la liste suivante :
      \begin{minted}{python}
liste_2 = [
      [1, 2, 3, 4, 5, 6, 7, 8],
      [2, 4, 8, 10, 12, 14, 16],
      [3, 6, 9, 12, 15, 18, 21],
      [4, 8, 12, 16, 20, 24, 28, 32]
]
      \end{minted}
  \end{enumerate}
\end{frame}

\section{Quizz 4}

\begin{frame}
  \frametitle{Quizz 4}
  \begin{enumerate}
    \item Écrire une fonction \mintinline{python}{contient_valeur()} qui prend en paramètres une liste d'entiers et un entier \mintinline{python}{n}, et qui renvoie \mintinline{python}{True} si \mintinline{python}{n} figure dans la liste, \mintinline{python}{False} sinon.
    \item Écrire une fonction \mintinline{python}{positions_valeur()} qui prend en paramètres une liste d'entiers et un entier \mintinline{python}{n}, et qui renvoie la liste des positions auxquelles \mintinline{python}{n} apparaît dans la liste (la fonction renvoie une liste vide si l'entier \mintinline{python}{n} n'apparaît pas dans la liste).
  \end{enumerate}
\end{frame}

\begin{frame}[fragile]
  \begin{enumerate}
    \setcounter{enumi}{2}
  \item Préciser \mintinline{python}{liste} après les lignes suivantes :
      \begin{minted}{python}
liste = []
for i in range(3):
    ligne = []
    for j in range(4):
        ligne.append(i - j) 
    liste.append(ligne)
      \end{minted}
  \end{enumerate}
\end{frame}
\end{document}

%%% Local Variables:
%%% mode: latex
%%% TeX-master: t
%%% End:
